\section{Helicity amplitudes}
\label{sect:helampconventions}
\index{helicity amplitudes}

This section will deal with some details related to helicity amplitudes
and their relations to partial wave amplitudes. In particular 
the sign conventions are described and
the relations between the Jackson~\cite{Jack64} and the 
Jackob-Wick~\cite{Jaco59} conventions
for the helicity formalism is explained.

This section is not meant as a complete guide to the use of
helicity amplitudes. There are several references that gives a 
good introduction to the use of helicity amplitudes for describing
the dynamics of particle decays. Richman~\cite{Rich84} gives
a pedagogical introduction following the conventions of
Jacob and Wick~\cite{Jaco59}. Jackson~\cite{Jack64} uses a slightly
different choice of conventions. One of the main purposes of this section is
to describe these conventions and establish a map between the two
conventions. Both of these conventions are used in the literature
and, unfortunately, it is not always clear which convention is used. 

The origin of the choice of conventions comes from how the 
Euler angles are chosen. To make sure that there is no ambiguity
about definitions and conventions the next section provides definitions
for the terminology used. 

\subsection{Preliminaries and definitions}

A few definitions are stated explicitly here as to avoid
confusion about what conventions are used.

First, the basis vectors $\hat x$, $\hat y$, and $\hat z$ are
said to form a right handed coordinate system if rotating the
$\hat x$-axis 90 degrees counter-clockwise, as seen from the positive
$\hat z$ directions, will take it to the direction of the $\hat y$
directions. It is assumed that $\hat x$, $\hat y$, and $\hat z$ are
mutually orthogonal. If the coordinate system is not right handed it is 
left handed -- there is no other alternative. In the discussion
below all coordinate systems considered are right handed.

The Euler angles, $\alpha$, $\beta$, and $\gamma$ defines a 
rotation $R(\alpha,\beta,\gamma)$ by
\begin{equation}
R(\alpha,\beta,\gamma)\equiv R_{z'}(\gamma)R_{y'}(\beta)R_{z}(\alpha)
\end{equation}
where this means that a rotation by $\alpha$ is first performed around the 
$z$-axis. Then a rotation by $\beta$ around the $y'$-axis is done. Where
the $y'$-axis is the new axis as obtained after the first rotation.
Last the rotation around the new $z$-axis, $z'$, is performed by an
amount given by $\gamma$. Note that these rotations are not according to
a fixed set of rotation axis. This is inconvenient as the rotation operators
that we have are with respect to a fixed coordinate system. However, 
there is a simple way of rewriting the Euler rotation in terms of
rotations around a fixed coordinate system,
\begin{equation}
R(\alpha,\beta,\gamma)=R_{z}(\alpha)R_{y}(\beta)R_{z}(\gamma).
\end{equation}
See e.g. Sakurai~\cite{Sakurai} page 171-174 for illustrations of the
rotations.

The $D^{J}_{m,m'}(\alpha,\beta,\gamma)$ functions are defined by
\begin{equation}
D^{J}_{m,m'}(\alpha,\beta,\gamma)\equiv 
             \langle Jm|R(\alpha,\beta,\gamma)|Jm'\rangle.
\end{equation}
Using $d^{J}_{m,m'}(\beta)\equiv \langle Jm|e^{-i\beta J_y}|Jm'\rangle$\$ we
can write
\begin{equation}
D^{J}_{m,m'}(\alpha,\beta,\gamma)=
       e^{-im\alpha}d^{J}_{m,m'}(\beta)e^{-im'\gamma}.
\end{equation}

\subsection{Plane wave states}

The state $\Psi_{p\lambda}$ denotes a state with momentum $p$
along the $z$-axis with helicity $\lambda$. This state is obtained
by applying a boost $L(p)$ along the $z$-axis to the state 
$|J=s\quad m=\lambda\rangle$, where $|J m\rangle$ are the 
canonical angular momentum states. The total angular momentum,
$s$, is suppressed in the notation below.

Following Jacob and Wick~\cite{Jaco59} we define the states $\chi_{p\lambda}$
which have momentum $p$ along the negative $z$-direction
\begin{equation}
\chi_{p\lambda}=(-1)^{s-\lambda}e^{-i\pi J_y}\Psi_{p\lambda}.
\end{equation}

First we will verify that this state has the properties that
we expect; 

\begin{eqnarray}
J_z\chi_{p\lambda} & = & J_z(-1)^{s-\lambda}e^{-i\pi J_y}\Psi_{p\lambda}\nonumber\\
                   & = & (-1)^{s-\lambda}e^{-i\pi J_y} e^{i\pi J_y} 
                          J_z e^{-i\pi J_y}\Psi_{p\lambda}\nonumber\\
                   & = & (-1){s-\lambda}e^{-i\pi J_y}(-1)J_z\Psi_{p\lambda}\nonumber\\
                   & = & -\lambda \chi_{p\lambda}
\end{eqnarray}
shows that the eigen value of $J_z$ is $-\lambda$ as expected. Further, we look
at the application of the lowering operator, $J_{-}=J_x-iJ_y$, on the 
state $\chi_{p\lambda}$
\begin{eqnarray}
J_-\chi_{p\lambda} & = & (-1)^{s-\lambda}(J_x-iJ_y)e^{-i\pi J_y}\Psi_{p\lambda}\nonumber\\
                   & = & (-1)^{s-\lambda}e^{-i\pi J_y}e^{i\pi J_y}(J_x-iJ_y)
                                         e^{-i\pi J_y}\Psi_{p\lambda}\nonumber\\
                   & = & (-1)^{s-\lambda}e^{-i\pi J_y}(-J_x-iJ_y)\Psi_{p\lambda}\nonumber\\
                   & = & -(-1)^{s-\lambda}e^{-i\pi J_y}J_+\Psi_{p\lambda}\nonumber\\
                   & = & -(-1)^{s-\lambda}e^{-i\pi J_y}\sqrt{(s-\lambda)(s+\lambda+1)}
                            \Psi_{p\lambda+1}\nonumber\\
	           & = & \sqrt{(s-\lambda)(s+\lambda+1)}(-1)^{s-(\lambda+1)}
                         e^{-i\pi J_y}\Psi_{p\lambda+1}
\end{eqnarray}
which shows that the application of $J_-$ on $\Psi_{p\lambda}$ behaves
as expected for a particle with helicity $\lambda$ along the 
negative $z$-direction. Note in particular that the factor $(-1)^{\lambda}$
is important to ensure the right phase when applying the lowering
operators. 
In fact, the properties that we have demonstrated above almost shows
the following property. If $p=0$, i.e. the particle is at rest, then
\begin{equation}
\chi_{0\lambda}=\Psi_{0-\lambda}.
\label{eq:statesame}
\end{equation}
The properties of the lowering
and raising operators and $J_z$ that we have shown above shows that the
states $\chi_{0\lambda}$ and $\Psi_{0-\lambda}$ differs at most by a 
phase and that this phase is independent of $\lambda$. However, it
is easiest to show Eq.~\ref{eq:statesame} using the Wigner 
$d$-functions
\begin{equation}
\chi_{0\lambda}=(-1)^{s-\lambda}e^{-i\pi J_y}\Psi_{0\lambda}
 = (-1)^{s-\lambda}\sum_{\lambda'}d^s_{\lambda'\lambda}(\pi)\Psi_{0\lambda'}
 = \Psi_{0-\lambda}
\end{equation}
since
\begin{equation}
d^s_{mm'}(\pi)=(-1)^{s-\lambda}\delta_{m,-m'}.
\end{equation}

An alternative way to define the states $\chi_{p\lambda}$ is by
\begin{equation}
\chi_{p\lambda}=(-i)^{2s}e^{-i\pi J_x}\Psi_{p\lambda}.
\label{eq:newdef}
\end{equation}
we find that
\begin{equation}
\chi_{0\lambda}=(-i)^{2s}e^{-i\pi J_x}\Psi_{0\lambda}=
(-i)^{2s}\sum_{\lambda'}\tilde d^s_{\lambda'\lambda}(\pi)\Psi_{0\lambda'}=\Psi_{0-\lambda}
\end{equation}
where
\begin{equation}
\tilde d^s_{m'm}(\theta)=\langle j m'|e^{-i\theta J_x}|j m\rangle.
\end{equation}
The $\tilde d$ function is similar to the standard Wigner $d$ function 
except that it refers to expectation values for rotations around the 
$x$-axis instead of the $y$-axis. An explicit formula for the 
$\tilde d$ function is given by
\begin{eqnarray}
\tilde d^j_{m'm}(\theta)& = & \nonumber
\sum {\sqrt{(j+m)!(j-m)!(j+m')!(j-m')!}\over (m+m'+k)!(j-m'-k)!k!(j-m-k)!}\times\\
& &\left(\cos{\theta\over 2}\right)^{m'+m+2k}
\left(-i\sin{\theta\over 2}\right)^{2j-2k-m'-m}
\end{eqnarray}
from which it is easy to show that
\begin{equation}
\tilde d^j_{m'm}(\pi)=(-1)^{2j}\delta_{m',-m}
\end{equation}

It is straight forward to show that the application of the raising and
lowering operators as well as the operators $J_z$ on the state defined
by Eq.~\ref{eq:newdef} is what is expected.
Note in particular that there is no longer any need for phase factors
to get the right phase on the different helicity states. This is
particularly useful because it allows us to simply interpret
the states $\chi_{p\lambda}$ in the rest frame of the particle with the
momentum $p$ along the negative $z$ direction. That is, the state
$\chi_{p\lambda}$ is equal to $|J=s\quad m=\lambda\rangle$ 
in a coordinate system that 
has been rotated by $\pi$ around the $x$ axis. This is very important
because it tells us how to construct the coordinates systems in
sequential decays when applying the helicity formalism, this will
be discussed further below.

We now define the two particle plane wave states when particle
A is along the positive $z$-direction by
\begin{equation}
|p\lambda_{A}\lambda_{B}\rangle=\Psi_{p\lambda_{A}}\chi_{p\lambda_{B}}.
\end{equation}
Applying $J_z$ to this state we find 
\begin{equation}
J_z|p\lambda_{A}\lambda_{B}\rangle=
(\lambda_A-\lambda_B)|p\lambda_{A}\lambda_{B}\rangle
\end{equation}
States where the relative momentum, $p$, is not along the $z$-axis are
obtained by rotating the state $|p\lambda_{A}\lambda_{B}\rangle$.
Let $\theta$ and $\phi$ denote the polar coordinates for particle A, then
the we define the state $|p\theta\phi\lambda_{A}\lambda_{B}\rangle$ by
\begin{eqnarray}
|p\theta\phi\lambda_{A}\lambda_{B}\rangle=R(\alpha,\beta,\gamma)
             |p\lambda_{A}\lambda_{B}\rangle
\end{eqnarray}
where $R(\alpha,\beta,\gamma)$ is an Euler rotation. Here there is a 
choice of conventions. Jackson~\cite{Jack64} takes 
$\alpha=\phi$, $\beta=\theta$, and $\gamma=0$ while Jacob 
and Wick~\cite{Jaco59} uses $\alpha=\phi$, $\beta=\theta$, and $\gamma=-\phi$.
The choice of the angle $\gamma$ is arbitrary but has to be used 
consistently. 

We have now defined the state $|p\theta\phi\lambda_A\lambda_B\rangle$
and explored its properties in some detail. This state is used as the
final state in the two body decays in the helicity formalism. 

Before we are ready to use these states we need to construct a set
of states that are labeled by $p$, $\lambda_A$, and $\lambda_B$ and
have definite values of $J$ and $m$, i.e., are eigenstates of
$J^2$ and $J_z$. That such states exists is obvious since $p$,
$\lambda_A$, and $\lambda_B$ are invariant under rotations. We 
denote these states by $|pJM\lambda_A\lambda_B\rangle$. The relation
between these states and the plane wave states created above is given by
\begin{equation}
|pJM\lambda_A\lambda_B\rangle=\sqrt{{2J+1}\over{ 4\pi}}
\int_{d\Omega}D^{*J}_{M\lambda_A-\lambda_B}(\phi,\theta,-\phi)
|p\theta\phi\lambda_A\lambda_B\rangle.
\label{eq:planewaveexpansion}
\end{equation}
Note that the states $|pJM\lambda_A\lambda_B\rangle$ are independent 
of the choice of Euler angles.


\subsection{Helicity amplitudes}

We will now consider the decay $C\rightarrow A+B$. The initial
particle, $C$, is assumed to be in the state $|J M\rangle$ and the 
final state, $A+B$, is $|pJM\lambda_A\lambda_B\rangle$. We will assume
that the interaction $U$ that causes this transition is invariant
under rotation, but is otherwise arbitrary. We which to evaluate
the matrix element 
\begin{equation}
M=\langle pJM\lambda_A\lambda_B\rangle | U | J M \rangle.
\end{equation}
To do this we insert the identity written as
\begin{equation}
I=\sum_{p'J'M'\lambda_A'\lambda_B'}
|p'J'M'\lambda_A'\lambda_B'\rangle\langle p'J'M'\lambda_A'\lambda_B'|
\end{equation}
and use Eq.~\ref{eq:planewaveexpansion}.
\begin{eqnarray}
M & = & \langle p\theta\phi \lambda_A \lambda_B | U | J M \rangle\nonumber \\
  & = & \sum_{p'J'M' \lambda_A' \lambda_B'}
        \langle p\theta\phi \lambda_A \lambda_B |
        p'J'M'\lambda_A'\lambda_B'\rangle\langle p'J'M'\lambda_A'\lambda_B'|
        U | J M \rangle \nonumber\\
  & = & \langle p\theta\phi \lambda_A \lambda_B |
        pJM\lambda_A\lambda_B\rangle\langle pJM\lambda_A\lambda_B|
        U | J M \rangle \nonumber\\
  & = & \sqrt{2J+1\over 4\pi}\langle p\theta\phi \lambda_A \lambda_B |
       \int_{d\Omega'}D^{*J}_{M,\lambda_A-\lambda_B}(\phi',\theta',-\phi')
       |p\theta'\phi'\lambda_A\lambda_B\rangle
       \langle pJM\lambda_A\lambda_B|U | J M \rangle\nonumber \\
  & = & \sqrt{2J+1\over 4\pi}D^{*J}_{M,\lambda_A-\lambda_B}(\phi,\theta,-\phi)
        H_{\lambda_A\lambda_B}
\label{eq:helicityformula}
\end{eqnarray}
where the helicity amplitudes are defined by 
\begin{equation}
H_{\lambda_A\lambda_B}=\langle pJM\lambda_A\lambda_B|U | J M \rangle. 
\end{equation}



\subsection{Helicity amplitudes and sequential decays}

The previous section explained how the helicity formalism 
is used to calculate the amplitudes for a two body decay, $A\rightarrow B+C$.
It is straight forward to now to use this in sequential decays.
The only thing that requires a it of care is the construction 
of the coordinate systems in which the decay angles are measured.

To explain how to use the helicity formalism in sequential decays 
we will consider the decay chain $A\rightarrow B+C$, $B\rightarrow D+E$, 
and $C\rightarrow F+G$. The initial particle, $A$, is in the 
state $|J=J_A\quad m=\lambda_A\rangle$. The amplitudes for
the decay $A\rightarrow B+C$ is now given by
\begin{equation}
A^{A\rightarrow B+C}_{\lambda_A\lambda_B\lambda_C}=
\sqrt{2J_A+1\over 4\pi} D^{*J_A}_{\lambda_A,\lambda_B-\lambda_C}
(\phi_B,\theta_B,-\phi_B)H^{A}_{\lambda_B\lambda_C}
\end{equation}
where $\theta_B$ and $\phi_B$ are the polar angles of particle
$B$ in the rest frame of particle $A$. 

Similarly, we can write the amplitude for the decay of particle $B$
\begin{equation}
A^{B\rightarrow D+E}_{\lambda_B\lambda_D\lambda_E}=
\sqrt{2J_B+1\over 4\pi} D^{*J_B}_{\lambda_B,\lambda_D-\lambda_E}
(\phi_D,\theta_D,-\phi_D)H^{B}_{\lambda_D\lambda_E}.
\end{equation}
The coordinate system in which the angles $\theta_D$ and $\phi_D$
are measured is obtained by rotating the coordinate system of the
parent particle, $A$, using the same Euler angles as was used
when calculating the amplitude for the decay of particle $A$.
This means that the coordinate system for particle $B$ is obtained
by doing the rotation $R(\phi_B,\theta_B,\phi_B)$ of the 
coordinate system of particle $A$.

In the same way we obtain the amplitude for the decay of particle
$C$,
\begin{equation}
A^{C\rightarrow F+G}_{\lambda_C\lambda_F\lambda_G}=
\sqrt{2J_C+1\over 4\pi} D^{*J_C}_{\lambda_C,\lambda_F-\lambda_G}
(\phi_F,\theta_F,-\phi_F)H^{C}_{\lambda_F\lambda_G}.
\end{equation}
As discussed above the coordinate system for the second particle,
here particle $C$ in the decay of particle $A$, is obtained by rotating
the coordinates system of the first particle, $B$, by $\pi$ around
its $x$ axis. I.e. the $x$-axis of particles $B$ and $C$ frames are
parallel. 

\subsubsection{Jackson convention}

In the Jackson convention the Euler rotation is taken to be
$R(\phi,\theta,0)$ where the angles $(\theta,\phi)$ are the 
polar coordinates for $\vec p$. This is a 
rotation first by $\phi$ around the 
$z$-axis and then a rotation by $\theta$ around the new $y$-axis.
The new $y$-axis is in the direction of $z\times \vec p$ and 
the new $x$-axis is therefore in the direction $(z\times \vec p)\times \vec p$.

This is a simple geometrical construction that allows the construction 
of the coordinate system used with the Jackson convention for the
Euler rotations.

\subsubsection{Jacob-Wick convention}

The coordinate system used in the Jacob-Wick convention is obtained by
performing an additional rotation of $-\phi$ around the new $z$-axis
as obtained in the Jackson conventions. 

In the sense that there is a simple geometrical construction for the 
coordinate system in the Jackson convention if might be argued that
this convention is somewhat simpler to use. 

%Consider the decay $A\rightarrow BC$. The particle $A$ is
%described by the state $|J,m_z=\lambda_A\rangle$ in the 
%coordinate system $x$, $y$, and $z$. As conventional, the 
%$z$ axis has been chosen as the quantization axis.
%
%\begin{figure}
%\begin{center}
%\psfig{figure=helamp.eps,height=2.5in}
%\caption{The state of the initial particle, A, with total angular
%momentm $J$ and spin projection $\lambda_A=m_z$ along the
%$z$-axis is labeled by $|J,m_z=\lambda_A\rangle$ in the 
%coordinate system shown in the figure. The decay products, B and C,
%are back to back in the rest frame of particle A and the
%angles $(\theta_A,\phi_A)$ and $(\pi-\theta_A,\pi+\phi_A)$
%gives the polar coordinates for B and C respectively. 
%\label{fig:helamp}}
%\end{center}
%\end{figure}
%
%Particle $B$, produced in the decay, will have a direction
%which will be labeled $\hat p$. The polar coordinates for 
%this direction, as measured in $A$'s coordinate system will
%be labeled $\theta_A$ and $\phi_A$, see Figure~\ref{fig:helamp}. 
%To write down the
%amplitude for this decay a choice of Euler angles has to be
%made that rotates the original $z$ axis to the directions $\hat p$.
%This can be accomplished by chosing $\alpha=\phi_A$ and
%$\beta=\theta_A$. However, the azimuthal angle $\gamma$ around 
%the new $z$ axis, $z'$, is arbitrary. This is the source 
%of conventions for the helicity formalism. The two common
%conventions are the Jackson convention which takes $\gamma=0$
%and the Jacob-Wick convention which takes $\gamma=-\phi_A$.
%
%Given the choice of the Euler angels, $\alpha$, $\beta$, and $\gamma$
%the amplitude for the decay can be written as
%\begin{eqnarray}
%A_{\lambda_A\lambda_B\lambda_C}&=&
%\sqrt{2J+1 \over 4\pi}H_{\lambda_B\lambda_C}
%D^{J*}_{\lambda_A,\lambda_B-\lambda_C}(\alpha,\beta,\gamma)\nonumber\\
%&=&\sqrt{2J+1 \over 4\pi}H_{\lambda_B\lambda_C}
%e^{i\lambda_A\alpha+i(\lambda_B-\lambda_C)\gamma}
%d^{J}_{\lambda_A,\lambda_B-\lambda_C}(\beta)\label{eq:helamp}
%\end{eqnarray}
%Using the explicit choices for the Euler angles according to the 
%conventions we obtain the amplitudes according to the two
%different conventions
%\begin{eqnarray}
%A^{\rm Jack}_{\lambda_A\lambda_B\lambda_C}&=&
%\sqrt{2J+1 \over 4\pi}H^{\rm Jack}_{\lambda_B\lambda_C}
%e^{i\lambda_A\phi_A}
%d^{J}_{\lambda_A,\lambda_B-\lambda_C}(\beta) \label{eq:helampjackson}\\
%A^{\rm JW}_{\lambda_A\lambda_B\lambda_C}&=&
%\sqrt{2J+1 \over 4\pi}H^{\rm JW}_{\lambda_B\lambda_C}
%e^{i\phi_A(\lambda_A-(\lambda_B-\lambda_C))}
%d^{J}_{\lambda_A,\lambda_B-\lambda_C}(\beta)\label{eq:helampJacobwick}
%\end{eqnarray}
%Since both of these amplitudes describe the same physics
%process it is tempting to equate the two amplitudes and 
%determine the relation between $H^{Jack}_{\lambda_B\lambda_C}$
%and $H^{JW}_{\lambda_B\lambda_C}$. But this in not correct
%since, they don't in fact describe the same system! This is
%due to different phases that has been picked up by different
%rotations in the two conventions.
%
%The Euler angles describes how the coordinate system that particle
%A is at ret in is related to coordinate systems for the 
%particles B and C. To derive the relation between the Jackson
%and Jacob-Wick conventions it is sufficient to study the
%simple case of $\theta_A=\phi_A=0$. First we consider the
%Jackson convention. For particle B the Euler angles will
%be $\alpha=0$, $\beta=0$, and $\gamma=0$. This means that
%the cordinate system in the rest frame of particle B has
%$z_B$ parallel to $z$ and that $x_B$ and $y_B$ are also
%parallel to the orixinal $x$ and $y$ axis. However, for 
%particle B the angles are $\alpha=\pi$, $\beta=\pi$, and $\gamma=0$..
%This means that the new $z$-axis, $z_C$ is in the direction oposite
%to the original $z$ axis. The new $x$-axis, $x_C$ is parallel to the 
%old and the new $y$ axis, $y_C$, is in the oposite direction
%to the old $y$-axis. This is illustrated in 
%Figure~\ref{fig:helampjackson}.
%
%\begin{figure}
%\begin{center}
%\psfig{figure=helampjackson.eps,height=1.5in}
%\caption{The coordinate systems in the Jackson convention for the
%case of $\theta_A=\phi_A=0$.
%\label{fig:helampjackson}}
%\end{center}
%\end{figure}
%
%Next consider the Jacob-Wick convention. Here the difference it that
%$\gamma$ is taken to $-\alpha$. This will not affect the coordinate system for
%particle B since $\alpha=0$. However, for particle C $\alpha=\pi$
%so here there is an additional rotation of $\pi$ around the new 
%$z$ axis. This means that the new $y$-axis $y_C$ is parallel to the
%old $y$ axis, and that $x_C$ is anti-parallel to original $x$
%axis, see Figure~\ref{fig:helampJacobwick}. 
%
%\begin{figure}
%\begin{center}
%\psfig{figure=helampjacobwick.eps,height=1.5in}
%\caption{The coordinate systems in the Jacob-Wick convention for the
%case of $\theta_A=\phi_A=0$.
%\label{fig:helampJacobwick}}
%\end{center}
%\end{figure}
%
%Since the coordinate systems for the particles in the 
%final states are not the same in the Jackson and the Jacob-Wick
%convention it is clear that the amplitudes in 
%Eq.~\ref{eq:helampjackson} and Eq.~\ref{eq:helampJacobwick}
%don't describe the same reaction. But it is straight
%forward to relate the to conventions. In fact, all we have to do
%to compare the two conventions is to rotate particle C by 
%an angle of $\pi$ around the $x_C$ axis. This rotation simply
%picks up a phase of $e^{i\pi\lambda_C}$ which allows us to relate 
%the helicity ampitudes of the Jackson and the Jacob-Wick conventions 
%according to
%\begin{equation}
%H^{\rm Jack}_{\lambda_B\lambda_C}=e^{i\pi\lambda_C}H^{\rm JW}_{\lambda_B\lambda_C}.
%\end{equation}
%This shows that there are phase differences between the two conventions.
%Note that this phase difference is easily observable in decays. E.g.
%in the decay of a scalar to two vector particles where each of the
%vectors decays to a pair of scalars this phase difference leads
%to angular distributions that has the wrong sign of the angle
%$\chi$\footnote{$\chi$ is the azimuthal angle between the decay planes 
%of the two vector mesons.}  
%if the phases are picked according to the wrong convention. 
%
%Also care has to be taken when converting from the helicty
%basis to the partial wave basis using the Jacob Wick transformation.
%If the helicity ampitudes are given in the Jackson convention
%the Jacob Wick formula needs to be modified with the
%appropriate phases. (Actually, I still have to convince myself that
%the Jacob Wick transformation is correct for the Jacob-Wick convention
%for helicity amplitudes.)
%
%Now consider the case of non zero angles $\theta_A$ and $\phi_A$ for
%the direction, $\hat p$, of particle B. With a little thought
%it is easy to convince oneself that the direction of the $y_B$ is
%along $z\times\hat p$ an consequently that the direction of 
%$y_B$ is $(z\times\hat p)\times \hat p$. Now applying this to 
%particle C which is in the direction of $-\hat p$ we find that 
%$x_C$ is parallel to $x_B$. This is just in agreement with what we
%saw in the special case studied above. 
%
%In the Jacob-Wick convention an additional rotation of $-\phi_A$
%is applied around the $z_B$ as comapred to the Jackson conventions.
%For particle C this rotation is $-\pi-\phi_A$ around the $x_C$
%axis. This gives an relative rotation of $\pi$ as compared to the
%Jackson convention, which shows that $y_B$ and $y_C$ are 
%parallel. Again this is consistent with the special case studied
%above.
%
%It is now simple to explain in words exactly what the helicity amplitude
%$A_{\lambda_A\lambda_B\lambda_C}$ means. (Note that we have suppresed
%the dependence on the Euler angles to simplify the notation.)
%Intuitively it is simply the amplitude for a state with spin
%projection $\lambda_A$ along the $z$ axis to decay to two 
%particles with helicity $\lambda_B$ and $\lambda_C$ respectively.
%As we have noticed above this leaves important phases
%unspecified. But it is easy to specify these phase by noting that
%state of the final state particle is nothing but the canonical
%state $|J_B,m_z=\lambda_B\rangle$ and $|J_C,m_z=\lambda_C\rangle$
%repectively for particle B and C. These states are defined in the 
%local coordinate system as specified by the Euler angels.
%
%\subsection{Example.}
%
%The have a concrete example to discuss, the decay 
%$B\rightarrow D^{*+}D^{*-}$ with $D^{*+}\rightarrow D^0\pi^+$ and
%$D^{*-}\rightarrow \bar D^0\pi^-$ is considered.
%
%First we discuss how the coordinate systems are set up in the
%two different conventions. This is shown in 
%figures~\ref{fig:jackson} and~\ref{fig:jacobwick}.
%In both conventions the z axis of the coordinate system that 
%describes the decay of the $D^*$'s is along the direction of
%the flight of the $D^*$. But in the Jacob-Wick convention the
%direction of the y axises are parallel while in the Jackson convention the
%direction of the x axises are parallel. 
%
%\begin{figure}
%\begin{center}
%\psfig{figure=jackson.eps,height=2.0in}
%\caption{The coordinate systems in the Jackson convention. The two
%x axises are parallel.
%\label{fig:jackson}}
%\end{center}
%\end{figure}
%
%\begin{figure}
%\begin{center}
%\psfig{figure=jacobwick.eps,height=2.0in}
%\caption{The coordinate systems in the Jacob-Wick convention. The two
%y axises are parallel.
%\label{fig:jacobwick}}
%\end{center}
%\end{figure}


\subsection{Explicit representations of SU(2)}

This section describes the explicit representations of SU(2) that
are used in the generator. As usual $\hbar=1$.

\subsubsection{$J=1/2$}

\begin{equation}
S_n={1\over 2}\gamma_0\gamma_5 n\!\!\!/.
\end{equation}

\subsubsection{$J=1$}

We will consider the representation of spin 1. 
\begin{equation}
J_x=\left[\begin{array}{rrr}
          0 & 0 & 0  \\
          0 & 0 &-i  \\
          0 & i & 0  \\
          \end{array}\right],\quad
J_y=\left[\begin{array}{rrr}
          0 & 0 & i  \\
          0 & 0 & 0  \\
          -i& 0 & 0  \\
          \end{array}\right],\quad
J_z=\left[\begin{array}{rrr}
          0 &-i & 0  \\
          i & 0 & 0  \\
          0 & 0 & 0  \\
          \end{array}\right]
\end{equation}
 
From these explicit representations it is trivial to show that they
obey the standard commutation relations, e.g.,
\begin{equation}
[J_x,J_y]=iJ_z
\end{equation}
Further we find, as expected, that
\begin{equation}
J^2=J_x^2+J_y^2+J_z^2=2I
\end{equation}
such that $J^2=j(j+1)$ which shows that this in fact is a 
representation of spin 1. The raising and lowering operators
are as usual given by

\begin{equation}
J_+=J_x+iJ_y=\left[\begin{array}{rrr}
          0 & 0 &-1  \\
          0 & 0 &-i  \\
          1 & i & 0  \\
          \end{array}\right],\quad
J_-=J_x-iJ_y\left[\begin{array}{rrr}
          0 & 0 & 1  \\
          0 & 0 &-i  \\
          -1& i & 0  \\
          \end{array}\right]
\end{equation}
It is also convenient to evaluate the expression for a finite rotation
\begin{equation}
R_z(\theta)=e^{-i\theta J_z}=\left[\begin{array}{ccc}
	\cos\theta & -\sin\theta & 0 \\
	\sin\theta &  \cos\theta & 0 \\
	 0         &  0          & 1 \\
	\end{array}\right].
\end{equation}
Of course we could have guessed the form of $R_z(\theta)$, but an
explicit evaluation of $e^{-i\theta J_z}$ is possible through
direct evaluation of the series expansion, but is more elegantly 
done using Cayley-Hamilton's theorem.\footnote{Cayley-Hamilton's 
theorem states that if $P_A(\lambda)=det(\lambda I-A)$ then $P_A(A)=0$
for any symmetric matrix $A$. This means that if $A$ has dimension
$n$ then $A^k$ can be written as a linear combination of $A^i$ for
$i<n$. In particular it can be shown that $f(A)=q(A)$ where $q$
is a polynomial of degree $<n$. The polynomial $q(A)$ is defined
by  
\begin{equation}
{d^jf\over dz^j}(\lambda_k)={d^jq\over dz^j}(\lambda_k)
\end{equation}
where $\lambda_k$ are the eigen values of $A$, with multiplicity
$n_k$ and $j=0..n_k-1$.
}

The states
\begin{equation}
\epsilon_+={1\over\sqrt{2}}\left[\begin{array}{r}
          -1  \\
          -i  \\
          0  \\
          \end{array}\right],\quad
\epsilon_0=\left[\begin{array}{r}
          0  \\
          0  \\
          1  \\
          \end{array}\right],\quad
\epsilon_-={1\over\sqrt{2}}\left[\begin{array}{r}
          1  \\
          -i  \\
          0  \\
          \end{array}\right]
\label{eq:spinonestates}
\end{equation}

are easily seen to satisfy 
\begin{equation}
J_z\epsilon_{\lambda}=\lambda\epsilon_{\lambda}
\end{equation}
and therefore form a helicity eigenstate basis for a particle
with $J=1$ moving in the direction of the positive $z$-axis.

\subsection{Projections of helicity amplitudes}

This section shows how the helicity amplitudes are related to the
invariant amplitudes. Consider the decay of a scalar to two
vector particles. Let the final states of the two vector particles
be denoted by $\epsilon^1_{s_1}$ and $\epsilon^2_{s_2}$ and $\vec p$
be the relative momentum of the two particles in the parents
rest frame. The amplitude for this process is then written as
\begin{equation}
A_{s_1s_2}=\epsilon^{1*}_{is_1}\epsilon^{2*}_{js_2}M_{ij}(\vec p)
\end{equation}
where $M_{ij}(p)$ is a rank 2 tensor as a function of the available
momenta in the process, i.e., a function of $\vec p$. The most
general form of M is given by
\begin{equation}
M_{ij}=a\delta_{ij}+b\epsilon_{ijk}p_k+cp_ip_j.
\end{equation}
The coefficients $a$, $b$, and $c$ are the invariant amplitudes, we 
wand to relate them to helicity amplitudes. First we note that there
are three invariant amplitudes, this is the same as the number of 
helicity amplitudes. 

From Eq.~\ref{eq:helicityformula} the amplitude is given by
\begin{equation}
A_{\lambda_1\lambda_2}=\sqrt{2J+1\over 4\pi}H_{\lambda_1\lambda_2}
D^{*J}_{M\lambda_1-\lambda_2}(\phi,\theta,-\phi).
\end{equation}

When relating the helicity amplitudes to the invariant amplitudes
it is sufficient to look at one kinematic configuration, we chose
the simplest possible in which $\theta=\phi=0$. From the
definition of the $D$ function it is obvious that 
$D^{*J}_{M\lambda_1-\lambda_2}(0,0,0)=\delta_{M,\lambda_1-\lambda_2}$.
This gives
\begin{equation}
H_{\lambda_1\lambda_2}=\sqrt{4\pi\over 2J+1}A_{\lambda_1\lambda_2}.
\end{equation}
This allows us to simply evaluate the helicity amplitudes from the
invariant amplitudes if we chose the states, $\epsilon^1$ and $\epsilon^2$,
to correspond to the states in the helicity formalism. 
For $\epsilon^1$ we take the states given by~\ref{eq:spinonestates}
\begin{equation}
\epsilon^1_+={1\over\sqrt{2}}\left[\begin{array}{r}
          -1  \\
          -i  \\
          0  \\
          \end{array}\right],\quad
\epsilon^1_0=\left[\begin{array}{r}
          0  \\
          0  \\
          1  \\
          \end{array}\right],\quad
\epsilon^1_-={1\over\sqrt{2}}\left[\begin{array}{r}
          1  \\
          -i  \\
          0  \\
          \end{array}\right].
\end{equation}
The states for the second particle have to satisfy 
$\chi_{0\lambda}=\Psi_{0-\lambda}$ which gives
\begin{equation}
\epsilon^2_+={1\over\sqrt{2}}\left[\begin{array}{r}
          1  \\
          -i  \\
          0  \\
          \end{array}\right],\quad
\epsilon^2_0=\left[\begin{array}{r}
          0  \\
          0  \\
          1  \\
          \end{array}\right],\quad
\epsilon^2_-={1\over\sqrt{2}}\left[\begin{array}{r}
          -1  \\
          -i  \\
          0  \\
          \end{array}\right].
\end{equation}
Note that this is consistent with Eq.~\ref{eq:newdef}.
Now it is straight forward to evaluate the helicity amplitudes
\begin{eqnarray}
 H_{++} & = & \sqrt{4\pi\over 3}(-a-ib), \\
 H_{00} & = & \sqrt{4\pi\over 3}(a+c),\\
 H_{--} & = & \sqrt{4\pi\over 3}(-a+ib).
\end{eqnarray}
These equations are easily inverted to give
\begin{eqnarray}
 a & = & -{1\over2}\sqrt{3\over 4\pi}(H_{++}+H_{--}), \\
 b & = & \phantom{-}{i\over2}\sqrt{3\over 4\pi}(H_{++}-H_{--}), \\
 c & = & \phantom{-{1\over2}}\sqrt{3\over 4\pi}(H_{00}+{1\over 2}(H_{++}+H_{--})), \\
\end{eqnarray}


\subsection{Jacob-Wick transformation}
\index{Jacob-Wick transformation}
\index{partial wave amplitudes}
\label{sect:jacobwick}

This section derives the relation between partial wave amplitudes 
and helicity amplitudes. 

The amplitude for a decay, given the partial wave amplitudes $M_{Ls}$,
is given by
$$A(\Omega,m1,m2,m3)=\sum_{L,s}C_{s_2s_3}(s\ m_s;m_2\ m_3)
                               C_{Ls}(s_1\ m_1;m_L\ m_s)Y^{m_L}_L(\Omega)
                               M_{Ls}.$$


\subsection{{\tt HELAMP} and {\tt PARTWAVE} model implementations}
\index{HELAMP}
\index{PARTWAVE}

The two models {\tt HELAMP} and {\tt PARTWAVE} implements generic
two body decays where the amplitudes are specified by the
helicity and partial wave amplitudes respectively. The partial
wave model uses the translation to helicity amplitudes as given in
Sect.~\ref{sect:jacobwick}. Henceforth, this section will focus
in the description of the evalution of helicity amplitudes.

 EvtGen uses a set of states, that are not the same as the helicity
states. In this section it is important to distinguish between these two
sets of states. The basis states used by EvtGen will be called
$|n\rangle$ and the helicity states will be denoted $|\lambda, s\rangle$,
or simply $|\lambda\rangle>$. 

 We consider the decay of the form $A\rightarrow B C$ where the state
of the initial particle is given by $|n_A\rangle$ and the two particles
in the final state are labeled by $|n_B\rangle$ and $|n_C\rangle$. 
For convenience we label the final state $|n_B, n_C\rangle=|n_B\rangle
\otimes |n_C\rangle$. Given this notation we write the amplitude that
we need to implement as 
$$A^{B\rightarrow BC}_{n_A,n_B,n_C}=\langle n_B, n_C|H|n_A\rangle.$$
Using the completeness of the basis states we write
\begin{eqnarray}
A^{B\rightarrow BC}_{n_A,n_B,n_C}&=&\sum_{\lambda_A,\lambda_B,\lambda_C}\langle n_B, n_C|\lambda_B\rangle\langle\lambda_B|\lambda_C\rangle\langle\lambda_C|H|\lambda_A\rangle\langle\lambda_A|n_A\rangle\\
                                 &=&\sum_{\lambda_A,\lambda_B,\lambda_C}\langle n_B|\lambda_B\rangle \langle n_C|\lambda_C\rangle\langle\lambda_B|\langle\lambda_C|H|\lambda_A\rangle\langle\lambda_A|n_A\rangle\\
                                 &=&\sum_{\lambda_A,\lambda_B,\lambda_C}R^{*}_{n_B\lambda_B} R^{*}_{n_C\lambda_C}\langle\lambda_B|\langle\lambda_C|H|\lambda_A\rangle R_{\lambda_A n_A}
\end{eqnarray}
Where 
\begin{eqnarray}
R_{n_A\lambda_A} & = & \langle \lambda_A | n_A \rangle \\ 
R_{n_B\lambda_B} & = & \langle \lambda_B | n_B \rangle \\
R_{n_C\lambda_C} & = & \langle \lambda_C | n_C \rangle
\end{eqnarray}






