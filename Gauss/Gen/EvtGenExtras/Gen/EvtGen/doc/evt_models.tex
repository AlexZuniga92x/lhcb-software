\section{Decay models}
\label{sect:models}
\index{models}

This section lists the different decay models that has been 
implemented in EvtGen. It is strongly urged that all decays that
are implemented are added to this list with, at least, a minimal 
description of what they are doing. 

The models are organized in alphabetical order here. Each model
is briefly described with respect to what decays it can handle and
what the arguments mean and one or more examples are given. For
further examples of use please see the decay table, {\tt DECAY.DEC}

\Model{BHADRONIC}

\label{bhadronic}

\Auth{Ryd}

\Usage{P1 P2 ... PN}{JH JW;}

\Expl
This is an experimental model for hadronic $B$ decays. Until
further developed this is not recommended to be used. For
questions ask Anders Ryd.

\Model{BTO3PI\_CP}

\label{bto3picp}

\Auth{Le Diberder, Versille}

\Usage{P1 P2 P3}{dm alpha;}

\Expl
This model is for neutral B decays in $\pi^-\pi^+\pi^0$. 
Several resonances are taken into account: the $\rho(770)$, the $\rho(1450)$, and the $\rho(1700)$, the generator therefore implements the interferences between all different final states (e.g. $B^0 \rightarrow \rho^+(770) \pi^- \rightarrow \pi^+\pi^0\pi^-$ and $B^0 \rightarrow \rho^-(770) \pi^+ \rightarrow \pi^-\pi^0\pi^+$). Several Breit-Wigners descriptions are available. By default, the generator is initialized with relativistic Breit-Wigners according to the Kuhn-Santamaria model where the parameters have been fitted by Aleph with $e^+e^-$ and $\tau^+\tau^-$ data~\cite{AlephRho}.
It uses a pole-compensation method to generate the events efficiently by taking into account the poles due to the Breit-Wigners of the $\rho$'s \cite{3piNote}. 

The generator returns the amplitudes for $B^0 \rightarrow 3\pi$, and $\overline{B^0} \rightarrow 3\pi$ for the kinematics of the generated final state. It makes use of values of Tree and Penguins amplitudes and phases which have
been computed by the LPTHE using the factorization approximation and the Orsay
quark model, on the basis of the Isospin relation (no ElectroWeak Penguins
are included here, and the strong phases are set to zero).The $\rho^0\pi^0$ gets a very low branching ratio due to color-suppression \cite{3piBook}.

\Example
\noindent The example shows how to generate $B^0 \rightarrow \pi^-\pi^+\pi^0$.
\begin{verbatim}
Decay B0
1.000 pi- pi+ pi0   BTO3PI_CP   dm alpha;
Enddecay
\end{verbatim}

\Notes
This routine makes use of a fortran routine to perform the 
actual calculation of the amplitude.


\Model{CB3PI-MPP}

\label{cbto3pimpp}

\Auth{Le Diberder, Versill\'e}

\Usage{P1 P2 P3}{dm alpha;}

\Expl
This model is for charged B decays in $\pi^{\pm}\pi^+\pi^-$.
It is built on the same basis than the BTO3PI\_CP model, making also use of interferences between the three $\rho$ bands: $\rho(770)$, the $\rho(1450)$, and the $\rho(1700)$. The amplitudes are computed by the LPTHE.

\Example

\noindent The example shows how to generate $B^+ \rightarrow \pi^+\pi^+\pi^-$.
\begin{verbatim}
Decay B+
1.000 pi+ pi+ pi-   CB3PI-MPP   dm alpha;
Enddecay
\end{verbatim}

\Notes
This routine makes use of a fortran routine to perform the 
actual calculation of the amplitude.


\Model{CB3PI-P00}

\label{cbto3pip00}

\Auth{Le Diberder, Versill\'e}

\Usage{P1 P2 P3}{dm alpha;}

\Expl
This model is for charged B decays in $\pi^{\pm}\pi^0\pi^0$.
It is built on the same basis than the BTO3PI\_CP model, making also use of interferences between the three $\rho$ bands: $\rho(770)$, the $\rho(1450)$, and the $\rho(1700)$. The amplitudes are computed by the LPTHE.

\Example

\noindent The example shows how to generate $B^+ \rightarrow \pi^+\pi^0\pi^0$.
\begin{verbatim}
Decay B+
1.000 pi+ pi0 pi0   CB3PI-P00   dm alpha;
Enddecay
\end{verbatim}

\Notes
This routine makes use of a fortran routine to perform the 
actual calculation of the amplitude.


\Model{BTOKPIPI\_CP}
\label{btoKpipicp}

\Auth{ Le Diberder, Versill\'e}

\Usage{P1 P2 P3}{dm alpha;}

\Expl
This model is for neutral B decays in $K \pi\pi$ (note that the 
$B^0$ decays in $K^+ \pi^- \pi^0$ and the $\overline{B^0}$ in
$K^- \pi^+ \pi^0$). It generates interferences between different resonances:
\begin{itemize}
\item{} $  B^0 \rightarrow K^{*+} \pi^- $, 
                       with $K^{*+} \rightarrow K^+ \pi^0$,
\item{} $  B^0 \rightarrow {K^{*0}} \pi^0 $, 
                       with ${K^{*0}} \rightarrow K^+ \pi^-$,
\item{} $  B^0 \rightarrow K^- \rho^{+} $, 
                       with $\rho^{+} \rightarrow \pi^+ \pi^0$
\end{itemize}
It also provides the amplitudes for the CP-conjugate channels
$\overline{B^0} \rightarrow K^- \pi^+ \pi^0$.
The Tree and Penguins amplitudes are computed by the LPTHE.

\Example
\noindent The example shows how to generate $B^0 \rightarrow K^+\pi^-\pi^0$.
\begin{verbatim}
Decay B0
1.000 K+ pi- pi0    BTOKPIPI_CP dm alpha;
Enddecay
\end{verbatim}

\Notes
This routine makes use of a fortran routine to perform the 
actual calculation of the amplitudes.

\Model{BTO4PI\_CP}

\label{bto4picp}

\Auth{Ryd}

\Usage{P1 P2 P3 P4}{dm alpha +8 amplitudes; }

\Expl
This model is for $B\rightarrow \pi^+\pi^-\pi^+\pi^-$. It implements the
time dependence of the decay correctly depending on where in the
dalitz plot you are. The amplitudes that needs to be specified are
$B\rightarrow a_1^+ \pi^-$, $\bar B\rightarrow a_1^+ \pi^-$,
$B\rightarrow a_2^+ \pi^-$, $\bar B\rightarrow a_2^+ \pi^-$,
$B\rightarrow a_1^- \pi^+$, $\bar B\rightarrow a_1^- \pi^+$,
$B\rightarrow a_2^- \pi^+$, and $\bar B\rightarrow a_2^- \pi^+$.

\Example
The example shows how to generate $B^0 \rightarrow \pi^+\pi^-\pi^+\pi^-$.
\begin{verbatim}
Decay B0
1.000 pi+ pi- pi+ pi-   BTP4PI_CP   dm alpha 1.0 0.0 1.0 0.0
                                               1.0 0.0 1.0 0.0
                                               1.0 0.0 1.0 0.0
                                               1.0 0.0 1.0 0.0;
Enddecay
\end{verbatim}

\Notes
This routine is still developing.


\Model{BTO2PI\_CP\_ISO}

\label{bto2picpiso}

\Auth{NK}

\Usage{P1 P2}{beta dm $|A_{2}|$ $\varphi_{A_{2}}$ $|\overline{A}_{2}|$ $\varphi_{\overline{A}_{2}}$ $|A_{0}|$ $\varphi_{A_{0}}$ $|\overline{A}_{0}|$ $\varphi_{\overline{A}_{0}}$;}

\Expl
This model approaches the three $B \rightarrow \pi \pi$ modes from the point of view of isospin analysis.  It is applicable to both the two $B^{0}$ ($\overline{B}^{0}$) modes, in which case it takes into account mixing, and to the $B^{+}$ ($B^{-}$) mode, as all three modes should indeed be treated together in this approach.  Following the conventions of Lipkin, Nir, Quinn, and Snyder (Phys. Rev. D{\bf 44}, 1454 (1991)), the various decay amplitudes can be written as follows:
\begin{equation}
A(B^{+} \rightarrow \pi^{+} \pi^{0}) \equiv A^{+0} = 3\,A_{2} 
\end{equation}  

\begin{equation}
A(B^{0} \rightarrow \pi^{+} \pi^{-}) \equiv \sqrt{\frac{1}{2}} \,A^{+-} = A_{2}\,- \,A_{0}
\end{equation} 

\begin{equation}
A(B^{0} \rightarrow \pi^{0} \pi^{0}) \equiv A^{00} = 2\,A_{2}\, +\,A_{0},  
\end{equation}

where $A_{2}$ is the amplitude for $I_{f}$ = 2 states (tree only), and $A_{0}$, for $I_{f}$ = 0 states (where both tree and penguin contribute).  

The model's parameters are:
\begin{itemize}
\item
beta = corresponding CKM angle
\item
dm = $B^{0} \overline{B}^{0}$ mass difference ($\approx 0.5 \times 10^{12} s^{-1}$).  
\item
$|A_{2}|$, $\varphi_{A_{2}}$  = magnitude and phase of the corresponding amplitude
\item
$|\overline{A}_{2}|$, $\varphi_{\overline{A}_{2}}$ = magnitude and phase of the amplitude for the CP-conjugate process
\item
$|A_{0}|$, $\varphi_{A_{0}}$  = magnitude and phase of the corresponding amplitude
\item
$|\overline{A}_{0}|$, $\varphi_{\overline{A}_{0}}$ = magnitude and phase of the amplitude for the CP-conjugate process
\end{itemize}

\Example

\begin{verbatim}
Decay B0
1.000 pi+ pi-  BT02PI_CP_ISO  beta dm 1.0 gamma 1.0 -gamma
                                      1.0 gamma 1.0 -gamma;
Enddecay
\end{verbatim}

\Notes
Precise numerical estimates for the amplitudes are not available at the moment.



\Model{BTOKPI\_CP\_ISO}

\label{btokpicpiso}

\Auth{NK}

\Usage{PI K}{beta dm $|U|$ $\varphi_{U}$ $|\overline{U}|$ $\varphi_{\overline{U}}$ $|V|$ $\varphi_{V}$ $|\overline{V}|$ $\varphi_{\overline{V}}$ $|W|$ $\varphi_{W}$ $|\overline{W}|$ $\varphi_{\overline{W}}$;}

\Expl
This model considers the four $B \rightarrow \pi K$ modes from the point of view of isospin analysis.  It is applicable to both the two $B^{0}$ ($\overline{B}^{0}$) modes and to the two $B^{+}$ ($B^{-}$) modes, as all four modes should indeed be treated together in this approach.  Following the conventions of Lipkin, Nir, Quinn, and Snyder (Phys. Rev. D{\bf 44}, 1454 (1991)), the various decay amplitudes can be written as follows:
\begin{equation}
A(B^{+} \rightarrow \pi^{0} K^{+}) \equiv A^{0+} = U\,-\,W 
\end{equation}  

\begin{equation}
A(B^{+} \rightarrow \pi^{+} K^{0}) \equiv \sqrt{\frac{1}{2}} \,A^{+0} = V\,+\,W 
\end{equation} 

\begin{equation}
A(B^{0} \rightarrow \pi^{-} K^{+}) \equiv \sqrt{\frac{1}{2}} \,A^{-+} = V\,-\,W 
\end{equation}

\begin{equation}
A(B^{0} \rightarrow \pi^{0} K^{0}) \equiv A^{00} = U\,+\,W,  
\end{equation}

where $W$, $U$, and $V$ are linear combinations of the three independent amplitudes $A_{I_{t},I_{f}}$ for various transition ($I_{t}$) and final ($I_{f}$) isospins (please see the reference for more details).  Note that both $U$ and $V$ are tree-only amplitudes, whereas $W$ includes both tree and penguin contributions.

The model's parameters are:
\begin{itemize}
\item
beta = corresponding CKM angle
\item
dm = $B^{0} \overline{B}^{0}$ mass difference ($\approx 0.5 \times 10^{12} s^{-1}$).  
\item
$|U|$, $\varphi_{U}$  = magnitude and phase of the corresponding amplitude
\item
$|\overline{U}|$, $\varphi_{\overline{U}}$ = magnitude and phase of the amplitude for the CP-conjugate process
\item
$|V|$, $\varphi_{V}$  = magnitude and phase of the corresponding amplitude
\item
$|\overline{V}|$, $\varphi_{\overline{V}}$ = magnitude and phase of the amplitude for the CP-conjugate process
\item
$|W|$, $\varphi_{W}$  = magnitude and phase of the corresponding amplitude
\item
$|\overline{W}|$, $\varphi_{\overline{W}}$ = magnitude and phase of the amplitude for the CP-conjugate process
\end{itemize}

\Example

\begin{verbatim}
Decay B0
1.000  K+ pi-  BTOKPI_CP_ISO   beta dm  1.0 gamma 1.0 -gamma 
                                        1.0 gamma 1.0 -gamma 
                                        1.0 gamma 1.0 -gamma; 
Enddecay
\end{verbatim}

\Notes
Precise numerical estimates for the amplitudes are not available at the moment.


\Model{BTOXSGAMMA}

\label{btoxsgamma}

\Auth{ Francesca Di Lodovico, Jane Tinslay, Mark Ian Williams}

\Usage{P1 P2}{model}
or\\
\Usage{P1 P2}{ model F $m_B$ $m_b$ $\mu$ $\lambda_1$ $\delta$ z (number of intervals to\\
compute $\alpha_s$) (number of intervals tocompute the hadronic mass)}

\Expl
This model is for two-body non-resonant $B \rightarrow X_{s} \gamma$ decays 
where strange hadrons, $X_{s}$, are generated with a linewidth given by 
the mass spectrum predicted by either Ali and Greub~\cite{AliGreub} or
Kagan and Neubert~\cite{KaganNeubert} model, according to the first parameter in the datacards given after the 
model is chosen. In case the Ali and Greub
model is chosen, a parameterisation of the mass spectrum predicted for given inputs is used.
The input parameters where based on PDG 2000 values plus a 
b quark Fermi momentum of 265 MeV for a spectator quark mass of 150 MeV, 
which was taken from CLEO fits of the semileptonic B momentum spectrum.
In case the Kagan and Neubert model is used, the input parameters can be given as an input 
in the datacards. The are: F = Fermi momentum model (1 = exponential shape function, 
2 = gaussian shape function, 3 = roman shape function), $m_B$, $m_b$, $\mu$, $\lambda_1$, 
$\delta$, z, number of intervals to compute $\alpha_s$, number of intervals to
compute the hadronic mass. Moreover, as a possible option, no input parameters can be given after the
Kagan and Neubert model is chosen, and in this case default input parameters are chosen
( F = 1, $m_B$ = 5.27885 GeV/c$^2$, $m_b$ =  4.80 GeV/c$^2$, $\mu$ = 4.80 GeV/c$^2$, $\lambda_1$ = 0.3, 
$\delta$ = 0.9, z = 0.084, number of intervals to compute $\alpha_s$ = 100, number of intervals to
compute the hadronic mass = 80). A a cut--off on the hadronic mass at 1.1 GeV/c$^2$ is applied in this case 
according to~\cite{KaganNeubert}.

The maximum mass value for all $X_{s}$ is 4.5 GeV/c$^2$ and the minimum mass 
value is at the K$\pi$ threshold for $X_{su}$ and $X_{sd}$, and at the KK
threshold for $X_{ss}$. JETSET is required to decay the resulting $X_{s}$ 
into hadrons via phase-space production from the available quarks. The decay
of $X_{s}$ needs to be switched on in the decay file using JETSET switches.

\Example
\noindent The example shows how to generate $B^{0} \rightarrow X_{sd} \gamma$ for the Ali and Greub Model.
\begin{verbatim}
#   Xsd meson  (sbar-d system, introduced for b->s gamma decays)

# Set Xsd so it can decay:
JetSetPar MDCY(455,1)=1
# Set decay table entry pt for Xsd: 
JetSetPar MDCY(455,2)=1154
# Number of decay channels for Xsd:                
JetSetPar MDCY(455,3)=1
# Switch on Xsd decay
JetSetPar MDME(1154,1)=1
# Phase space decays into hadrons from available quarks
JetSetPar MDME(1154,2)=11
# Xsd decays into two quarks a d and an anti-s
JetSetPar KFDP(1154,1)=-3
JetSetPar KFDP(1154,2)=1

Decay B0
1.0000 Xsd   gamma   BTOXSGAMMA 1;
Enddecay
\end{verbatim}

\Notes
P1 should always be reserved for the $X_{s}$ particle and P2 should
always be a gamma. Also, this model requires Jst74 V00-00-11 or higher 
to work. 

\Model{D\_DALITZ;}

\label{dplusdalitz}

\Auth{Kuznetsova}

\Usage{D1 D2 D3}{}

\Expl
The Dalitz amplitude for three-body $K \pi \pi$ D decays; namely, for decays 
\begin{itemize}
\item
$D^+\rightarrow K^- \pi^+ \pi^+$ or $D^-\rightarrow K^+ \pi^- \pi^-$,\\
with the resonances (for the $D^{+}$ mode) $\overline{K}^{\,\ast}(892)^{0}\pi^{+}$, $\overline{K}^{\,\ast}(1430)^{0}\pi^{+}$, and $\overline{K}^{\,\ast}(1680)^{0}\pi^{+}$, using data from the E691 Fermilab experiment ~\cite{Anjos93}.
\item
$D^+\rightarrow \overline{K}^{0} \pi^+ \pi^0$ or $D^-\rightarrow K^0 \pi^- \pi^0$,\\
with the resonances (for the $D^{+}$ mode) $\overline{K}^{\,\ast}(892)^{0}\pi^{+}$, and $\overline{K}^{0}\rho^{+}$, using data from MARK III ~\cite{Adler87}. 
\item
$D^0\rightarrow \overline{K}^0\, \pi^+ \pi^-$ or $\overline{D}^0\rightarrow K^0\, \pi^- \pi^+$,\\
with the resonances (for the $D^{0}$ mode) $K^{\,\ast}(892)^{-} \pi{+}$ and $\overline{K}^{0} \rho(770)^{0}$, using data from ~\cite{Anjos93}. 
\item
$D^0\rightarrow K^{-} \pi^+ \pi^0$ or $\overline{D}^0\rightarrow K^+ \pi^- \pi^0$,\\
with the resonances (for the $D^{0}$ mode) $\overline{K}^{\,\ast}(892)^{0}\pi^{0}$, $K^{\,\ast}(892)^{-}\pi^{+}$, and $K^{-}\rho(770)^{+}$, using data from ~\cite{Anjos93}. 
\end{itemize}

Be aware that the $D^+\rightarrow \overline{K}^{0} \pi^+ \pi^0$ and $D^-\rightarrow K^0 \pi^- \pi^0$ modes currently use the results from Mark III ~\cite{Adler87}, which are based on rather limited statistics.    
\Example
To generate the decay $D^+\rightarrow K^- \pi^+ \pi^+$ the 
following entry in the decay table should be used
\begin{verbatim}
Decay D+
1.000 K- pi+ pi+      D_DALITZ;
Enddecay
\end{verbatim}

\Notes
The order in which the particles are listed is very important: the kaon should 
always be first, and for the modes with the neutral pion the $\pi^{0}$ should always be last.



\Model{GOITY\_ROBERTS}

\label{goityroberts}

\Auth{Alain,Ryd}

\Usage{M1 M2 L N}{;}

\Expl
Model for the non-resonant $D^{(*)}\pi\ell\nu$ decays of
$B$ mesons. The daughters are in the order: $D$-meson,
pion, lepton and last the neutrino.

\Example
\begin{verbatim}
Decay B0
1.000 D0B  pi- e+ nu_e                  GOITY_ROBERTS;
Enddecay
\end{verbatim}

\Notes
This is not exactly what was published by Goity and Roberts~\cite{Goity95a}, 
partly
due to errors in the paper and because the $D^*$ had to be 
removed from the $D\pi$ non-resonant.



\Model{HELAMP}

\label{helamp}

\Auth{Ryd}

\Usage{M D1 D2}{Amplitudes;}

\Expl
This model allows simulation of any two body decay by specifying the 
helicity amplitudes for the final state particles. The helicity amplitudes
are complex numbers specified as pairs of magnitude and phase. The amplitudes
are ordered, starting by the highest allowed helicity for the first particle.
For a fixed helicity of the first particle the amplitudes are then specified
starting with the highest allowed helicity of the second particle. This means
that the helicities $H_{\lambda_1\lambda_2}$ are ordered first according the
the value of $\lambda_1$ and then the value of $\lambda_2$.

\Example
Decay of $B^0->D^*\rho$, 
\begin{verbatim}
Decay B+
1.000   anti-D*0  rho+           HELAMP 0.228 0.95 0.283 1.13 0.932 0;
Enddecay
\end{verbatim}

\Notes
The amplitudes are taken from ICHEP 98-852. This model has been tested on
many special cases, but further testing is needed.


\Model{HQET}

\label{hqet}

\Auth{Lange}

\Usage{M L N}{RHO2 R1 R2;}

\Expl
Model for the $D^{*}\ell\nu$ decay of
$B$ mesons according to a HQET inspired parameterization. 
The daughters are in the order: $D^*$,
lepton and last the neutrino. Since only the three
form factors that contributes in the zero lepton mass
limit are included the model is not accurate for $\tau$'s.
The arguments, {\tt RHO2}, {\tt R1}, and {\tt R2} are the 
form factor slope, $\rho^2_{A_1}$ and the form factor
ratios $R_1$ and $R_2$ respectively. These are defined,
and measured, in~\cite{Dubo96}

\Example
Decay of $B^0\rightarrow D^*\ell\nu$ using HQET model
\begin{verbatim}
Decay B0
1.000 D*-  e+ nu_e              HQET3S1 0.92 1.18 0.72;
Enddecay
\end{verbatim}

\Notes
The values in the example above comes from the measurement 
in~\cite{Dubo96} by the CLEO collaboration.


\Model{HQET2}

\label{hqet2}

\Auth{Ishikawa}

\Usage{M L N}{RHO2 R1 R2;}

\Expl
Model for the $D^{*}\ell\nu$ decay of $B$ mesons according to 
the dispersive relation \cite{Cap98}. 
The daughters are in the order: $D^*$,
lepton and last the neutrino. 
The arguments, {\tt RHO2}, {\tt R1}, and {\tt R2} are the 
form factor slope, $\rho^2_{A_1}$ and the form factor
ratios $R_1$ and $R_2$ respectively. 

\Example
Decay of $B^0\rightarrow D^*\ell\nu$ using HQET model
\begin{verbatim}
Decay B0
1.000 D*-  e+ nu_e              HQET2 1.35 1.3 0.8;
Enddecay
\end{verbatim}

\Notes
The values in the example above comes from the measurement 
in~\cite{Abe02} by the Belle collaboration.



\Model{ISGW}

\label{isgw}

\Auth{Lange, Ryd}

\Usage{D1 D2 D3}{;}

\Expl
This is a model for semileptonic decays of $B$, and $D$ mesons 
according to the ISGW model~\cite{Isgur89a}. The first daughter
is the meson produced in the semileptonic decay. The second and third
argument is the lepton and the neutrino respectively.
See Section~\ref{semileptonic} for more details about semileptonic
decays.

\Example
The example shows how to generate $\bar B^0\rightarrow D^{*+}e\nu$
\begin{verbatim}
Decay anti-B0
1.000 D*+ e- anti-nu_e     ISGW;
Enddecay
\end{verbatim}

\Notes
This model does not include the $A_3$ form factor that is 
needed for non-zero mass leptons, i.e., tau's. If tau's
are generated the $A_3$ form factor will be zero.




\Model{ISGW2}

\label{isgw2}

\Auth{Lange, Ryd}

\Usage{D1 D2 D3}{;}

\Expl
This is a model for semileptonic decays of $B$, $D$, and $D_s$ mesons 
according to the ISGW2 model~\cite{Scora95}. The first daughter
is the meson produced in the semileptonic decay. The second and third
argument is the lepton and the neutrino respectively.
See Section~\ref{semileptonic} for more details about semileptonic
decays.

\Example
The example shows how to generate $\bar B^0\rightarrow D^{*+}e\nu$
\begin{verbatim}
Decay anti-B0
1.000 D*+ e- anti-nu_e     ISGW2;
Enddecay
\end{verbatim}

\Notes
This model has been fairly well tested for $B$ decays, most form
factors and distributions have been compared to the original 
code that we obtained from D. Scora.


\Model{JETSET}

\label{jetset}

\Auth{Ryd, Waldi}

\Usage{D1 D2 DN}{MODE;}

\Expl
A particle who's decay is not implanted in EvtGen
can be decayed by calling JetSet using this model as an interface.
The decays that uses the {\tt JETSET} model are converted
into the JetSet decay table format and read in by JetSet.
However, if JetSet produces a final state which
is explicitly listed as another decay of the parent it
is rejected. E.g. consider this example:
\begin{verbatim}
Decay J/psi
0.0602  e+  e-                            VLL;
0.0602  mu+ mu-                           VLL;
.
.
.
0.8430  rndmflav  anti-rndmflav           JETSET         12;
Enddecay
\end{verbatim}
In this example if JetSet decays the $J/\Psi$ to $e^+e^-$
or $\mu^+\mu^-$ the decay is rejected and regenerated.
For more details about the EvtGen-Jetset interface see
Appendix~\ref{sect:jetsetinterface}.

\Notes
As discussed in Appendix~\ref{sect:jetsetinterface} the {\tt JETSET}
model can not be used to decay a particle that is an alias. This is 
bacause JetSet does not allow for more than one decay table per
particle.



\Model{JSCONT}

\Auth{Ryd, Kim}

\Usage{}{Flavor;}

\Expl
This decay model is for generation of continuum events at the 
$\Upsilon(4S)$. It uses JetSet to fragment quark strings. The
flavor of the primary string is given as the argument to the
model and 1 means a $d\bar d$, 2 is $u\bar u$, 3 is $s\bar s$ and
4 is $c\bar c$. If the flavor is 0 a mixture of the quarks will
be generated in the appropriate amounts. 
The first particle that is created is the {\tt vpho} which can be
decayed using this decay model.
The primary jets are created according to a $1+\cos^2\theta$
distribution, where $\theta$ is the angle of the primary jet
with respect to the beam line, or more precisely the $z$-axis.

\Example
\begin{verbatim}
Decay vpho
1.000          JSCONT 1;
Enddecay
\end{verbatim}




\Model{KLL3P}

\label{KLL3P}

\Auth{Rotondo}

\Usage{K L1 L2}{;}

\Expl
Implementation for the process $B\rightarrow K l^+ l^-$, as the previous model
the form-factors are calculated in the framework of three-point QCD sum-rules ~\cite{Colangelo96}.
 
\Example
The example shows how to generate $B^+\rightarrow K^{+} e^+ e^-$
\begin{verbatim}
Decay B+
1.000 K+ e+ e- KLL3P;
Enddecay
\end{verbatim}

\Notes
Only Short Distance interaction are considered.







%\Model{KS}
%
%\label{ks}
%
%\Auth{Lange, Ryd}
%
%\Usage{D1 D2 D3}{;}
%
%\Expl
%This is a model for semileptonic decays of $B$, and $D$ mesons 
%according to the KS model~\cite{Korner88}. The first daughter
%is the meson produced in the semileptonic decay. The second and third
%argument is the lepton and the neutrino respectively.
%See Section~\ref{semileptonic} for more details about semileptonic
%decays.
%
%\Example
%The example shows how to generate $\bar B^0\rightarrow D^{*+}e\nu$
%\begin{verbatim}
%Decay anti-B0
%1.000 D*+ e- anti-nu_e     KS;
%Enddecay
%\end{verbatim}
%
%\Notes
%This model does not include the $A_3$ form factor that is 
%needed for non-zero mass leptons, i.e., tau's. If tau's
%are generated the $A_3$ form factor will be zero.
%
%
\Model{KSLLLCQCD}

\label{KSLLLCQCD}

\Auth{Rotondo}

\Usage{K* L1 L2}{;}

\Expl
Implementation of the process $B\rightarrow K^* l^+ l^-$.
In this model the hadronic part of the matrix element is 
calculated in the framework of the light-cone QCD
sum rules ~\cite{Aliev97}.
 
\Example
The example shows how to generate $B^0\rightarrow K^{*0} \tau^+ \tau^-$.
\begin{verbatim}
Decay B0
1.000 K*0 tau+ tau- KSLLLCQCD;
Enddecay
\end{verbatim}

\Notes
In the Aliev's paper ~\cite{Aliev97} are taken in account some
effects out the Standard-Model, this implementation recover only
the $SM$ part of the interaction. \\
Only Short Distance contribution are considered.\\
Warning: a cut-off on the $q^2$ of the lepton pair are introduced. 
The $q^2$ is required to be greater than $0.08 GeV^2$, 
to avoid a large spend of time for the generation of the 
right configuration. This approximation is 
not useful for $B\rightarrow K^* e^+ e^-$.   




\Model{KSLL3PQCD}

\label{KSLL3PQCD}

\Auth{Rotondo}

\Usage{K* L1 L2}{;}

\Expl
Implementation for the process $B\rightarrow K^* l^+ l^-$ in which the hadronic 
part of the matrix element is calculated in the framework of the three-point QCD 
sum rules ~\cite{Colangelo96}.
 
\Example
The example shows how to generate $B^0\rightarrow K^{*0} \mu^+ \mu^-$.
\begin{verbatim}
Decay B0
1.000 K*0 mu+ mu- KSLL3PQCD;
Enddecay
\end{verbatim}

\Notes
Only Short Distance interaction are considered.\\
Warning: as the previous model.


\Model{LNUGAMMA}

\label{lnugamma}

\Auth{edward}

\Usage{L NU GAMMA}{PMC R M\_B FAFVZERO;}


\Expl
Calculation of the tree-level matrix element for the process ${ B^{+}} \rightarrow l^{+} \nu_{l} \gamma$ 
  ~\cite{Korchemsky00}.
 
\Example
The example shows how to generate ${ B^{+}} \rightarrow l^{+} \nu_{l} \gamma$.
\begin{verbatim}
Decay B0
1.0000 e+  nu_e  gamma  LNUGAMMA 0.35 3.0 5.0 0;
Enddecay
\end{verbatim}

\Notes See the citation given above for more detail.

Arg(0) is the photon mass cutoff in $GeV$, Arg(1) is $R$ in $GeV^{-1}$, Arg(2) is $m_b$ in $GeV$, and Arg(3) is set to 0 if the user wants $|f_{a}/f_{v}| = 1$, and set to 1 if the user wants $f_{a}/f_{v} = 0$. Arg(3)
is optional, defaulting to 0. 


\Model{MELIKHOV}

\label{MELIKHOV}

\Auth{Lange}

\Usage{M L NU}{;}

\Expl
Implements the form factor model for $B \rightarrow \rho \ell \nu$
according to Melikhov, as described in hep-ph/9603340.  There is
one argument, which should be an integer between 1 and 4.  The
arguement sets which set of form factors from Melikhov should be used.  
 
\Example
The example shows how to generate $B^0\rightarrow \rho^- \mu^+ \nu_{\mu}$.
\begin{verbatim}
Decay B0
1.000 rho- mu+ nu_mu MELIKHOV 1;
Enddecay
\end{verbatim}

%\Notes

\Model{OMEGA\_DALITZ}

\label{omegadalitz}

\Auth{Lange}

\Usage{P1 P2 P3}{;}

\Expl
The dalitz amplitude for the decay $\omega\rightarrow \pi^+\pi^-\pi^0$.
The amplitude for this process is given by 
$A=\epsilon_{\mu\nu\alpha\beta}
p^{\mu}_{\pi^+}p^{\nu}_{\pi^-}p^{\alpha}_{\pi^0}\varepsilon^{\beta}$.

\Example
\begin{verbatim}
Decay omega
1.000 pi+  pi-  pi0           OMEGA_DALITZ;
Enddecay
\end{verbatim}



\Model{PARTWAVE}

\label{partwave}

\Auth{Ryd}

\Usage{M D1 D2}{Amplitudes;}

\Expl
This model is similar to the {\tt HELAMP} model in that it allows
any two-body decay specified by the partial wave amplitudes. This model
translates the partial wave amplitudes to helicity amplitudes using the 
Jacob Wick transformation. The partial wave amplitudes are complex numbers,
specified as a magnitude and a phase. The amplitudes $M_{LS}$ are
sorted on the highest value of $L$ and then on the highest value of $S$.

\Example
Decay of $B^0->D^*\rho$ in this example would be in pure $P$-wave.
\begin{verbatim}
Decay B+
1.000   anti-D*0  rho+           PARTWAVE 0.0 0.0 1.0 0.0 0.0 0.0;
Enddecay
\end{verbatim}

\Notes
This model has been tested on
some special cases, but further testing is needed.





\Model{PHSP}

\label{phsp}

\Auth{Ryd}

\Usage{P1 P2 ... PN}{;}

\Expl
Generic phase space to n-bodies. All spins of particles in
the initial state and the final state are averaged.

\Example
As an example of using this model the decay 
$D^0\rightarrow K^{*-}\pi^+\pi^0\pi^0$
is used.
\begin{verbatim}
Decay D0
1.000 K*-  pi+ pi0 pi0                    PHSP;
Enddecay
\end{verbatim}

\Model{PTO3P}
\label{pto3p}
\Auth{Dvoretskii}

The \texttt{PTO3P} model is a generic decay-file driven model for simulating 
decays of a scalar (typically pseudoscalar, hence the \texttt{P}) particle
into a final state composed of three scalar particles. $B^+\to K^+\pi^+\pi^-$
would be an example of such a decay.

It is possible to specify several channels through which the decay can proceed.
The interference effects are proprely handled by the model. It is also possible
to include time-dependent mixing in decays of neutral mesons.

\Expl
For an example of a decay, see the example below.
The first two parameters specify the PDF maximum (e.g. \texttt{MAXPDF 116.2}).
The PDF maximum is needed to perform accept/reject during generation. 
Alternatively if the maximum is not known one can specify the number of points
that will be sampled \texttt{SCANPDF 10000}. The PDF will be evaluated at each
point. To be conservative this maximum will be increased by 20\%. Typically
it's a good idea to do the scan once for a large number of events. Determine
the maximum and then put it explicitly in the decay file.

The other parameters are grouped into partial \texttt{AMPLITUDE} specifications and 
\texttt{COEFFICIENT} specifications. Compelex coefficients can be in cartesian 
or polar coordinates. The keywords are: \texttt{CARTESIAN} for 
cartesian coordinates and \texttt{POLAR\_RAD} and \texttt{POLAR\_DEG} for polar coordinates.   


Partial amplitudes can be either \texttt{PHASESPACE} or \texttt{RESONANCE}.
For amplitudes describing intermediate resonances one should specify which 
two particles form the resonance (\texttt{AB}, \texttt{BC}, \texttt{CA}),
and the parameters of the resonance - spin, mass and width. The resonance parameters 
can be specified, either as three numbers or as the particle name. In the latter case
the parameters will be taken from the evt.pdl file. Finally, it's possible to 
get the spin from the evt.pdl file and override the mass and the width of the particle.
Examples:

\begin{verbatim}
RESONANCE BC 1 0.77 0.15
RESONANCE BC rho+
RESONANCE BC rho+ 0.77 0.15
\end{verbatim}
    
\texttt{ANGULAR AB} declares between which two particles the helicity angle will be evaluated.
(The rest frame was specified previously, e.g. \texttt{RESONANCE BC}. This disambiguates
the sign of the amplitude.

\texttt{TYPE} specifies the type of the propagator. Choose between non-relativistic
Breit-Wigner (\texttt{NBW}), relativistic Zemach expression (\texttt{RBW\_ZEMACH}),
Kuehn-Santamaria propagator (\texttt{RBW\_KUEHN}) and relativistic propagator 
used e.g. in \texttt{CLEO hep-ex/0011065} (\texttt{RWB\_CLEO}). 

Finally it's possible to supply Blatt-Weisskopf form factors at the 
production (birth) vertex of the resonance and its decay vertex.

\begin{verbatim}
DVFF BLATTWEISSKOPF 3.0
BVFF BLATTWEISSKOPF 1.0
\end{verbatim} 




For decays with mixing, e.g. $B^0 \to \pi^+ \pi^- \pi^0$ first 
specify partial amplitudes for $B^0 \to \pi^+ \pi^- \pi^0$,
then stick in keyword \texttt{CONJUGATE} followed by mixing 
parameters (currently dm). Then specify partial amplitudes for
$\bar{B^0}   \to \pi^+ \pi^- \pi^0$ in the usual way.


For very narrow resonances the generation efficiency may be very low. 
In that case one should use pole compensation. In PTO3P pole-compensation is 
automatically turned on for all resonances. The pole-compensator PDF is created by 
the same factory that creates the amplitude. EvtDalitzBwPdf is used for that purpose.
If you would like to switch off pole-compensation you'll need to edit EvtPto3PAmpFactory.cc,
there is no way to control it via the decay file at this point.

\Example
The example below shows the decay $D^+\to \bar K^0\pi^+\pi^0$ including
the $\rho^+$ and $\bar K^{*0}$ resonances.

\begin{verbatim}
Decay D+

1.0 anti-K0 pi+ pi0 PTO3P       
        MAXPDF          75.0
        #SCANPDF 10000
        #gives 73.5

        # Non-resonant
        AMPLITUDE       PHASESPACE
        COEFFICIENT     POLAR_RAD       0.9522  -1.8565

        # rho+ (770) 
        AMPLITUDE       RESONANCE       BC      rho+            0.7699  0.1512
                        ANGULAR         AC
                        TYPE            RBW_CLEO
                        DVFF            BLATTWEISSKOPF  25.38
        COEFFICIENT     POLAR_RAD       0.389   0.0 

        # anti-K*0 (770) 
        AMPLITUDE       RESONANCE       AC      anti-K*0        0.89159 0.0498
                        ANGULAR         BC
                        TYPE            RBW_CLEO
                        DVFF            BLATTWEISSKOPF  10.15
        COEFFICIENT     POLAR_RAD       0.194   0.7191
        ;
Enddecay

\end{verbatim}




\Model{SINGLE}

\label{single}

\Auth{Ryd}

\Usage{P}{pmin pmax [cthetamin cthetamax [phimin phimax]];}

\Expl
Generates single particle rays in the region of phase space 
specified by the arguments. This single particle generator 
generates decays uniformly in the parents rest frame in 
the momentum range from {\tt pmin} to {\tt pmax}. However,
the range of $\theta$ is specified in the lab frame.

The last two and four arguments need not be specified. If
the last two are omitted the $\phi$ range is from $0$ to $2\pi$
and if the last four arguments are omitted the $\cos\theta$ range
is from $-1$ to $+1$.

\Example 
Generates $\mu^+$ with momentum from 0.5 to 1.0 GeV over $4\pi$.
\begin{verbatim}
Decay Upsilon(4S)
1.000 mu+       SINGLE  0.5 1.0 -1.0 1.0 0.0 6.283185;
Enddecay
\end{verbatim}

or simply

\begin{verbatim}
Decay Upsilon(4S)
1.000 mu+       SINGLE  0.5 1.0;
Enddecay
\end{verbatim}

\Model{SLN}

\label{sln}

\Auth{Songhoon,Ryd}

\Usage{L N}{;}

\Expl
This decay generates the decay of a scalar to a lepton and a 
neutrino. The amplitude for this process is given by
$A=P^{\nu}\langle \ell | (V-A)_{\nu} | \nu \rangle$.

\Example 
As an example of using this model the decay $D_s^+\rightarrow \mu^+\bar\nu$
is used.
\begin{verbatim}
Decay DS+
1.000 mu+  nu_mu                         SLN;
Enddecay
\end{verbatim}



\Model{SLPOLE}

\label{SLPOLE}

\Auth{Lange}

\Usage{M L NU}{arguments;}

\Expl
Implements a semileptonic decay according to a pole form parametrization.
For definition of the form factors that are used see 
section~\ref{sect:EvtSLPole}.
 
\Example
The example shows how to generate $B^0\rightarrow \rho^- \mu^+ \nu_{\mu}$.
\begin{verbatim}
Decay B0
1.000 rho- mu+ nu_mu SLPOLE 0.27 -0.11 -0.75 1.0 0.23 -0.77 
             -0.40 1.0 0.34 -1.32 0.19 1.0 0.37 -1.42 0.50 1.0;

Enddecay
\end{verbatim}

%\Notes





\Model{SSD\_CP}

\label{ssdcp}

\Auth{Ryd}

\Usage{S D}{dm dgog |q/p| arg(q/p) |A\_f| argA\_f |barA\_f| argbarA\_f\\
            |A\_barf| argA\_barf |barA\_barf| argbarA\_barf |z| arg(z);}

\Expl
This model simulates the decay of a $B$ meson to a scalar and
one other particle of arbitrary (integer) spin.
An example of using this model is $B\to J/\psi K_S$
\begin{verbatim}
Decay B0
1.000 J/psi K0S    SSD_CD dm dgog |qop| arg(qop) 
                          |Af| arg(Af) |Abarf| arg(Abarf) 
                          |Afbar| arg(Afbar) |Abarfbar| arg(Abarfbar)
                          |z| arg(z);
Enddecay
\end{verbatim}

where {\tt dm} is the mass difference of the two mass eigenstates, {\tt dgog} 
is $2y$, $y\equiv (\Gamma_H-\Gamma_L)/(\Gamma_H+\Gamma_L)$. {\tt qop} is
$q/p$ where $|B_{L,H}\rangle=p|B^0\rangle\pm q|\bar B^0\rangle$.
{\tt Af} and {\tt Abarf} are the amplitudes for the decay of a $B^0$ and a 
$\bar B^0$ respectively to the final state $f$. The set of amplitudes,
{\tt Afbar} and {\tt Abarfbar} corresponds to the decay to the 
$CP$ conjugate final state. These amplitudes are optional and are by
default $A_{\bar f}=\bar A^*_{f}$ and ${\bar A}_{\bar f}=A^*_f$,
consistent with $CPT$ for a common final state of the $B^0$ and $\bar B^0$.
However, in modes such as $B\to D^*\pi$ it is usefull to be able to 
specify these amplitudes separately. 

The example below shows the decays $B\to J/\psi K_S$ and $B\to J/\psi K_L$
\begin{verbatim}
Define dm 0.472e12
Define minusTwoBeta -0.85
Decay B0
0.5000 J/psi K0S   SSD_CD dm 0.0 1.0 minusTwoBeta 1.0 0.0 -1.0 0.0;
0.5000 J/psi K0L   SSD_CD dm 0.0 1.0 minusTwoBeta 1.0 0.0  1.0 0.0;
Enddecay
\end{verbatim}
Note that the sign of the amplitude for the $\bar B^0$ decay have the 
oposite sign for the $K_S$ as this final state is odd under parity.

To generate the final state $\pi^+\pi^-$. 
\begin{verbatim}
Define dm 0.472e12
Define minusTwoBeta -0.85
Define gamma 1.0
Decay B0
1.0000 pi+ pi-   SSD_CD dm 0.0 1.0 minusTwoBeta 1.0 gamma 1.0 -gamma;
Enddecay
\end{verbatim}
These examples have used $|q/p|=1$ and $\Delta\Gamma=0$. An
example with non-trivial values for these parameters
would be $B_s\to J/\psi \eta$
\begin{verbatim}
Define dms 14e12
DEfine dgog 0.1
Decay B_s0
1.0000 J/psi eta  SSD_CD dms dgog 1.0 0.0 1.0 0.0 1.0 0.0;
Enddecay
\end{verbatim}

This model can also be used for final states that are not $CP$
eigenstates, such as $B^0\to D^{*+}\pi^-$
and $B^0\to D^{*-}\pi^+$. We can generate these decays using
\begin{verbatim}
Define dms 14e12
Define minusTwoBeta -0.85
Define gamma 1.0
Decay B0
1.0000 D*+ pi-  SSD_CD dm 0.0 1.0 minusTwoBeta 1.0 0.0 0.3 gamma;
Enddecay
\end{verbatim}
Where the Cabibbo-suppressed decay has a relative strong phase $\gamma$ with
respect to the Cabibbo-favored decay.



\Notes
For more details about the treatment of CP violating decays 
see Section~\ref{sect:cpviolation}.




\Model{SSS\_CP}

\label{ssscp}

\Auth{Ryd}

\Usage{S S}{ALPHA dm cp |A| argA |barA| argbarA;}

\Expl
Decay of a scalar to two scalar and allows for CP violating 
time asymmetries. The first argument is the relevant CKM
angle in radians. The second argument is the mass difference
in s$^-1$ (approx $0.5\times 10^{12}$). 
cp is the CP of the final state, it is $\pm 1$. 
Next is the amplitude 
of a $B^0$ to decay to the final state, where the third argument is the 
magnitude of the amplitude and the fourth is the phase. The
last two arguments are the magnitude and phase of the
amplitude for a decay of a $\bar B^0$ to decay to the 
final state. This model then uses these amplitudes 
together with the time evolution of the
$B\bar B$ system and the flavor of the other
$B$ to generate the time distributions.

\Example
This example decays the $B$ meson to $\pi+\pi^-$
\begin{verbatim}
Decay B
1.000 pi+  pi-                  SSS_CP alpha dm 1.0 0.0 1.0 0.0;
Enddecay
\end{verbatim}

\Notes
For more details about the treatment of CP violating decays 
see Section~\ref{sect:cpviolation}.


\Model{SSS\_CP\_PNG}

\label{ssscppng}

\Auth{Ryd, Kuznetsova}

\Usage{S S}{\,BETA GAMMA DELTA dm cp $|A_{tree}|$ $|A_{tree}|/|A_{penguin}|$;}

\Expl
This model takes into account penguin contributions in $B \rightarrow \pi \,\pi$ decays.  It assumes single (top) quark dominance for the penguin.  The first two arguments are the relevant CKM angles in radians; the third argument is the relative strong phase in radians; dm is the mass difference
in s$^{-1}$ (approx $0.5\times 10^{12}$); cp is the CP of the final state; $|A_{tree}|$ is the tree-level amplitude, and ${|A_{tree}|}/{|A_{penguin}|}$ is the ratio of the amplitudes for the tree and penguin diagrams ($\approx$ 0.2 for this decay mode).  This model automatically takes into account the correct number of $B^{0}$ tags for this decay, which is given by:
\begin{equation}
f =  \frac{|\overline{A}_{f}|^2 \left(1 + |\overline{r}_{f}|^2 + \frac{(1 - |\overline{r}_{f}|^2)}{1+x_{d}^2} \right)}{|\overline{A}_{f}|^2 \left(1 + |\overline{r}_{f}|^2 + \frac{(1 - |\overline{r}_{f}|^2)}{1+x_{d}^2} \right) + |{A}_{f}|^2 \left(1 + |r_{f}|^2 + \frac{(1 - |r_{f}|^2)}{1+x_{d}^2} \right)}
\end{equation}    
where $x_{d} \equiv \frac{\Delta m}{\Gamma} \approx 0.65$, and  
\begin{equation}
r_{f} = e^{2i\,\phi_{M}}\,\frac{\overline{A}_{f}}{A_{f}}, \,\,\overline{r}_{f} = \frac{1}{r_{f}}
\end{equation}
$\phi_{M}$ being the mixing angle, and the amplitude $A_{f}$ being:
\begin{equation}
A_{f} \equiv A(B^{0} \rightarrow \pi^{+} \pi^{-}) = A_{t}\,e^{i\phi_{t}} + A_{p}\,e^
{i\phi_{p}}\,e^{i\delta}
\end{equation}
with
\begin{equation}
\overline{A}_{f} \equiv A(\overline{B}^{0} \rightarrow \pi^{+} \pi^{-}) = A_{t}\,e^{-i\phi_{t}} + A_{p}\,e^{-i\phi_{p}}\,e^{i\delta}
\end{equation}

Here, $A_{t}$, $\phi_{t}$ are tree-level amplitude and phase, respectively, $A_{p}$, $\phi_{p}$
 are those for the penguin, and $\delta$ is the relative strong phase.


\Example
This example generates $B^{0} \rightarrow \pi^{+}\, \pi^{-}$.
\begin{verbatim}
Decay B0
1.000 pi+  pi-    SSS_CP_PNG beta gamma 0.1 dm 1.0 1.0 0.2;
Enddecay
\end{verbatim}

\Notes
For more details about the treatment of CP violating decays 
see section~\ref{sect:cpviolation}.


\Model{STS}

\label{sts}

\Auth{Ryd}

\Usage{T S}{;}

\Expl
This model decays a scalar meson to a tensor and a scalar.

\Example
This example decays the $B^+$ meson to $D_2^*0\pi^+$
\begin{verbatim}
Decay B+
1.000 D_2*0  pi+                  STS;
Enddecay
\end{verbatim}

%\Notes



\Model{STS\_CP}

\label{stscp}

\Auth{Ryd}

\Usage{T S}{ALPHA dm cp |A| argA |barA| argbarA;}

\Expl
Decay of a scalar to a tensor and a scalar and allows for CP violating 
time asymmetries. The first argument is the relevant CKM
angle in radians. The second argument is the mass difference
in s$^-1$ (approx $0.5\times 10^{12}$).
cp is the CP of the final state, it is $\pm 1$. 
Next is the amplitude 
of a $B^0$ to decay to the final state, where the third argument is the 
magnitude of the amplitude and the fourth is the phase. The
last two arguments are the magnitude and phase of the
amplitude for a decay of a $\bar B^0$ to decay to the 
final state. This model then uses these amplitudes 
together with the time evolution of the
$B\bar B$ system and the flavor of the other
$B$ to generate the time distributions.

\Example
This example decays the $B$ meson to $a_2^0\pi^0$
\begin{verbatim}
Decay B0
1.000 a_20  pi0                  STS_CP alpha dm 1.0 0.0 1.0 0.0;
Enddecay
\end{verbatim}

\Notes
For more details about the treatment of CP violating decays 
see section~\ref{sect:cpviolation}.




\Model{SVP\_HELAMP}

\label{svphelamp}

\Auth{Ryd}

\Usage{V P}{|H+| argH+ |H-| argH-;}

\Expl
The decay of a scalar to a vector and a photon. This decay is
parameterized by the helicity amplitudes $H_+$ and $H_-$.
For more information about helicity amplitudes see 
Section~\ref{sect:helampconventions}/

\Example
\begin{verbatim}
Decay B0
1.000 K*0  gamma                  SVP_HELAMP 1.0 0.0 1.0 0.0;
Enddecay
\end{verbatim}

\Model{SVS}

\label{svs}

\Auth{Ryd}

\Usage{V S}{;}

\Expl
The decay of a scalar to a vector and a scalar. The first 
daughter is the vector meson.

\Example
As an example we consider $B\rightarrow D^*\pi$.

\begin{verbatim}
Decay B0
1.000 D*+  pi-                  SVS;
Enddecay
\end{verbatim}


\Model{SVS\_CP}

\label{svscp}

\Auth{Ryd}

\Usage{V S}{ALPHA dm cp |A| argA |barA| argbarA;}

\Expl
Decay of a scalar to a vector and a scalar and allows for CP violating 
time asymmetries. The first daughter has to be the vector.
The first argument is the relevant CKM
angle in radians. The second argument is the mass difference
in s$^-1$ (approx $0.5\times 10^{12}$). 
cp is the CP of the final state, it is $\pm 1$. 
Next is the amplitude 
of a $B^0$ to decay to the final state, where the third argument is the 
magnitude of the amplitude and the fourth is the phase. The
last two arguments are the magnitude and phase of the
amplitude for a decay of a $\bar B^0$ to decay to the 
final state. This model then uses these amplitudes 
together with the time evolution of the
$B\bar B$ system and the flavor of the other
$B$ to generate the time distributions.


\Example
This example decays the $B^0$ meson to $J/\Psi K_s$
\begin{verbatim}
Decay B0
1.000 J/psi  K_S0                  SVS_CP beta dm 1.0 0.0 1.0 0.0;
Enddecay
\end{verbatim}

\Notes
For more details about the treatment of CP violating decays 
see Section~\ref{sect:cpviolation}.


\Model{SVS\_CP\_ISO}

\label{svscpiso}

\Auth{NK}

\Usage{V S}{beta dm flip |$T^{+0}$| arg$T^{+0}$ |$\overline{T^{+0}}$| arg$\overline{T^{+0}}$ \\
               |$T^{0+}$| arg$T^{0+}$ |$\overline{T^{0+}}$| arg$\overline{T^{0+}}$ \\
               |$T^{+-}$| arg$T^{+-}$ |$\overline{T^{+-}}$| arg$\overline{T^{+-}}$ \\
               |$T^{-+}$| arg$T^{-+}$ |$\overline{T^{-+}}$| arg$\overline{T^{-+}}$ \\
               |$P_{0}$| arg$P_{0}$ |$\overline{P_{0}}$| arg$\overline{P_{0}}$ \\
               |$P_{2}$| arg$P_{2}$ |$\overline{P_{2}}$| arg$\overline{P_{2}}$;}

\Expl
This model considers $B$ decays into a vector ($V$) and a scalar ($S$) 
from the point of view of isospin analysis.  The vector should always 
be listed first.  For the three $B^{0}$ (or $\overline{B}^{0}$) 
modes ($B^{0} \rightarrow V^{+} S^{-}$, $B^{0} \rightarrow V^{-} 
S^{+}$, and $B^{0} \rightarrow V^{0} S^{0}$),
it takes into account mixing, and generates the corresponding 
CP-violating asymmetries.  It can also be used for the two 
isospin-related $B^{+}$ 
($B^{-}$) modes (e.g., $B^{+} \rightarrow V^{+} S^{0}$ and $B^{+} 
\rightarrow V^{0} S^{+}$), as all five modes should 
be treated together in this approach.  
Following the conventions of Lipkin, Nir, Quinn, and Snyder 
(Phys. Rev. D{\bf 44}, 1454 (1991)), the various decay 
amplitudes can be written as follows:
\begin{equation}
A(B^{+} \rightarrow V^{+} S^{0}) \equiv \sqrt{2}A^{+0} = T^{+0} + 2 P_{1} 
\end{equation}
\begin{equation}
A(B^{+} \rightarrow V^{0} S^{+}) \equiv \sqrt{2}A^{0+} = T^{0+} - 2 P_{1} 
\end{equation}
\begin{equation}
A(B^{0} \rightarrow V^{+} S^{-}) \equiv A^{+-} = T^{+-} + P_{1} + P_{0} 
\end{equation}
\begin{equation}
A(B^{0} \rightarrow V^{-} S^{+}) \equiv A^{-+} = T^{-+} - P_{1} + P_{0} 
\end{equation}
\begin{equation}
A(B^{0} \rightarrow V^{0} S^{0}) \equiv 2 A^{00} = T^{0+} + T^{+0} - T^{-+} - T^{+-} - 2 P_{0} 
\end{equation}

where the amplitudes $T^{ij}$ contain no penguin contributions, $P_{1}$ is penguin amplitude for the final $I$ = 1 state, and $P_{0}$, for the final $I$ = 0 state.  

The model's arguments are:
\begin{itemize}
\item
beta = corresponding CKM angle
\item
dm = $B^{0} \overline{B}^{0}$ mass difference ($\approx 0.5 \times 10^{12} s^{-1}$).   
\item
``flip'' sets the fraction of $B \rightarrow f$ to $B \rightarrow \overline{f}$ decays, where the state specified in the .DEC table is considered the ``$f$'' state.  Set it to 0 to always get the $B \rightarrow f$ case, and to 1 to always get the $B \rightarrow \overline{f}$ case.  
\item
$|T^{+0}|$, $\varphi_{T^{+0}}$  = magnitude and phase of the corresponding amplitude
\item
$|\overline{T^{+0}}|$, $\varphi_{\overline{T^{+0}}}$  = magnitude and phase of the corresponding amplitude for the CP-conjugate process.
\item
$|T^{0+}|$, $\varphi_{T^{0+}}$  = magnitude and phase of the corresponding amplitude
\item
$|\overline{T^{0+}}|$, $\varphi_{\overline{T^{0+}}}$  = magnitude and phase of the corresponding amplitude for the CP-conjugate process.
\item
$|T^{+-}|$, $\varphi_{T^{+-}}$  = magnitude and phase of the corresponding amplitude
\item
$|\overline{T^{+-}}|$, $\varphi_{\overline{T^{+-}}}$  = magnitude and phase of the corresponding amplitude for the CP-conjugate process.
\item
$|T^{-+}|$, $\varphi_{T^{-+}}$  = magnitude and phase of the corresponding amplitude
\item
$|\overline{T^{-+}}|$, $\varphi_{\overline{T^{-+}}}$  = magnitude and phase of the corresponding amplitude for the CP-conjugate process.
\item
$|P_{0}|$, $\varphi_{P_{0}}$  = magnitude and phase of the corresponding amplitude
\item
$|\overline{P_{0}}|$, $\varphi_{\overline{P_{0}}}$  = magnitude and phase of the corresponding amplitude for the CP-conjugate process.
\item
$|P_{2}|$, $\varphi_{P_{2}}$  = magnitude and phase of the corresponding amplitude
\item
$|\overline{P_{2}}|$, $\varphi_{\overline{P_{2}}}$  = magnitude and phase of the corresponding amplitude for the CP-conjugate process.
\end{itemize}

\Example
This example decays the $B^{0}$ meson to $a_{1}^{-} \pi^{+}$ assuming no penguin contributions
\begin{verbatim}
Decay B0
1.000 a_1-  pi+      SVS_CP_ISO beta dm 0.0 1.0 0.0 1.0 0.0
                                              1.0 0.0 1.0 0.0
                                              1.0 gamma 3.0 -gamma
                                              3.0 gamma 1.0 -gamma
                                              0.0  0.0  0.0 0.0;

Enddecay
\end{verbatim}

\Notes
For more details about the treatment of CP violating decays 
see section~\ref{sect:cpviolation}.


\Model{SVS\_NONCPEIGEN}

\label{svsnoncpeigen}

\Auth{Natalia,Ryd}

\Usage{V S}{\\[0.05truein] 
	\begin{tabular}{ll} 
	alpha dm flip 
   &  |$A_{f}$| arg$A_{f}$ |$\overline{A}_{f}$| arg$\overline{A}_{f}$ 
		\\[0.05truein]
   & |$A_{\overline{f}}$| arg$A_{\overline{f}}$
   |$\overline{A}_{\overline{f}}|$ args$\overline{A}_{\overline{f}}$; 
	[these are optional]\\
	\end{tabular}
}

\Expl
This model allows to generate scalar $\rightarrow$ vector + scalar
decays, where the final state is not a CP-eigenstate.  The {\tt flip}
parameter sets the fraction of $f$ to $\bar f$ decays, where the state
specified in the .DEC table is considered the ``$f$'' state.  Set it
to 0 to always get the final $f$ case, and to 1 to always get the
$\overline{f}$ final state. Otherwise, set it to 0.5 to get the
physical situation. This model automatically generates the correct
number of $B^{0}$ and $\bar{B}^{0}$ tags, depending on the specified
amplitudes.

Note that the last four parameters are optional. If they are not
specified, then they are evaluated from the following relations
between the complex amplitudes:
%%%
\begin{eqnarray}
          A_{\overline f} &=& \overline A_f \nonumber\\
\overline A_{\overline f} &=&           A_f  
\label{eq:svs_noncpeigen}
\end{eqnarray}

\Example
This example will generate a mixture of $a_1^+ \pi^-$ and $a_1^- \pi^+$
final states with the appropriate number of $B^0$ and $\bar B^0$ tags.
Note that the last 4 parameters could have been omitted, since they 
agree with Eq.~(\ref{eq:svs_noncpeigen})

\begin{verbatim}
Alias MYB B0
Decay Upsilon(4S)
1.00  MYB B0
Enddecay
Decay MYB
1.000 a_1- pi+    SVS_NONCPEIGEN alpha dm 0.5 
                                           1.0 0.0 3.0 0.0 
                                           3.0 0.0 1.0 0.0;
Enddecay
\end{verbatim}
For $B \rightarrow D{*\pm}\pi^\mp$, use the CKM phase betaPlusHalfGamma.

{\bf Note:} Temporarily, this model only works for B0, not anti-B0. This
will be fixed later.

%Note that 0.5 will $50\%$ of the time charge conjugate the final state
%$a_1^-\pi^+$ and that the tag $B$, the other $B$ in the $\Upsilon(4S)$ decay
%will be selected as either a $B^0$ or a $\bar B^0$ according to the
%amplitudes that are specified. Also, note that the $B$ that decays into
%the CP mode has to be an alias for either a $B^0$ or a $\bar B^0$, it
%can not be a $B^0$ or a $\bar B^0$ directly, since the {\tt SVS\_NONCPEIGEN}
%model will create the other $B$ either as a $B^0$ or a $\bar B^0$!


\Notes
For more details about the treatment of CP violating decays 
see Section~\ref{sect:cpviolation}.



\Model{SVV\_CP}

\label{svvcp}

\Auth{Ryd}

\Usage{V1 V2}{BETA dm eta |G1+| argG1+ |G0+| argG0+\\ |G1-| argG1-;}

\Expl
Decay of a scalar to two vector mesons and allows for CP violating 
time asymmetries. The first argument is the relevant CKM
angle in radians. The second argument is the $B^0-\bar B^0$ mass difference
in s$^{-1}$ (approximately $0.5\times 10^{12}$). 
The next argument is called $\eta$ in Ref.~\cite{Duni91} and
is either $+1$ or $-1$. The last six arguments are $G_{1+}$,
$G_{0+}$, and $G_{1-}$, and are expressed as their absolute values
and phases again the definition of these parameters are in Ref.~\cite{Duni91}.
This model then uses these amplitudes 
together with the time evolution of the
$B\bar B$ system and the flavor of the other
$B$ to generate the time distributions.


\Example
This example decays the $B^0$ meson to $J/\Psi K^{*0}$
\begin{verbatim}
Decay B0
1.000 J/psi  K*0        SVV_CP beta dm 1.0 1.0 0.0 1.0 0.0 1.0 0.0;
Enddecay
\end{verbatim}

\Notes
For more details about the treatment of CP violating decays 
see Section~\ref{sect:cpviolation}. Note that the value of $\eta$
depends on how the $K^{*0}$ decays, it is either $+1$ or $-1$ 
depending on weather a $K_S$ or a $K_L$ is produced. (It needs
to be checked which sign goes with the $K_S$ and the $K_L$.





\Model{SVV\_CPLH}

\label{svvcplh}

\Auth{Ryd}

\Usage{V1 V2}{BETA dm eta |G1+| argG1+ |G0+| argG0+\\ |G1-| argG1-;}

\Expl
Decay of a scalar to two vector mesons and allows for CP violating 
time asymmetries including different lifetimes for the different
mass eigenstates, the Light and Heavy state. This model is particularly
intended for decays like $B_s\rightarrow J/\psi \phi$. The first argument is 
the relevant CKM angle in radians. The second argument is the 
$B_s-\bar B_s$ mass difference
in s$^{-1}$ ($>1.8\times 10^{12}$). 
The width difference is not an input parameter to the model.
It is determined via the definition of B\_s0L and B\_s0H in the evt.pdl.
%The third argument is the widht
%difference between the two states. Given in units of mm, i.e. 
%$c\hbar/(\Delta \Gamma)$.
The next argument is called $\eta$ in Ref.~\cite{Duni91} and
is either $+1$ or $-1$. The last six arguments are $G_{1+}$,
$G_{0+}$, and $G_{1-}$, and are expressed as their absolute values
and phases again the definition of these parameters are in Ref.~\cite{Duni91}.
This model then uses these amplitudes 
together with the time evolution of the $B_s$ to generate the time
dependent angular distributions.


\Example
This example decays the $B_s$ meson to $\phi K^{*0}$
\begin{verbatim}
Decay B_s0
1.000   J/psi  phi   SVV_CPLH  0.4 3.0e12 2.0 1 1.0 0.0 1.0 0.0 1.0 0.0;
Enddecay
\end{verbatim}

\Notes
For more details about the treatment of CP violating decays 
see Section~\ref{sect:cpviolation}. This code is not well tested
at all. Please be aware that there can be serious mistakes in
this model!



\Model{SVS\_CPLH}

\label{svscplh}

\Auth{Ryd}

\Usage{V S}{dm dGoG |q/p|  arg(q/p) |Af| arg(Af) |Abarf| arg(Abarf);}

\Expl
Decay of a neutral $B$ meson to a scalar and a vector CP eigenstate,
e.g. $B^0\to J/\psi K_S$. The first argument is the $B^0-\bar B^0$ 
mass difference. The second argument in $\Delta\Gamma/\Gamma$. The 
third and fourth argument is the magnitude and phase of $q/p$, and the 
last four arguments are the magnitude and phases of the amplitude
for $B^0$ and $\bar B^0$ to decay to the final state $f$.



\Example
This example decays the $B^0$ meson to $J/\psi K_S$
\begin{verbatim}
Decay B0
1.000   J/psi  K_S0   SVS_CPLH  0.472e12 0.1  1.0  0.7  1.0 0.0 1.0 0.0;
Enddecay
\end{verbatim}

%\Notes



\Model{SVV\_NONCPEIGEN}

\label{svvnoncpeigen}

\Auth{ Kurup}

\Usage{V1 V2}{\\[0.05truein] 
	\begin{tabular}{lll} 
 dm beta gamma  
& |$A_{+f}$| arg$A_{+f}$ |$A_{0f}$| arg$A_{0f}$ |$A_{-f}$| arg$A_{-f}$ 
	&\\[0.05truein]
%
& |$\overline{A}_{+f}$| arg$\overline{A}_{+f}$ |$\overline{A}_{0f}$|
 arg$\overline{A}_{0f}$ |$\overline{A}_{-f}$| arg$\overline{A}_{-f}$
	&\\[0.05truein]
%
& |$A_{+\overline{f}}$| arg$A_{+\overline{f}}$ |$A_{0\overline{f}}$|
 arg$A_{0\overline{f}}$ |$A_{-\overline{f}}$| arg$A_{-\overline{f}}$
	&[optional] \\[0.05truein]
%
& |$\overline{A}_{+\overline{f}}$| arg$\overline{A}_{+\overline{f}}$
|$\overline{A}_{0\overline{f}}$| arg$\overline{A}_{0\overline{f}}$
|$\overline{A}_{-\overline{f}}$| arg$\overline{A}_{-\overline{f}}$;
&[optional]\\ 
\end{tabular}
}

\Expl
This model is based on the SVS\_NONCPEIGEN model and allows the
generation of CP violation in scalar~$\rightarrow$~vector~+~vector
decays, where the final state is not a CP-eigenstate.  The first
argument is the $B^0-\bar B^0$ mass difference. The second argument
is the angle beta.  The third argument is the angle relevant to the
decay mode being generated.  In the example below it is gamma (in
fact, it's enough to specify 2~beta~+~gamma, perhaps in the next
round of fixes).  The next 24 arguments are the magnitudes and phases
of the amplitudes for the four types of decay, $A_f$, $\overline
A_f$, $A_{\overline f}$ and $\overline A_{\overline f}$, which are
split into the three different helicity states +, 0 and $-$.
Depending on the specified amplitudes, the final state will be charge
conjugated and the correct number of $B^0$ and $\bar B^0$ tags are
generated.

Note that the last 12 parameters are optional. If they are not
specified, then they are evaluated according to the following relation
between the complex amplitudes (with $i=+,0,-$):
%%%
\begin{eqnarray}
          A_{i \overline f} &=& \overline A_{if} \nonumber\\
\overline A_{i \overline f} &=&           A_{if}  
\label{eq:svv_noncpeigen}
\end{eqnarray}


\Example
This example will generate $B \rightarrow D^{*\pm} \rho^\mp$ 
final states with the appropriate number of $B^0$ and $\bar B^0$ tags.
The helicity amplitude parameters chosen for the first line are those
measured by CLEO. The amplitudes on the second line are identical, but
suppressed by a factor of 100. The last two lines were omitted, so that
Eq~(\ref{eq:svv_noncpeigen}) takes effect:
%%
\begin{verbatim}
Alias MYB B0
Decay Upsilon(4S)
1.00  MYB anti-B0
Enddecay
Decay MYB
1.000 D*- rho+    SVV_NONCPEIGEN dm beta gamma  
                                 0.152   1.47 0.936   0 0.317   0.19 
                                 0.00152 1.47 0.00936 0 0.00317 0.19; 
Enddecay
\end{verbatim}

{\bf Note:} Temporarily, this model only works for B0, not anti-B0. This
will be fixed later.


\Model{SVV\_HELAMP}

\label{svvhelamp}

\Auth{Ryd}

\Usage{V1 V2}{|H+| argH+ |H0| argH0 |H-| argH-; }

\Expl
The decay of a scalar to two vectors. The decay amplitude is
specified by the helicity amplitudes which are given as  arguments
for the decay. The arguments are $H_+$, $H_0$, and $H_-$. Where
these complex amplitudes are specified as magnitude and phase.
The convention for the helicity amplitudes are that of 
Jacob and Wick (at least I hope this is what it is!).
For more details about helicity amplitudes see 
Section~\ref{sect:helampconventions}.

\Example
\begin{verbatim}
Decay D0
1.000 K*0 rho0           SVV_HELAMP 1.0 0.0 1.0 0.0 1.0 0.0;
Enddecay
\end{verbatim}






\Model{TAULNUNU}

\Auth{Ryd}

\Usage{L N1 N2}{;}

\Expl
The decay of a tau to a lepton and two neutrinos. The first
daughter is the produced lepton the second is the associated
neutrino and the third is the tau neutrino. 
The amplitude for this decay is given by 
$A=\langle \tau | (V-A)_{\alpha} | \nu_{\tau} \rangle
   \langle \ell | (V-A)^{\alpha} | \nu_{\ell} \rangle$.
 

\Example
The example shows the decay $\tau\rightarrow e\nu_{e}\bar\nu_tau$
\begin{verbatim}
Decay tau-
1.000 e- anti-nu_e  nu_tau              TAULNUNU;
Enddecay
\end{verbatim}





\Model{TAUSCALARNU}

\label{tauscalarnu}

\Auth{Ryd}

\Usage{S N}{}

\Expl
The decay of a tau to a scalar meson and a tau neutrino.
The meson is the first daughter.
The amplitude for this decay is given by 
$A=\langle \tau | (V-A)_{\alpha} | \nu_{\tau} \rangle P^{\alpha}$.

\Example
An example of the use of this model is in the decay 
$\tau\rightarrow \pi\nu_{\tau}$

\begin{verbatim}
Decay tau-
1.000 pi-   nu_tau              TAUSCALARNU;
Enddecay
\end{verbatim}


\Model{TAUVECTORNU}

\label{tauvector}

\Auth{Ryd}

\Usage{V N}{}

\Expl
The decay of a tau to a vector meson and a tau neutrino.
The meson is the first daughter.
The amplitude for this decay is given by 
$A=\langle \tau | (V-A)_{\alpha} | \nu_{\tau} \rangle \varepsilon^{\alpha}$.

\Example
An example of the use of this model is in the decay 
$\tau\rightarrow \rho\nu_{\tau}$

\begin{verbatim}
Decay tau-
1.000 rho-   nu_tau              TAUVECTORNU;
Enddecay
\end{verbatim}



\Model{TSS}

\label{tss}

\Auth{Ryd}

\Usage{S1 S2}{}

\Expl
The decay of a tensor particle to two scalar mesons.

\Example
As an example the decay $D_2^{*0}\rightarrow D^0\pi^0$is used.
\begin{verbatim}
Decay D_2*0
1.000 D0   pi0              TSS;
Enddecay
\end{verbatim}



\Model{TVS\_PWAVE}

\Auth{Ryd}

\Usage{V S}{|P| argP |D| argD |F| argF;}

\Expl
The decay of a tensor particle to a vector and a scalar. The decay
takes six arguments, which parameterizes the $P$, $D$, and $F$
wave amplitudes. The first two arguments are the magnitude 
and the phase of the $P$-wave amplitude, the third and forth
are the $D$-wave amplitude and the last two are the $F$-wave
amplitude. 

\Example
The decay $D_2^{*0}\rightarrow D^{*0} \pi^0$ which is expected, 
by HQET, to be dominated by $D$ wave.
\begin{verbatim}
Decay D_2*0
1.000 D*0 pi0           TVS_PWAVE 0.0 0.0 1.0 0.0 0.0 0.0;
Enddecay
\end{verbatim}

\Notes
This model has only been used yet for $D$-wave so
further test are needed before it is safe to use for nonzero
$P$ and $F$ wave amplitudes.




\Model{VECTORISR}

\label{vectorisr}

\Auth{Zallo,Ryd}

\Usage{VECTOR GAMMA}{CSFWMN CSBKMN;}

\Expl
Generates the interaction, $e^+ e^- \rightarrow V \gamma$ where $V$
is a vector meson according
to~\cite{Bonn71}. This model should be used as a decay of the {\tt vpho}.

\Example
Example below shows how to generate the $\phi\gamma$ final state from
an virtual photon.
\begin{verbatim}
Decay vpho
1.000 phi gamma                    VECTORISR 0.878 0.95;
Enddecay
\end{verbatim}

\Notes
This model produces an unpolarized vector meson.



\Model{VLL}

\label{vll}

\Auth{Ryd}

\Usage{L1 L2}{;}

\Expl
Decay of a vector meson to a pair of charged leptons, e.g., 
$J/\psi\rightarrow\ell^+\ell^-$. The amplitude for this
process is given by $A=\varepsilon^{\mu}L_{\mu}$ where
$L_{\mu}=\langle \ell | V_{\mu} | \bar\ell \rangle$.

\Example
The example shows how to generate $J/\Psi \rightarrow e^-e^+$
\begin{verbatim}
Decay J/psi
1.000 e- e+     VLL;
Enddecay
\end{verbatim}










\Model{VSP\_PWAVE}

\label{vsppwave}

\Auth{Ryd}

\Usage{S gamma}{;}

\Expl
The decay of a vector to a scalar meson and a photon, the decay 
goes in P-wave. The first daughter is the scalar meson and the second
daughter is the photon. 

\Example
This decay is useful for example in the decay $D^{*0}\rightarrow D^0\gamma$
\begin{verbatim}
Decay D*0
1.000 D0  gamma                    VSP_PWAVE;
Enddecay
\end{verbatim}



\Model{VSS}

\label{vss}

\Auth{Ryd}

\Usage{S1 S2}{;}

\Expl
Decays a vector particle into two scalars. It generates the 
correct decay angle distributions for the produced scalars.
The amplitude for this decay is given by 
$A=\varepsilon^{\mu}v_{\mu}$ where $\varepsilon$ is the 
polarization vector of the parent particle and the $v$ is the
(four) velocity of the first daughter.

\Example
The example shows how to generate $D^{*+} \rightarrow D^{0}\pi^+$
\begin{verbatim}
Decay D*+
1.000 D0 pi+   VSS;
Enddecay
\end{verbatim}



\Model{VSS\_MIX}

\label{vssmix}

\Auth{Ryd}

\Usage{B1 B2}{dm;}

\Expl
Decays a vector particle into two scalar and generates the
correct angular and time distributions for the particles  in the
decay $\Upsilon(4S) \rightarrow B^0\bar B^0$. The mass
difference is supplied as an argument to the model

\Example
The example shows how to generate the mixture of mixed and
unmixed $B^0$ and $\bar B^0$ events.
\begin{verbatim}
Define dm 0.474e12
Decay Upsilon(4S)
0.420 B0      anti-B0                         VSS_MIX dm;
0.040 anti-B0 anti-B0                         VSS_MIX dm;
0.040 B0      B0                              VSS_MIX dm;
Enddecay
\end{verbatim}

\Notes
The user has to manually specify the fractions of mixed and un-mixed event
through the branching fraction. This means that all this model does is to
generate the right time distribution for the given final state.  Use the
new VSS\_BMIX model to generate mixing in the correct proportions using a
single decay channel.  See Section~\ref{sect:cpviolation} for more details
about how mixing is implemented and how it works with CP violation.


\Model{VSS\_BMIX}

\label{vssbmix}

\Auth{Kirkby}

\Usage{B1 B2}{dm;}

\Expl
Decays a C=-1 vector particle into two scalar particles using $B^0 \bar
B^0$-like coherent mixing. The two possible daughter particles must be
charge conjugates and have the same lifetime. Their mass difference is
supplied as an argument to the model, in units of $\hbar$/s.  
While the mass difference is a required arguement, $\Delta \Gamma / \Gamma$
and $\vert q/p \vert$ can be supplied as optional arguements,
with defaults of 0 and 1 respectively.  The examples below
illustrate how this model accomadates aliased daughters.

\Example
The example shows how to generate $\Upsilon(4S)\rightarrow B^0 \bar
B^0$ decays with coherent mixing (but without CP violating effects).
\begin{verbatim}
Define dm 0.474e12
Decay Upsilon(4S)
  1.0 B0  anti-B0  VSS_BMIX dm;
Enddecay
\end{verbatim}
to include a non-zero $\Delta \Gamma / \Gamma$:
\begin{verbatim}
Define dm 0.474e12
Define dgog 0.5
Decay Upsilon(4S)
  1.0 B0  anti-B0  VSS_BMIX dm dgog;
Enddecay
\end{verbatim}
and to specify $\vert q / p \vert$
\begin{verbatim}
Define dm 0.474e12
Define dgog 0.5
Define qoverp 1.2
Decay Upsilon(4S)
  1.0 B0  anti-B0  VSS_BMIX dm dgog qoverp;
Enddecay
\end{verbatim}
Finally, aliased particles can be generated using
this model
\begin{verbatim}
Define dm 0.474e12
alias myB0 B0
alias myanti-B0 anti-B0
Decay Upsilon(4S)
  1.0 B0  anti-B0 myB0 myanti-B0 VSS_BMIX dm;
Enddecay
\end{verbatim}
generates either {\tt B0 myanti-B0},
{\tt anti-B0 myanti-B0},
{\tt myB0 anti-B0}, or
{\tt myB0 B0}.

\Notes
This model is similar to the VSS\_MIX model, but it eliminates the need to
manually specify the fractions of mixed and un-mixed events through
branching fractions. This approach has the effect that the resulting mixing
distributions are necessarily self consistent, which is not true for the
VSS\_MIX model when using the wrong branching fractions.  See
Section~\ref{sect:cpviolation} for more details about how mixing is
implemented and how it works with CP violation.

\Model{VVPIPI}

\label{vvpipi}

\Auth{Ryd}

\Usage{V S S}{;}

\Expl
This decay model was constructed for the decay 
$\psi'\rightarrow J/\psi \pi^+\pi^-$ but should work
for any $V\rightarrow V' \pi\pi$ decay in which the
approximation that the $\pi\pi$ system can be treated
as one particle which combined with the $V'$ meson
is dominated by $S$-wave. The amplitude for the
mass of the $\pi\pi$ sstem is given by 
$A\propto (m^2_{\pi\pi}-4m^{2}_{\pi})$.

\Example
\begin{verbatim}
Decay psi(2S)
1.000 J/psi pi+ pi-           VVPIPI;
Enddecay
\end{verbatim}

%\Notes





\Model{VVS\_PWAVE}

\label{vvspwave}

\Auth{Ryd}

\Usage{V S}{|S| argS |P| argP |D| argD; }

\Expl
The decay of a vector particle to a vector and a scalar. The decay
takes six arguments, which parameterizes the $S$, $P$, and $D$
wave amplitudes. The first two arguments are the magnitude 
and the phase of the $S$-wave amplitude, the third and forth
are the $P$-wave amplitude and the last two are the $D$-wave
amplitude. 

\Example
The example below shows how to decay the $a_1^0$ in pure 
$P$ wave to $\rho\pi$.
\begin{verbatim}
Decay a_10
1.000 rho0 pi0           VVS_PWAVE 0.0 0.0 1.0 0.0 0.0 0.0;
Enddecay
\end{verbatim}

\Notes
This model has only been used yet for $P$-wave so
further test are needed before it is safe to use use
for nonzero $S$ and $D$ wave amplitudes.





\Model{WSB}

\label{wsb}

\Auth{Lange, Ryd}

\Usage{M L N}{;}

\Expl
This is a model for semileptonic decays of $B$, and $D$ mesons 
according to the WSB model~\cite{Wirbel85}. The first daughter
is the meson produced in the semileptonic decay. The second and third
argument is the lepton and the neutrino respectively.
See Section~\ref{semileptonic} for more details about semileptonic
decays.


\Example
The example shows how to generate $\bar B^0\rightarrow D^{*+}e\bar\nu$
\begin{verbatim}
Decay anti-B0
1.000 D*+ e- anti-nu_e     WSB;
Enddecay
\end{verbatim}

\Notes
This model does not include the $A_3$ form factor that is 
needed for non-zero mass leptons, i.e., tau's. If tau's
are generated the $A_3$ form factor will be zero.




















