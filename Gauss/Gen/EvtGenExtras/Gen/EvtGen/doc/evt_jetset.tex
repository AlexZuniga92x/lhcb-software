\section{Interface to JetSet}
\label{sect:jetsetinterface}
\index{jetset interface}
\index{JETSET}
\index{EvtStringParticle}

The {\tt JETSET} and {\tt JSCONT} models provides interface to JetSet.
The {\tt JSCONT} model is specialized for producing $q\bar q$ jets
from an $e^+e^-$ interaction. The {\tt JETSET} model is a generic interface
to use JetSet for decaying particles. 

When JetSet is asked to decay a particle it uses a decay table that has
been build from the EvtGen decay table, DECAY.DEC. Entries in the decay
table that uses the {\tt JETSET} model is copied into a JetSet format
decay table. This decay table is read by JetSet after EvtGen is done
reading the decay tables, including user decay tables. (This file is
currently called jet.d and is left after the jobs is finished in the 
current directory.) The decay table for JetSet is constructed from the 
list of daughters that were listed in the EvtGen decay table and the
model is taken as the argument to the {\tt JETSET} model.

To implement the partons that jetset generates a new particle in EvtGen was
created to hold the list of partons. This class is called EvtStringParticle,
the name is taken from the idea in JetSet that these partons form a string.
The EvtStringParticle is derived from EvtParticle, and addes member data to
store the four momenta and ids of the partons. 

\index{alias}
The functionality of aliases in EvtGen and the {\tt JETSET} model don't work
well together. The implementation of aliases in EvtGen creates a new decay
table for the aliased particle. However, JetSet don't have the functionality
to allow two, or more, different decay tables for the same particle. Hence,
EvtGen will not allow you to decay a particle that is an alias using the 
{\tt JETSET} model. 


