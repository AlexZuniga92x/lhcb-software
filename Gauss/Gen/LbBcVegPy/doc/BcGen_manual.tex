\documentclass[12pt]{article}
\usepackage{graphicx}
\usepackage{psfrag}
\usepackage[backref=page,citecolor=blue,colorlinks=true,
            breaklinks=true]{hyperref}


\setlength{\unitlength}{1mm}
\textwidth 15.0 true cm
\textheight 22.0 true cm
\headheight 0.0 cm
\headsep 0.0 cm
\topmargin 0.4 true in
\oddsidemargin 0.25 true in
%\input epsf
%\thispagestyle{empty}

\def \bn{\begin{enumerate}}
\def \en{\end{enumerate}}
\def\beq{\begin{eqnarray}}
\def\eeq{\end{eqnarray}}
\def\lsim{\mathrel{\rlap{\lower4pt\hbox{\hskip1pt$\sim$}}
    \raise1pt\hbox{$<$}}}         %less than or approx. symbol
\def\gsim{\mathrel{\rlap{\lower4pt\hbox{\hskip1pt$\sim$}}
    \raise1pt\hbox{$>$}}}         %greater than or approx. symbol

%\newcommand{\gsim}{\lower.7ex\hbox{$\;\stackrel{\textstyle>}{\sim}\;$}}
%\newcommand{\lsim}{\lower.7ex\hbox{$\;\stackrel{\textstyle<}{\sim}\;$}}


\begin{document}

\vskip 1cm
\begin{center}
{\Large \bf Manual of BcVegPy \vskip 0.5cm} \vskip 0cm {Jibo He,
Zhenwei Yang} \vskip 0cm \vskip 0cm {Center for High Energy Physics,
Tsinghua University} \vskip 0cm\
\end{center}

\section{Introduction}
This manual is for those who want to use BcVegPy ($B_c$ generator
``BCVEGPY" \cite{BCVEGPY} interfaced to Gauss) to generate $B_c$
states in Gauss. Users are supposed to have experience on Gauss. For
things related to Gauss, please refer to Gauss's manual. If you have
any questions or want to report bugs, please contact
he-jb02@mails.tsinghua.edu.cn or yangzhw@mail.tsinghua.edu.cn.

BCVEGPY is an event generator for the hadronic production of the
$B_c$ meson through the dominant hard subprocess $gg\rightarrow
B_c(B_c^{*})+b+\bar{c}$, written in a PYTHIA-compatible format. In
this generator, the helicity technique is used to make the program
compact, VEGAS is used to obtain an importance sampling function
(grade function) so as to increase the efficiency of event
generating. The current version of BCVEGPY is v2.0. With this
version, besides individually the event samples of the $S$-wave and
$P$-wave excited $B_c$ states,
%(denoted by $B_{cJ,L=1}^*$ or by $h_{B_c}$ and $\chi_{B_c}$),
the mixtures can be generated by relevant options too.

All the things related are included in 3 packages: LbBcVegPy,
BcVegPy, BcVegPyData.
\begin{itemize}
\item LbBcVegPy is a package like LbPythia, whose purpose is to interface BCVEGPY to Gauss.
It contains the functions related to the use of BCVEGPY (e.g., setting the parameters and
controlling the generation process) and has the following directory structure:

\begin{tabular}{|l l|}
\hline
   -cmt      & CMT requirements file \\
   -doc      & release notes \\
   -LbBcVegPy  & header files \\
   -src      & source files\\
\hline
\end{tabular}
\item BcVegPy is a package in which BCVEGPY 2.0 source files put.
\item BcVegPyData is a package in which the data, i.e., the grade files and the outer PDF
(CTEQ6L1 is used for this version) file are put.
\end{itemize}


\section{How to Use}
To use it, the version of Gauss in your server should be v24r1 or
above. That means, you should have LbBcVegPy (v1r0), BcVegPy (v1r0)
and BcVegPyData (v1r0) packages in
\$MYSITEROOT/lhcb/GAUSS/GAUSS\_v24r1/Gen/. Then you may follow the
steps below (take Gauss v24r1 as an example):

\bn
\item In the requirements file of Gauss, just below

\begin{tabular}{|l l l|}
\hline
\multicolumn{3}{|l|}{\#=====================}\\
\multicolumn{3}{|l|}{\#-- Generator}\\
use ParticleGuns & v2r0 & Gen\\
use Generators   & v4r1 & Gen\\
use LbPythia     & v*   & Gen\\
\hline
\end{tabular}

add

\begin{tabular}{|l l l|}
\hline
use LbBcVegPy      & v1r0 & Gen\\
\hline
\end{tabular}

\item In the cmt directory of Gauss, run the following command to configure Gauss again:

\begin{tabular}{|l|}
\hline
$>$ GaussEnv v24r1\\
$>$ cmt config\\
$>$ . setup.sh\\
$>$ gmake\\
\hline
\end{tabular}

\item Modify the Common option file of Gauss, \emph{Common.opts}, change the line

\begin{tabular}{|l|}
\hline
ApplicationMgr.DLLs += \{ "LbPythia", "Generators"\} ;\\
\hline
\end{tabular}

to

\begin{tabular}{|l|}
\hline
ApplicationMgr.DLLs += \{ "LbPythia", "Generators", "LbBcVegPy"\} ;\\
\hline
\end{tabular}

\item In the option directory of Gauss, create a new option file, take \emph{14900000.opts} as an example:

\begin{tabular}{|l l|}
\hline
// Option file for $B_c$ generation & \\
Generation.EventType                  &= 14900000;\\
Generation.SampleGenerationTool       &= "SignalPlain";\\
Generation.SignalPlain.ProductionTool &= "BcVegPyProduction";\\
Generation.SignalPlain.SignalPIDList  &= \{ 541 , -541 \};\\
Generation.PileUpTool                 &= "FixedNInteractions";\\
Generation.SignalPlain.BcVegPyProduction.BcVegPyCommands &= \{"mixevnt imix 1",\\
                                                     &    "mixevnt imixtype 1",\\
                                                     &    "counter ibcstate 2"\\
                                                     &    "loggrade igrade 1"\};\\
// add the dkfile of $B_c$ here & \\
.... & \\
\hline
\end{tabular}\\

The default option for BcVegPy is hardcoded in
\emph{BcVegPyProduction.cpp}. If you are not familiar with BCVEGPY,
please don't touch it. In the above option file, the line
"Generation.SignalPlain.BcVegPyProduction.BcVegPyCommands=..." is to
specify the type of $B_c$ states to generate, which obey the
following syntax:

\begin{tabular}{|l l|}
\hline
Generation.SignalPlain.BcVegPyProduction.BcVegPyCommands & \\
=\{"$\langle comonblock\rangle$ $\langle entry\rangle$  $\langle value\rangle$"\}; & \\
\hline
\end{tabular}\\

You can also use this to modify the opitons hardcoded in
\emph{BcVegPyProduction.cpp}, if you do know what you are doing.
Please note that there is only ONE space between the "commonblock"
and "entry". All that you may change is the values.
\begin{itemize}
{\item "mixevnt imix 1"\\
whether to generate the eight states ($|(c\bar{b})_1(^1S_0)\rangle$, $|(c\bar{b})_1(^3S_1)\rangle$,
$|(c\bar{b})_1(^1P_1)\rangle$, $|(c\bar{b})_1(^3P_{J=0,1,2})\rangle$, $|(c\bar{b})_8(^1S_0)g\rangle$,
$|(c\bar{b})_8(^3S_1)g\rangle$) together. "1" for yes; "0" for no.}
{\item "mixevnt imixtype 1"\\
set which type of $B_c$ state to generate and mix for the case
of generating mixtures of the eight states.
"1" for all the eight states; "2" for only $S$-wave states $|(c\bar{b})_1(^1S_0)\rangle$
and $|(c\bar{b})_1(^3S_1)\rangle$;
 "3" for only $P$-wave states: $|(c\bar{b})_1(^1P_1)\rangle$, $|(c\bar{b})_1(^3P_{J=0,1,2})\rangle$,
$|(c\bar{b})_8(^1S_0)g\rangle$, $|(c\bar{b})_8(^3S_1)g\rangle$.}
{\item "counter ibcstate 2"\\
set which $B_c$ state to generate for the case of generating one of the eight states.
"1", "2" , ... , "8" for the eight states $|(c\bar{b})_1(^1S_0)\rangle$, $|(c\bar{b})_1(^3S_1)\rangle$,
$|(c\bar{b})_1(^1P_1)\rangle$, $|(c\bar{b})_1(^3P_{J=0,1,2})\rangle$, $|(c\bar{b})_8(^1S_0)g\rangle$ and
$|(c\bar{b})_8(^3S_1)g\rangle$, respectively. }
{\item "loggrade igrade 1"\\
whether to use the grade files, "1" for yes, "0" for no.}
\end{itemize}

\item Include the option file \emph{14900000.opts} into \emph{Gauss.opts}.

\item Run Gauss to generate $B_c$ events.

\en


\begin{thebibliography}{99}

\bibitem{BCVEGPY} Chao-Hsi Chang, Chafik Driouichi, Paula Eerola, Xing-Gang Wu, Comp. Phys. Comm. 159,
     192 (2004).\\
     Chao-Hsi Chang, Jian-Xiong Wang, Xing-Gang Wu, Comp. Phys. Comm. 174, 241 (2006).



\end{thebibliography}


\end{document}

