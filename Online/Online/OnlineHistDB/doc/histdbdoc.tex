\documentclass{lhcbnote}
\usepackage{amsmath}
\usepackage{amssymb}
\usepackage{epsfig}
\title{Histogram DB for Online Monitoring -- User's Manual}
\doctyp{}
\dociss{2}
\docrev{4}
\doccre{March 21, 2007}
\docmod{\today}
\docref{}
\author{G.~Graziani}

\begin{document}
\maketitle


\section{DB design}
The requirements and use cases of an Histogram Database for Online
Monitoring in the context of a common Histogramming Framework~\cite{genhisto}
have been defined in ~\cite{dbdesign}.

The present design of DB tables is shown in figure \ref{DBschema}


\begin{figure}[htb]
\centerline{\epsfig{figure=DBschema.eps,width=\textwidth}}
\caption{Scheme of DB tables. }
\label{DBschema}
\end{figure}

\subsection{Definition of Histograms}
Histograms are uniquely identified by the expression:\\
{\it Taskname}/{\it Algorithmname}/{\it HistogramName}\\
The histogram name can contain a subname:
{\it HistogramName} = {\it HistogramSetName}\_\${\it Subname} \\
Histograms differing only by {\it Subname} are part of the same
Histogram set. These should be histograms that have identical binning,
e.g. containing the same distribution for different channels of a detector.

For easier reference, an internal unique identifier is created for
each histogram, in the form\\
{\it HID} = {\it HSID}/{\it IHS}\\
where {\it HSID} is an integer number identifying the histogram set,
and {\it IHS} is a sequence number (starting from 1) to identify histograms in
the same set.

\subsubsection{Properties of TASK} 
\begin{description}
\item{TaskName} (string of max length 100) \\
unique task identifier
\item{RunOnPhysics, RunOnCalib, RunOnEmpty} (boolean)\\
specify for which type of data task is running 
\item{Subsys1, Subsys2, Subsys3} (string of length 10)\\
up to 3 subdetector/subsystem can be associated to task. 
\item{Reference}  (string of length 100)\\
link to the location of reference histograms for this task
\item{SaveFrequency} saving frequency for the corresponding saveset
\end{description} 

\subsubsection{Properties of HISTOGRAMSET}
\begin{description}
\item{HSID} (integer)
\item{NHS} (integer)\\
number of histograms in set
\item{Task} (valid TaskName) 
\item{Algorithm} (string of max length 100) 
\item{HistogramSetName} (string of max length 200) 
\item{Type} \\
'H1D', 'H2D' for normal histograms, 'P1D', 'P2D' for profile
histograms, 'CNT' for counters
\item{Nanalysis} (integer)\\
number of analysis to be performed on set
\item{Description} (string of max length 4000) 
\item{Documentation} (string of max length 200)\\
link to a more extensive documentation
\item{HSDisplay} (valid DOID)\\
identifier of display option set associated to Histogram set 
\end{description}

\subsubsection{Properties of HISTOGRAM}
\begin{description}
\item{HID} (string of max length 12)\\
{\it HSID}/{\it IHS}
\item{Subname} (string of max length 50)
\item{DIMServiceName} (string of max length 130)\\
Name of the DIM service that is currently publishing the histogram 
\item{IsAnalysisHist} (boolean)\\
true if histogram is produced at analysis level
\item{CreationTime} (timestamp)\\
recording the first time the histogram is seen
\item{ObsoletenessTime} (timestamp)\\
can be set by hand if histogram is not produced any more 
\item{Display} (valid DOID)
identifier of display option set associated to Histogram 
\end{description}

\subsection{Definition of Pages and Display Options}
Pages and Page Folders are uniquely identified by their name. Page
folders can have a hierarchical structure, for example:\\
{\it Folder1}\\
{\it Folder1/Folder2}\\
{\it Folder1/Folder2/Folder3}\\
Pages are associated to a list of valid histograms through the SHOWHISTO
table, containing the layout of each histogram on the page.

As shown in figure \ref{DBschema}, a set of display options can
be defined for:
\begin{itemize}
\item an histogram on a given page
\item an histogram
\item an histogram set
\end{itemize}
so that the most specific available set is used, but one can use the
same default for, say, the 2000 histograms of a certain set. 

\subsubsection{Properties of DISPLAYOPTIONS}\label{dispopts}
\begin{description}
\item{DOID} (integer)\\
unique identifier 
\item{LABEL\_X}  (string of max length 50)
\item{LABEL\_Y}  (string of max length 50)
\item{LABEL\_Z}  (string of max length 50)
\item{YMIN} (float)
\item{YMAX} (float)
\item{STATS} (int)
\item{FILLSTYLE} (int)
\item{FILLCOLOR} (int)
\item{LINESTYLE} (int)
\item{LINECOLOR} (int)
\item{LINEWIDTH} (int)
\item{DRAWOPTS}  (string of max length 50)
\end{description}
\subsubsection{Properties of PAGEFOLDER}
\begin{description}
\item{PageFolderName} (string of max length 30)\\
unique page folder identifier
\item{Parent} (string of max length 30)\\
Folder containing this folder (NULL for ``root'' folders)
\end{description}

\subsubsection{Properties of PAGE}
\begin{description}
\item{PageName} (string of max length 50) \\
unique page identifier
\item{Folder} (valid PageFolderName) 
\item{Nhisto} (integer) \\
number of histograms on page
\item{PageDoc} (string of max length 100)\\
short page description
\end{description}

\subsubsection{Properties of SHOWHISTO}
The unique identifier is the combination (Page,Histo,Instance)
\begin{description}
\item{Page} (valid PageName)
\item{Histo} (valid HID)
\item{Instance} (int)\\
sequence number (starting from 1) to distinguish different instances
of the same histogram on the same page
\item{Cx, Cy, Sx, Sy} (float numbers from 0 to 1)\\
coordinates of the histogram pad on the page: Cx and Cy define the
position of the top left corner, Sx and Sy the size, relatively to the
window size
\item{Sdisplay} (valid DOID)\\
identifier of display option set associated to this Histogram on this Page
\end{description}

\subsection{Definition of Automatic Analysis}
The ALGORITHM table contains the definition of the algorithms
available for analysis. They can be used to create new histograms at
analysis level (these will be called ``Analysis Histograms'' and are
defined by the HCREATOR table), or to perform automatic checks,
defined in the ANALYSIS table. Analyses are properties of an histogram
set, though their parameters can be specified for each histogram in
the ANASETTINGS table.

\subsubsection{Properties of ALGORITHM}
\begin{description}
\item{AlgorithmName} (string of max length 30)\\
unique algorithm identifier
\item{AlgType} \\
'HCREATOR' or 'CHECK'
\item{Ninput} (integer)\\
number of input histograms (for 'HCREATOR' algorithms)
\item{Npars} (integer)\\
number of parameters
\item{AlgPars} (array(any length) of string of max length 15)\\
parameter names
\item{HCTYPE}\\
for 'HCREATOR' algorithms, type of histogram generated by this algorithm 
\item{AlgDoc} (string of max length 1000)
documentation

\end{description}

\subsubsection{Properties of ANALYSIS}
\begin{description}
\item{AID} (integer)\\
unique analysis identifier (allowing to assign the same algorithm more
than once to the same histogram)
\item{HSET}  (valid HSID)
\item{Algorithm}  (valid AlgorithmName)
\end{description}

\subsubsection{Properties of ANASETTINGS}
\begin{description}
\item{AnaID} (valid AID)
\item{Histogram}  (valid HID)
\item{Mask} (boolean)\\
allow to mask the analysis for a single histogram
\item{Warnings, Alarms} (arrays(Npars) of floats)\\
2 sets of threshold levels
\end{description}

\subsubsection{Properties of HCREATOR}
when a HCREATOR entry is defined, the corresponding histogram is
created with Task='ANALYSIS' and Algorithm= the name of the analysis algorithm
\begin{description}
\item{HCID} (valid HID)
\item{Algorithm} (valid AlgorithmName)
\item{Sourceh} (arrays(8) of string of max length 12)\\
list of input histograms
\item{SourceHSet} (valid HSID)\\
input histogram set (if required by the algorithm)
\item{HCPARS} (arrays(Npars) of floats)\\
set of needed parameters 
\end{description}

\section{DB implementation}
A first prototype of the DB has been implemented under Oracle on the
CERN Oracle server and is available for tests.
	
In June 2007 the DB has migrated to the LHCb Oracle server at point8,
that can be seen from the cern network only.

The connection string is \\
\centerline{$<$account$>$/$<$password$>$@lbora01:1528/HISTOGRAMDB}
where $<$account$>$ is \\
{HIST\_READER}   (read--only, password=''welcome'') or \\
{HIST\_WRITER} 

The DB can be accessed through a C++ API or interactively through a
Web interface written in PHP. In order to minimize client load and
network traffic, and ease the maintenance of interface code, both
interfaces  are based on a set of common PL/SQL procedures that are
precompiled on the Oracle server.

\section{Web interface}

It is available for test at the address\\
{\it https://webafs3.cern.ch/ggrazian/lhcb/OnlineHistDB/index.php}

while the production address (visible only from the CERN network)
will be \\
\centerline{ \it http://lbweb01:8090/histogramdb}  

It is intended to be the most suitable tool to browse available
histograms, edit the display options and the automatic analysis, including
the definition of histograms to be produced at analysis level.

Presently, it is also possible to edit the viewer page configurations, though
a graphical editor in the presenter application will likely be the
most suitable tool for that task.

It is also suited for uploading {\bf reference histograms}. Users should
provide ROOT files containing all the reference histograms of a given
task, that could be simply a saveset files. \\
These can be uploaded through the ``Task Editor'' link at \\
{ \it http://lbweb01:8090/histogramdb} \\
A tool is provided to check the file content for missing or unknown histograms.
According to the convention of \cite{dbdesign}, files are saved as \\
$<$Reference\_root$>$/$<$task name$>$/$<$datatype$>$\_$<$startrun$>$.\\
where the fields $<$datatype$>$ and $<$startrun$>$ allow to define
different reference for different run condition and time range. 

\section{C++ Interface}

The interface is available as a link library that can be compiled from
the package \\
{\it Online/OnlineHistDB} \\
in the LHCb code repository.

The C++ API allows to perform practically any operation on the DB .
%but the deletion of objects, that is reserved to the DB administrator.

You can add entries to the DB through the methods beginning with {\it
declare}, that create the specified entry if not existing, or 
update its fields otherwise. If you use these methods, running the
same code twice is equivalent to run it once.


\subsection{OnlineHistDB class}
Each instantiation of this class opens a transaction with the DB
server. Its methods allow to define new DB entries (histograms,
tasks, subsystems, pages, algorithms) and query the DB content. \\
Histograms, tasks and pages can be manipulated using the corresponding
classes described in the next sections. For each DB entry, an object can be
instantiated through the {\it getHistogram},  {\it getTask} and {\it
getPage} methods. Such objects make sense only within the transaction
and should never be deleted by the user (they are destroyed by the
OnlineHistDB destructor). 

Changes are committed to the DB only by an explicit call to the {\it
commit} method.

By default, DB errors don't throw exceptions to the client code but
for severe inconsistencies (i.e. bugs). The default behaviour can be
changed (see section \ref{dbenvclass}). Normally, one can detect
errors by the method return values and choose if commit or not. \\
Histogram declarations are a special case since, 
in order to improve performance, they are stored in
a buffer, with default depth 1000, and actually sent to the DB
server only when the buffer is full or at commit time. In order to
check for errors in histogram declarations, one can use the {\it
sendHistBuffer} method to force buffer sending, and check its return
value before committing. Note that if an error occurs and you commit anyway, 
only all histograms declared before the error are sent to the DB, and
the buffer is emptied.


\begin{list}{$\bullet$}{}

\item    {\bf OnlineHistDB }(std::string passwd,\\\mbox{}~~~~~~~~~ 
		std::string user=OnlineHistDBEnv\_constants::ACCOUNT,\\\mbox{}~~~~~~~~~ 
		std::string db=OnlineHistDBEnv\_constants::DB);

 constructor


\item    bool {\bf commit}();


 commits all changes to the DB. Returns true if there are no errors.


\item    bool {\bf declareTask}(std::string Name,\\\mbox{}~~~~~~~~~ 
		   std::string SubDet1="NULL",\\\mbox{}~~~~~~~~~ 
		   std::string SubDet2="NULL",\\\mbox{}~~~~~~~~~ 
		   std::string SubDet3="NULL",\\\mbox{}~~~~~~~~~
		   bool RunsOnPhysics=false,\\\mbox{}~~~~~~~~~ 
		   bool RunsOnCalib=false,\\\mbox{}~~~~~~~~~ 
		   bool RunsOnEmpty=false,\\\mbox{}~~~~~~~~~
		   float SavingFrequency=0);

 declares a new task to the DB, or updates its configuration 


\item    OnlineHistTask* {\bf getTask}(std::string Name);


 get an OnlineHistTask object, holding informations of an existing task, that can be used to view/edit a task record


\item    OnlineHistogram* {\bf getHistogram}(std::string Identifier,\\\mbox{}~~~~~~~~~
				std::string FullPathPageName="\_NONE\_",\\\mbox{}~~~~~~~~~
				int Instance = 1);

 gets a pointer to an OnlineHistogram object that can be used to view/edit an histogram record. If FullPathPageName
 is specified, the default display options for the histogram are those associated to the page (if available).
 Uses cached histogram objects if available


\item    OnlineHistogram* {\bf getNewHistogram}(std::string Identifier,\\\mbox{}~~~~~~~~~
				   std::string FullPathPageName="\_NONE\_",\\\mbox{}~~~~~~~~~
				   int Instance = 1);

 same as getHistogram, but a new object is always created (no caching)


\item    virtual bool {\bf removeHistogram}(OnlineHistogram* h,\\\mbox{}~~~~~~~~~
			       bool RemoveWholeSet = false);

 removes an histogram, and optionally its full set. 
 ({\bf TEMPORARY METHOD TO BE REMOVED AT PRODUCTION STAGE})


\item    OnlineHistPage* {\bf getPage}(std::string Name);


 get an OnlineHistPage object, to create a new page or view/edit an existing one


\item    bool {\bf removePage}(OnlineHistPage* Page);


 removes completely the page, and all associated options (HANDLE WITH CARE!)


\item    bool {\bf declareSubSystem}(std::string SubSys);


 declares a  subsystem, returning true on success


\item    void {\bf declareHistByServiceName}(const std::string \&ServiceName);


 declares an Histogram by its DIM service name. In the LHCb
 DAQ, this is intended to be used only by the Experiment Control
 System to dynamically update the DB with the published histograms.
 Tasks not known to the DB are automatically created.
 If histogram already exists, just updates the current DIM service name 


\item    void {\bf declareHistogram}(std::string TaskName,\\\mbox{}~~~~~~~~~
			std::string AlgorithmName,\\\mbox{}~~~~~~~~~
			std::string HistogramName,\\\mbox{}~~~~~~~~~
			HistType Type);

 declares an Histogram to the DB by its attributes.  The enum HistType is defined in class OnlineHistDBEnv


\item    OnlineHistogram* {\bf declareAnalysisHistogram}(std::string Algorithm,\\\mbox{}~~~~~~~~~
					    std::string Name,\\\mbox{}~~~~~~~~~
					    std::vector$<$OnlineHistogram*$>$ \&Sources,\\\mbox{}~~~~~~~~~
					    std::vector$<$float$>$* Parameters = NULL);

 declares an histogram to be produced at analysis
 level using algorithm Algorithm. Name is the histogram name. Sources
 must contain the pointers to the input histograms. Parameters is a
 pointer to a parameter vector, optionally needed by the algorithm. If
 the algorithm requires an histogram set as input, use any histogram of the
 set. Returns the pointer to the new histogram object.


\item    bool {\bf declareCheckAlgorithm}(std::string Name,\\\mbox{}~~~~~~~~~ 
			     int Npars,\\\mbox{}~~~~~~~~~ 
			     std::vector$<$std::string$>$ *pars=NULL,\\\mbox{}~~~~~~~~~ 
			     std::string doc="NONE");

 declares to the DB an Analysis algorithm implemented in the Analysis
 library. Npars is the number of algorithm's parameters, pars should
 point to an array containing the parameter names, doc is a short
 description of the algorithm.


\item    bool {\bf declareCreatorAlgorithm}(std::string Name,\\\mbox{}~~~~~~~~~ 
			       int Ninput=0,\\\mbox{}~~~~~~~~~ 
			       HistType OutputType = H1D,\\\mbox{}~~~~~~~~~
			       int Npars=0,\\\mbox{}~~~~~~~~~ 
			       std::vector$<$std::string$>$ *pars=NULL,\\\mbox{}~~~~~~~~~
			       std::string doc="NONE");

 declares to the DB an available algorithm to produce histograms at
 analysis time. Ninput is the number of input histograms, Npars the
 number of optional parameters (pars containing their names), doc is a short
 description of the algorithm.


\item    int {\bf getAlgListID}() const ;

 gets the algorithm list version


\item    bool {\bf setAlgListID}(int algListID);


 sets the algorithm list version (works only for DB admin account)


\item    bool {\bf removePageFolder}(std::string Folder);


 removes Page Folder only if it doesn't have pages (useful for cleanup)


\item    int {\bf nHistograms}() ;

 total number of histograms in the DB


\item    int {\bf nPages}() ;

 total number of pages in the DB


\item    int {\bf nPageFolders}() ;

  total number of page folders in the DB


\item    int {\bf getAllHistograms}(std::vector$<$OnlineHistogram*$>$* list = NULL,\\\mbox{}~~~~~~~~~
		       std::vector$<$string$>$* ids = NULL,\\\mbox{}~~~~~~~~~
		       std::vector$<$string$>$* types = NULL);

 gets the full list of histograms. Returns the number of histograms found. Vectors with pointers
 to OnlineHistogram objects, histogram identifiers, histogram types can optionally created  by the user
 and filled if not null


\item    int {\bf getHistogramsWithAnalysis}(std::vector$<$OnlineHistogram*$>$* list = NULL,\\\mbox{}~~~~~~~~~
				std::vector$<$string$>$* ids = NULL,\\\mbox{}~~~~~~~~~
				std::vector$<$string$>$* types = NULL);

 gets the list of histograms on which some check analysis has to be performed 


\item    int {\bf getAnalysisHistograms}(std::vector$<$OnlineHistogram*$>$* list = NULL,\\\mbox{}~~~~~~~~~
			    std::vector$<$string$>$* ids = NULL,\\\mbox{}~~~~~~~~~
			    std::vector$<$string$>$* types = NULL);

 gets the list of histograms that have to be produced by analysis task


\item    int {\bf getHistogramsBySubsystem}(std::string SubSys,\\\mbox{}~~~~~~~~~
			       std::vector$<$OnlineHistogram*$>$* list = NULL,\\\mbox{}~~~~~~~~~
			       std::vector$<$string$>$* ids = NULL,\\\mbox{}~~~~~~~~~
			       std::vector$<$string$>$* types = NULL);

 gets the list of histograms related to subsystem SubSys


\item    int {\bf getHistogramsByTask}(std::string Task,\\\mbox{}~~~~~~~~~
			  std::vector$<$OnlineHistogram*$>$* list = NULL,\\\mbox{}~~~~~~~~~
			  std::vector$<$string$>$* ids = NULL,\\\mbox{}~~~~~~~~~
			  std::vector$<$string$>$* types = NULL);

 gets the list of histograms related to task Task


\item    int {\bf getHistogramsByPage}(std::string Page,\\\mbox{}~~~~~~~~~
			  std::vector$<$OnlineHistogram*$>$* list = NULL,\\\mbox{}~~~~~~~~~
			  std::vector$<$string$>$* ids = NULL,\\\mbox{}~~~~~~~~~
			  std::vector$<$string$>$* types = NULL);

 gets the list of histograms displayed on page Page


\item    int {\bf getHistogramsBySet}(std::string SetName,\\\mbox{}~~~~~~~~~
			 std::vector$<$OnlineHistogram*$>$* list = NULL,\\\mbox{}~~~~~~~~~
			 std::vector$<$string$>$* ids = NULL,\\\mbox{}~~~~~~~~~
			 std::vector$<$string$>$* types = NULL);

 gets the list of histograms in a Set


\item    int {\bf getPageFolderNames}(std::vector$<$string$>$\& list,\\\mbox{}~~~~~~~~~ std::string Parent="\_ALL\_");


 gets the list of page folders, Parent can be "/", a page folder name or "\_ALL\_" for all folders  


\item    int {\bf getPageNamesByFolder}(std::string Folder,\\\mbox{}~~~~~~~~~
			   std::vector$<$string$>$\& list);

 gets the list of pages in a folder


\item    int {\bf getSubsystems}(std::vector$<$string$>$\& list);


 gets the list of known subsystems


\item    int {\bf getTasks}(std::vector$<$string$>$\& list);


 gets the list of tasks


\item    int {\bf getAlgorithms}(std::vector$<$string$>$\& list,\\\mbox{}~~~~~~~~~std::string type="\_ALL\_");


 gets the list of algorithms, type can be "\_ALL\_", "CHECK", "HCREATOR"




\end{list}


\subsection{OnlineHistogram class}

OnlineHistogram objects are instantiated within an
OnlineHistDB object, i.e. a DB transaction, through the {\it
getHistogram} method.

\begin{list}{$\bullet$}{}
\item    inline std::string {\bf identifier}() const ;

 full histogram unique identifier Taskname/AlgorithmName/HistogramName


\item    inline std::string {\bf page}() const ;

 full path name of the page to which this histogram object is attached


\item    inline int {\bf instance}() const ;

 counter (starting from 1) to distinguish several instances of the same histogram on the same page


\item    inline std::string {\bf dimServiceName}() const  ;

 name of the DIM service that is currently publishing the histogram


\item    inline int {\bf hsid}() const ;

 internal histogram set ID


\item    inline int {\bf ihs}() const ;

 position of this histogram in set (starting from 1). Can be larger than nhs() if some histogram in the set has been deleted


\item    inline std::string {\bf hid}() const ;

 internal histogram ID (equivalent to hsid()/ihs())


\item    inline int {\bf nhs}() const ;

 number of histograms in set


\item    inline std::string {\bf hstype}() const ;

 histogram type (``H1D'', ``H2D'', ``P1D'',``P2D'' or ``CNT'' )


\item    inline OnlineHistDBEnv::HistType {\bf type}() const ;

 histogram type (``H1D'', ``H2D'', ``P1D'',``P2D'' or ``CNT'' )


\item    inline int {\bf dimension}() ;

 histogram dimension (1 or 2 for histograms, 0 for counters)


\item    inline std::string {\bf hname}() const ;

 histogram name


\item    inline std::string {\bf htitle}() const ;

 standard DB histogram title (not necessarly equal to the published ROOT title)


\item    inline std::string {\bf hsname}() const ;

 histogram set name


\item    inline std::string {\bf subname}() const ;

 subname


\item    inline std::string {\bf task}() const ;

 task name


\item    inline std::string {\bf algorithm}() const ;

 algorithm name


\item    std::string {\bf descr}() const ;

 short description of the histogram


\item    std::string {\bf doc}() const ;

 link to a more extensive documentation


\item    int {\bf creation}() const ;

 creation date, as a unix timestamp


\item    int {\bf obsoleteness}() const ;

 if the histogram is no more in use, returns the end--of--validity date
 as a unix timestamp, otherwise returns 0.


\item    bool {\bf setPage}(std::string FullPathPageName,\\\mbox{}~~~~~~~~~
	       int Instance=1);

 sets page on which histogram is displayed (reload display options if needed). Histogram has to be already
 been attached to the page through OnlineHistPage::declareHistogram()


\item    void {\bf unsetPage}();


 unsets page associated to histogram object


\item    bool {\bf setDimServiceName}(std::string DimServiceName);


 sets the DIM service name that is currently publishing the histogram. Returns true on success


\item    bool {\bf setDoc}(std::string doc);


 provide a short description to be optionally printed on the plot 


\item    bool {\bf setDescr}(std::string descr);


 provide a  description of the histogram content 


\item    void {\bf dump}();


 dumps histogram data


\item    void {\bf dumpDisplayOptions}();


 dumps histogram display options


\item    int {\bf nPageInstances}();


 number of instances of this histogram on any page


\item    int {\bf nThisPageInstances}();


 number of instances of this histogram on its current page


\item    OnlineHistTask* {\bf getTask}();


 gets associated task object


\item   typedef enum { NONE, SET, HIST, HISTPAGE } DisplayOptionMode;
\item    DisplayOptionMode {\bf domode}() const ;

 specifies if the display options in this object are: not defined,
 associated to the histogram set, associated to the histogram,
 associated to the histogram on  page {\it page()}


\item    bool {\bf getDisplayOption}(std::string ParameterName,\\\mbox{}~~~~~~~~~
			void* option);

 gets the current value of display option ParameterName, returning false if
 the option is unset.
 option must be a pointer to a variable allocated by the user, of type 
 std::string, int (signed) or float depending on the option type.
 The available option names and the corresponding types are listed 
 in section \ref{dispopts}.


\item    bool {\bf isSetDisplayOption}(std::string ParameterName);


 checks is display option is set


\item    virtual bool {\bf setDisplayOption}(std::string ParameterName,\\\mbox{}~~~~~~~~~
			void* option);

 sets a display option (in the current display mode). Change is sent to the DB only 
 after a call to saveDisplayOptions()


\item    virtual bool {\bf unsetDisplayOption}(std::string ParameterName);


 unsets a display option (in the current display mode)


\item    virtual bool {\bf changedDisplayOption}(std::string ParameterName,\\\mbox{}~~~~~~~~~ 
				    void* option);

 returns true if the current option value is different from *option


\item    virtual bool {\bf saveDisplayOptions}();


 saves current display options.  This is the recommended saving method, that
 chooses the most appropriate display option mode 
 (Histogram set, Histogram, or Histogram on Page) calling one of the next methods.


\item    virtual bool {\bf saveHistoSetDisplayOptions}();


 saves current display options for the whole histogram set


\item    virtual bool {\bf saveHistDisplayOptions}();


 saves current display options for the present histogram.


\item    virtual bool {\bf saveHistoPageDisplayOptions}(std::string Page = "\_DEFAULT\_",\\\mbox{}~~~~~~~~~
					   int Instance=-1);

 attaches current display options to instance Instance of the present histogram on page 
 Page (default values for Page, Instance are {\it page()},{\it instance()}  )


\item    bool {\bf initHistDisplayOptionsFromSet}(); 


 initializes display options associated to this histogram with the
 options defined for the histogram set (if available). Returns true on
 success.


\item    bool {\bf initHistoPageDisplayOptionsFromSet}(std::string FullPathPageName = "\_DEFAULT\_",\\\mbox{}~~~~~~~~~
					  int Instance=-1);

 initializes display options associated to this histogram on page
 FullPathPageName (default is {\it page()}) with the
 options defined for the histogram set (if available). Returns true on
 success.


\item    bool {\bf initHistoPageDisplayOptionsFromHist}(std::string FullPathPageName = "\_DEFAULT\_",\\\mbox{}~~~~~~~~~
					   int Instance=-1);

 initializes display options associated to this histogram on page
 FullPathPageName (default is {\it page()}) with the
 options defined for the histogram (if available). Returns true on
 success.


\item    bool {\bf getCreationDirections}(std::string \&Algorithm,\\\mbox{}~~~~~~~~~
			     std::vector$<$std::string$>$ \&source\_list,\\\mbox{}~~~~~~~~~
			     std::vector$<$float$>$ \&parameters);

 for analysis histogram, get the directions for creating histogram


\item    int {\bf nanalysis}() const ;

 number of analysis to be performed on the histogram set


\item    void {\bf getAnalyses}(std::vector$<$int$>$\& anaIDs,\\\mbox{}~~~~~~~~~
		   std::vector$<$std::string$>$\& anaAlgs) ;

 get analysy description as vectors of length {\it  nanalysis}() containing 
 the analysis internal IDs and the analysis algorithm names


\item    bool {\bf isAnaHist}() const ;

 true if the histogram is produced at analysis level


\item    int {\bf declareAnalysis}(std::string Algorithm,\\\mbox{}~~~~~~~~~ 
                      std::vector$<$float$>$* warningThr=NULL,\\\mbox{}~~~~~~~~~ 
                      std::vector$<$float$>$* alarmThr=NULL,\\\mbox{}~~~~~~~~~ 
                      std::vector$<$float$>$* inputs=NULL,\\\mbox{}~~~~~~~~~ 
                      int instance=1);

 declare an analysis to be performed on the histogram set. If the algorithm
 requires some parameters, the input parameters and the warning and alarm 
 thresholds for output parameters must be
 specified as vectors of floats and will be set for all histograms in
 set (then, you can specify values for single histograms with the {\it
 setAnalysis} method. 
 You can create more than one analysis
 with the same algorithm by using instance $>$ 1. If the analysis
 identified by Algorithm and instance already exists, parameters are
 updated. Returns the internal analysis ID.


\item    bool {\bf setAnalysis}(int AnaID,\\\mbox{}~~~~~~~~~ 
                   std::vector$<$float$>$* warningThr=NULL,\\\mbox{}~~~~~~~~~ 
                   std::vector$<$float$>$* alarmThr=NULL,\\\mbox{}~~~~~~~~~ 
                   std::vector$<$float$>$* inputs=NULL);

 updates parameters for analysis with ID AnaID (for this histogram only). Returns true on success


\item    bool {\bf getAnaSettings}(int AnaID,\\\mbox{}~~~~~~~~~
                      std::vector$<$float$>$* warn,\\\mbox{}~~~~~~~~~ 
                      std::vector$<$float$>$* alarm,\\\mbox{}~~~~~~~~~
                      std::vector$<$float$>$* inputs,\\\mbox{}~~~~~~~~~
                      bool \&mask);

 gets parameters for analysis with ID AnaID. Returns true on success


\item    bool {\bf maskAnalysis}(int AnaID,\\\mbox{}~~~~~~~~~
                    bool Mask=true);

 masks analysis with ID AnaID. Use Mask=false to unmask. Returns true on success



\end{list}

\subsection{OnlineRootHist class}

This class inherits from OnlineHistogram and ROOT's TH1 classes. It
can be used as an interface between the histogram DB and ROOT, in
order to load or save display parameters trasparently.

\begin{list}{$\bullet$}{}
\item    {\bf OnlineRootHist}(std::string Identifier,\\\mbox{}~~~~~~~~~
		 OnlineHistDB *Session = NULL,\\\mbox{}~~~~~~~~~
		 std::string Page="\_NONE\_",\\\mbox{}~~~~~~~~~
		 int Instance=1);

 constructor using identifier and possibly DB session


\item    {\bf OnlineRootHist}(OnlineHistogram*  oh);


 constructor from existing OnlineHistogram object


\item    std::string {\bf identifier}()  ;

 histogram identifier


\item    OnlineHistogram* {\bf dbHist}() ;

 corresponding OnlineHistogram object


\item    TH1* {\bf rootHist}() ;

 corresponding ROOT TH1 object


\item    virtual bool {\bf setdbHist}(OnlineHistogram*  oh);


 link to an existing OnlineHistogram object, returns true on success


\item    bool {\bf setrootHist}(TH1*  rh);


 link to an existing ROOT TH1 object, returns true on success


\item    bool {\bf connected}() ;

 true if object knows an existing DB session


\item    OnlineHistDB* {\bf dbSession}() ;

 returns link to DB session


\item    bool {\bf connectToDB}(OnlineHistDB* Session,\\\mbox{}~~~~~~~~~
		   std::string Page="\_NONE\_",\\\mbox{}~~~~~~~~~
		   int Instance=1);

 connect to DB session, returns true if the corresponding DB histogram entry is found


\item    void {\bf setTH1FromDB}();


 updates ROOT TH1 display properties from Histogram DB (via OnlineHistogram object) 
 (normally called when connecting)


\item    void {\bf setDrawOptionsFromDB}(TPad* \&Pad);


 updates current drawing options from Histogram DB (via OnlineHistogram object)


\item    bool {\bf saveTH1ToDB}(TPad* Pad = NULL);


 saves current ROOT display options to OnlineHistogram object and to Histogram DB


\item    virtual bool {\bf setDisplayOption}(std::string ParameterName,\\\mbox{}~~~~~~~~~ 
				void* value);

 sets display option (see OnlineHistogram method). Change is sent to the DB only 
 after a call to saveTH1ToDB()


\item    bool {\bf setReference}(TH1 *ref,\\\mbox{}~~~~~~~~~
		    int startrun = 1,\\\mbox{}~~~~~~~~~
		    std::string DataType = "default");

 save provided histogram as a reference in the standard file, with optional run period and data type


\item    TH1* {\bf getReference}(int startrun = 1,\\\mbox{}~~~~~~~~~
		    std::string DataType = "default");

 get reference histogram if available


\item    virtual void {\bf Draw}(TPad* \&Pad);


 calls TH1 Draw method, calls setDrawOptions()


\item    void {\bf normalizeReference}();


 normalize reference (if existing and requested) to current plot



\end{list}


\subsection{OnlineHistTask class}

OnlineHistTask objects are instantiated within an
OnlineHistDB object, i.e. a DB transaction, through the {\it getTask} method.

\begin{list}{$\bullet$}{}
\item    std::string {\bf name}() ;

 task name


\item    int {\bf ndet}() ;

 number of associated subdetector/subsystems (up to 3)


\item    std::string {\bf det}(int i) ;

 name of associated subdetector/subsystems (-1 $<$ i $<$ ndet() )


\item    bool {\bf runsOnPhysics}() ;

 true if task is configured to run for physics events


\item    bool {\bf runsOnCalib}() ;

 true if task is configured to run for calibration events


\item    bool {\bf runsOnEmpty}() ;

 true if task is configured to run for empty events


\item    float {\bf savingFrequency}() ;

 task saving frequency


\item    std::string {\bf reference}() ;

 location of latest reference file


\item    bool {\bf setSubDetectors}(std::string SubDet1="NULL",\\\mbox{}~~~~~~~~~ 
			std::string SubDet2="NULL",\\\mbox{}~~~~~~~~~ 
			std::string SubDet3="NULL");

 sets the associated subdetector/subsystems. "NULL" unsets the value 


\item    bool {\bf addSubDetector}(std::string SubDet);


 adds an associated subdetector/subsystems, returning true on success


\item    bool {\bf setRunConfig}(bool RunsOnPhysics,\\\mbox{}~~~~~~~~~ 
		    bool RunsOnCalib,\\\mbox{}~~~~~~~~~ 
		    bool RunsOnEmpty) ;

 sets run configuration bits


\item    bool {\bf setSavingFrequency}(float SavingFrequency) ;

 sets task saving frequency


\item    bool {\bf setReference}(std::string Reference) ;

 sets the location of latest reference file



\end{list}


\subsection{OnlineHistPage class}

OnlineHistPage objects are instantiated within an
OnlineHistDB object, i.e. a DB transaction, through the {\it
getPage} method.

\begin{list}{$\bullet$}{}
\item    int {\bf nh}() const ;

 number of histograms on page


\item    const std::string\& {\bf name}() const ;

 page name (with full path)


\item    const std::string\& {\bf folder}() const ;

 page folder name


\item    const std::string\& {\bf doc}() const ;

 short page description


\item    const std::string\& {\bf patternFile}() const ;

 file name containing optional ROOT background pattern 


\item    const bool {\bf syncWithDB}() const ;

 check if the page object is in sync with the DB


\item    bool {\bf setDoc}(std::string \&Doc) ;

 set short page description


\item    bool {\bf setPatternFile}(std::string \&Pattern) ;

 set file name containing optional ROOT background pattern 


\item    OnlineHistogram* {\bf declareHistogram}(OnlineHistogram* h,\\\mbox{}~~~~~~~~~
                                    double Xmin,\\\mbox{}~~~~~~~~~
                                    double Ymin,\\\mbox{}~~~~~~~~~
                                    double Xmax,\\\mbox{}~~~~~~~~~
                                    double Ymax,\\\mbox{}~~~~~~~~~
                                    unsigned int instance=1);

 adds or updates an histogram on the page. Use instance $>$ 1 to use the
 same histogram more than once. Returns the object attached to page (the input
 one, or a new copy if a new instance needs to be created), or NULL in case of failure.
 Positions are relative to the Pad size (i.e. should be numbers from 0 to 1)


\item    OnlineHistogram* {\bf declareOverlapHistogram}(OnlineHistogram* h,\\\mbox{}~~~~~~~~~
                                           OnlineHistogram* onTopOf,\\\mbox{}~~~~~~~~~
                                           unsigned int instance=1,\\\mbox{}~~~~~~~~~
                                           int onTopOfInst=1,\\\mbox{}~~~~~~~~~
                                           int OverlapIndex = 1);

 adds or updates an histogram that has to be overdrawn on an existing histogram
 in case of overlap of several histograms, use OverlapIndex to specify the overlap order


\item    OnlineHistogram* {\bf addHistogram}(OnlineHistogram* h,\\\mbox{}~~~~~~~~~
                                double Xmin,\\\mbox{}~~~~~~~~~
                                double Ymin,\\\mbox{}~~~~~~~~~
                                double Xmax,\\\mbox{}~~~~~~~~~
                                double Ymax);

 like declareHistogram, but a new instance is always created


\item    OnlineHistogram* {\bf addOverlapHistogram}(OnlineHistogram* h,\\\mbox{}~~~~~~~~~
                                       OnlineHistogram* onTopOf=NULL,\\\mbox{}~~~~~~~~~
                                       int onTopOfInst=1,\\\mbox{}~~~~~~~~~
                                       int OverlapIndex = 1);

 like declareOvlapHistogram, but a new instance is always created


\item    bool {\bf removeHistogram}(OnlineHistogram* h,\\\mbox{}~~~~~~~~~
                       unsigned int instance=1);

 removes histogram from page, or one of its instances, returning true on success


\item    bool {\bf removeAllHistograms}();


 clean up the page removing all histograms


\item    bool {\bf remove}();


 remove the page


\item    void {\bf rename}(std::string NewName);


 changes Page name 


\item    void {\bf getHistogramList}(std::vector$<$OnlineHistoOnPage*$>$ *hlist) ;

 fills the hlist vector with pointers to the histograms attached to this page


\item    bool {\bf save}();


 saves the current page layout to the DB


\item    void {\bf dump}();


 dumps the current page layout 



\end{list}


\subsection{OnlineHistDBEnv class}\label{dbenvclass}
All previous classes derivate from this one, that has a few public
members:
\begin{list}{$\bullet$}{}
\item   typedef enum { H1D=0, H2D, P1D, P2D, CNT, SAM, TRE} HistType;
\item    inline int {\bf debug}() const ;

 get verbosity level (0 for no debug messages, up to 3)


\item    void {\bf setDebug}(int DebugLevel) ;

 set verbosity level


\item    int {\bf excLevel}() const ;

 exception level: 0 means never throw exc. to client code, 1 means only
 severe errors (default value), 2 means throw all exceptions.
 The default value is 1, meaning that exceptions are thrown only in
 case of severe DB inconsistency. All other errors, e.g. syntax errors,
 can be checked from the method return values and from the warning
 messages on the standard output.
 Exceptions can be catched using {\it catch(std::string message)}


\item    void {\bf setExcLevel}(int ExceptionLevel) ;




\item    std::string {\bf PagenameSyntax}(std::string \&fullname,\\\mbox{}~~~~~~~~~ std::string \&folder);


 check the syntax of the page full name, returning the correct syntax and the folder name 


\item    int {\bf getAlgorithmNpar}(std::string\& AlgName,\\\mbox{}~~~~~~~~~
		       int* Ninput = NULL);

 gets the number of parameters, and optionally the number of input histograms, needed by algorithm AlgName.


\item    std::string {\bf getAlgParName}(std::string\& AlgName,\\\mbox{}~~~~~~~~~
			    int Ipar);

 gets the name of parameter Ipar (starting from 1) of algorithm AlgName


\item    inline std::string\& {\bf refRoot}() ;

 get reference histograms root directory 


\item    inline std::string\& {\bf savesetsRoot}() ;

 get saveset root directory 



\end{list}

Note that the debug and exception levels defined for the OnlineHistDB object are
propagated to all new objects created through the {\it getPage} and
{\it getHistogram} methods.



\section{Code Example}

\begin{verbatim}
#include <OnlineHistDB/OnlineHistDB.h>
int main ()
{
\end{verbatim}
Open DB transaction:
\begin{verbatim}
 OnlineHistDB *HistDB = new OnlineHistDB(PASSWORD,
				OnlineHistDBEnv_constants::ACCOUNT,
			 	OnlineHistDBEnv_constants::DB);
 bool ok=true;
\end{verbatim}
Declare the features of your task 
\begin{verbatim}
 ok &= HistDB->declareTask("EXAMPLE","MUON","GAS","",true,true,false);
 OnlineHistTask* mytask = HistDB->getTask("EXAMPLE");
 if (mytask)
   mytask->setSavingFrequency(3.5);
\end{verbatim}
Declare some histograms (this is not normally needed for histograms
handled by the Online system)
\begin{verbatim}
 if (ok) {
   string ServiceName="H1D/nodeMF001_EXAMPLE_01/SafetyCheck/Trips";
   HistDB->declareHistByServiceName(ServiceName);
   ServiceName="H1D/nodeMF001_EXAMPLE_01/SafetyCheck/Trips_after_use_of_CRack";
   HistDB->declareHistByServiceName(ServiceName);
   ServiceName="H2D/nodeMF001_EXAMPLE_01/OccupancyMap/Hit_Map_$Region_M1R1";
   HistDB->declareHistByServiceName(ServiceName); 
   ServiceName="H2D/nodeMF001_EXAMPLE_01/OccupancyMap/Hit_Map_$Region_M1R2";
   HistDB->declareHistByServiceName(ServiceName);
   ServiceName="H2D/nodeMF001_EXAMPLE_01/OccupancyMap/Hit_Map_$Region_M3R1";
   HistDB->declareHistByServiceName(ServiceName);
   
   // alternative declaration not using DIM Service Name
   HistDB->declareHistogram("EXAMPLE","Timing","Coincidences",OnlineHistDBEnv::H1D);
   HistDB->declareHistogram("EXAMPLE","Timing","Time_of_flight",OnlineHistDBEnv::H1D);
   // DIM Service Name can be specified afterwards
   OnlineHistogram* thisH = HistDB->getHistogram("EXAMPLE/Timing/Time_of_flight");
    if(thisH)
     thisH->setDimServiceName("H1D/nodeA01_Adder_01/EXAMPLE/Timing/Time_of_flight");
    
   ok &= HistDB->sendHistBuffer(); // needed to send histogram buffer to DB
 }
\end{verbatim}
Now declare an histogram to be produced at analysis level by some
algorithm, and an automatic check to be performed on it (the
declaration of algorithms should be normally done by the developers of analysis
library)  
\begin{verbatim}
 HistDB->declareCreatorAlgorithm("Subtraction",2,OnlineHistDBEnv::H1D,0,NULL,
				 "bin-by-bin subtraction");

 OnlineHistogram* s1=HistDB->getHistogram("EXAMPLE/SafetyCheck/Trips");
 OnlineHistogram* s2=HistDB->getHistogram("EXAMPLE/SafetyCheck/Trips_after_use_of_CRack");
 OnlineHistogram* htrips=0;
 std::vector<OnlineHistogram*> sources;
 if(s1 && s2) {
   sources.push_back(s1);
   sources.push_back(s2);
   htrips=HistDB->declareAnalysisHistogram("Subtraction",
					   "Trips_due_to_CRack",
					   sources);
 }

 std::string mypar[1]={"Max"};
 bool algok=HistDB->declareCheckAlgorithm("CheckMax",1,mypar,
			       "Checks all bins to be smaller than Max");
 
 if (htrips && algok) {
   std::vector<float> warn(1,100.);
   std::vector<float> alarm(1,500.);
   htrips->declareAnalysis("CheckMax", &warn, &alarm);
 } 
\end{verbatim}
Now create a page and edit the display options of its
histograms. 
\begin{verbatim}
 OnlineHistogram* h1=HistDB->getHistogram
   ("EXAMPLE/OccupancyMap/Hit_Map_$Region_M1R1");
 OnlineHistogram* h2=HistDB->getHistogram
   ("EXAMPLE/OccupancyMap/Hit_Map_$Region_M1R2");
 OnlineHistogram* h3=HistDB->getHistogram
   ("EXAMPLE/OccupancyMap/Hit_Map_$Region_M3R1");

 if (h1 && h2 && h3) {
   std::string hcpar[2]={"w1","w2"};
   algok = HistDB->declareCreatorAlgorithm("Weighted mean",
					   2,
					   OnlineHistDBEnv::H1D,
					   2,
					   hcpar,
				"weighted mean of two histograms with weights w1 and w2");
   sources.clear();
   sources.push_back(h1);
   sources.push_back(h2);
   std::vector<float> weights(2,1.);
   weights[1]=0.5;
   if(algok) HistDB->declareAnalysisHistogram("Weighted mean",
					      "Silly plot",
					      sources,
					      &weights);

   OnlineHistPage* pg=HistDB->getPage("My Example Page","Examples/My examples");
   if(pg) {
     pg->declareHistogram(h1,0. ,0. ,0.5,0.5);
     pg->declareHistogram(h2,0. ,0.5,0.5,0.5);
     pg->declareHistogram(h3,0.5,0.5,0.5,0.4);
   
     int lc=2, fs=7, fc=3;
     float ymax=20000.;
     h1->setDisplayOption("LINECOLOR",(void*) &lc);
     h1->setDisplayOption("FILLSTYLE",(void*) &fs);
     h1->setDisplayOption("FILLCOLOR",(void*) &fc); 
     h1->setDisplayOption("YMAX",(void*) &ymax); 

     h1->dump();

     // second instance of h1
     OnlineHistogram* newh = pg->declareHistogram(h1,0.5,0. ,0.5,0.4,2);
     ymax=200000.;
     newh->setDisplayOption("YMAX",(void*) &ymax); 
     newh->unsetDisplayOption("LINECOLOR");

     newh->dump();
   }
 }
\end{verbatim}

Now display the list of pages with their histograms
\begin{verbatim}
 std::vector<string> folders;
 std::vector<string> pages;
 std::vector<OnlineHistogram*> histos; 
 int nfold=HistDB->getPageFolderNames(folders);
 int i,j,k;

 for (i=0;i<nfold;i++) {
   cout << "Folder " << folders[i] <<endl;
   pages.clear();
   int np=HistDB->getPageNamesByFolder(folders[i],pages);
   for (j=0;j<np;j++) {
     cout << "     Page " << pages[j] <<endl;
     histos.clear();
     int nh=HistDB->getHistogramsByPage(pages[j],histos);
     for (k=0;k<nh;k++) {
       cout << "           Histogram " << histos[k]->name() <<endl;
     }    
   }
 }
\end{verbatim}
and the lists of subsystems, tasks, and algorithms:
\begin{verbatim}
 std::vector<string> mylist;
 cout << "----------------------------------------"<<endl;
 int nss=HistDB->getSubsystems(mylist);
 for (i=0;i<nss;i++) {
   cout << "Subsys "<<mylist[i]<<endl;
 }
 mylist.clear();
 cout << "----------------------------------------"<<endl;
 nss=HistDB->getTasks(mylist);
 for (i=0;i<nss;i++) {
   cout << "Task "<<mylist[i]<<endl;
 }
 mylist.clear();
 cout << "----------------------------------------"<<endl;
 nss=HistDB->getAlgorithms(mylist);
 for (i=0;i<nss;i++) {
   cout << "Algorithm "<<mylist[i]<<endl;
 }
\end{verbatim}

finally commit changes to the DB if there were no errors
\begin{verbatim}
 ok &= HistDB->sendHistBuffer();
 if (ok) 
   HistDB->commit();
 else 
   cout << "commit aborted because of previous errors" <<endl;

 HistDB->setDebug(3); // close transaction verbosely
 delete HistDB;
}
\end{verbatim}


\begin{thebibliography}{99}
\bibitem{genhisto}
LHCb Commissioning Group, ``Histogramming Framework'', EDMS 748834
\bibitem{dbdesign}
``Histogram DB and Analysys Tools for Online Monitoring'', EDMS 774740

\end{thebibliography}


\end{document}
