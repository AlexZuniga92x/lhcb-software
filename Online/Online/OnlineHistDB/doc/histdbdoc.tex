\documentclass{lhcbnote}
\usepackage{amsmath}
\usepackage{amssymb}
\usepackage{epsfig}
\title{Histogram DB for Online Monitoring -- User's Manual}
\doctyp{}
\dociss{1}
\docrev{1}
\doccre{January 12, 2007}
\docmod{\today}
\docref{}
\author{G.~Graziani}

\begin{document}
\maketitle

%\begin{abstract}
%\end{abstract}

%\begin{status}
%\entry{1}{0}{January 16, 2007}{First release}
%\end{status}


\section{DB design}
The requirements and use cases of an Histogram Database for Online
Monitoring in the context of a common Histogramming Framework~\cite{genhisto}
have been defined in ~\cite{dbdesign}.

The present design of DB tables is shown in figure \ref{DBschema}


\begin{figure}[htb]
\centerline{\epsfig{figure=DBschema.eps,width=\textwidth}}
\caption{Scheme of DB tables. }
\label{DBschema}
\end{figure}

\subsection{Definition of Histograms}
Histograms are uniquely identified by their name:\\
{\it Taskname}/{\it Algorithmname}/{\it HistogramTitle}\\
The histogram title can have a subtitle:
{\it HistogramTitle} = {\it HistogramSetTitle}\_\$\$\_{\it Subtitle}\\
Histograms differing only by {\it Subtitle} are part of the same
Histogram set. These should be histograms that have identical binning,
e.g. containing the same distribution for different channels of a detector.

For easier reference, an internal unique identifier is created for
each histogram, in the form\\
{\it HID} = {\it HSID}/{\it IHS}\\
where {\it HSID} is an integer number identifying the histogram set,
and {\it IHS} is a sequence number (starting from 1) to identify histograms in
the same set.

\subsubsection{Properties of TASK}
\begin{description}
\item{TaskName} (string of max length 100) \\
unique task identifier
\item{RunOnPhysics, RunOnCalib, RunOnEmpty} (boolean)\\
specify for which type of data task is running 
\item{Subsys1, Subsys2, Subsys3} (string of length 10)\\
up to 3 subdetector/subsystem can be associated to task. 
\item{Reference}  (string of length 100)\\
link to the location of reference histograms for this task
\end{description}

\subsubsection{Properties of HISTOGRAMSET}
\begin{description}
\item{HSID} (integer)
\item{NHS} (integer)\\
number of histograms in set
\item{Task} (valid TaskName) 
\item{Algorithm} (string of max length 100) 
\item{HistogramSetTitle} (string of max length 200) 
\item{Type} \\
'1D', '2D', or 'PROF' for profile histogram
\item{Nanalysis} (integer)\\
number of analysis to be performed on set
\item{Description} (string of max length 4000) 
\item{Documentation} (string of max length 200)\\
link to a more extensive documentation
\item{HSDisplay} (valid DOID)\\
identifier of display option set associated to Histogram set 
\end{description}

\subsubsection{Properties of HISTOGRAM}
\begin{description}
\item{HID} (string of max length 12)\\
{\it HSID}/{\it IHS}
\item{Subtitle} (string of max length 50)
\item{DIMServiceName} (string of max length 500)\\
Name of the DIM service that is currently publishing the histogram 
\item{IsAnalysisHist} (boolean)\\
true if histogram is produced at analysis level
\item{CreationTime} (timestamp)\\
recording the first time the histogram is seen
\item{ObsoletenessTime} (timestamp)\\
can be set by hand if histogram is not produced any more 
\item{Display} (valid DOID)
identifier of display option set associated to Histogram 
\end{description}

\subsection{Definition of Pages and Display Options}
Pages and Page Folders are uniquely identified by their name. Pages
are associated to a list of valid histograms through the SHOWHISTO
table, containing the layout of each histogram on the page.

 As shown in figure \ref{DBschema}, a set of display options can
be defined for:
\begin{itemize}
\item an histogram on a given page
\item an histogram
\item an histogram set
\end{itemize}
so that the most specific available set is used, but one can use the
same default for, say, the 2000 histograms of a certain set. 

\subsubsection{Properties of DISPLAYOPTIONS}\label{dispopts}
\begin{description}
\item{DOID} (integer)\\
unique identifier 
\item{LABEL\_X}  (string of max length 50)
\item{LABEL\_Y}  (string of max length 50)
\item{LABEL\_Z}  (string of max length 50)
\item{YMIN} (float)
\item{YMAX} (float)
\item{STATS} (int)
\item{FILLSTYLE} (int)
\item{FILLCOLOR} (int)
\item{LINESTYLE} (int)
\item{LINECOLOR} (int)
\item{LINEWIDTH} (int)
\item{DRAWOPTS}  (string of max length 50)
\end{description}
\subsubsection{Properties of PAGEFOLDER}
\begin{description}
\item{PageFolderName} (string of max length 30)\\
unique page folder identifier
\end{description}

\subsubsection{Properties of PAGE}
\begin{description}
\item{PageName} (string of max length 50) \\
unique page identifier
\item{Folder} (valid PageFolderName) 
\item{Nhisto} (integer) \\
number of histograms on page
\item{PageDoc} (string of max length 100)\\
short page description
\end{description}

\subsubsection{Properties of SHOWHISTO}
\begin{description}
\item{Page} (valid PageName)
\item{Histo} (valid HID)
\item{Cx, Cy, Sx, Sy} (float numbers from 0 to 1)\\
coordinates of the histogram pad on the page: Cx and Cy define the
position of the top left corner, Sx and Sy the size, relatively to the
window size
\item{Sdisplay} (valid DOID)\\
identifier of display option set associated to this Histogram on this Page
\end{description}

\subsection{Definition of Automatic Analysis}
The ALGORITHM table contains the definition of the algorithms
available for analysis. They can be used to create new histograms at
analysis level (these will be called ``Analysis Histograms'' and are
defined by the HCREATOR table), or to perform automatic checks,
defined in the ANALYSIS table. Analyses are properties of an histogram
set, though their parameters can be specified for each histogram in
the ANASETTINGS table.

\subsubsection{Properties of ALGORITHM}
\begin{description}
\item{AlgorithmName} (string of max length 30)\\
unique algorithm identifier
\item{AlgType} \\
'HCREATOR' or 'CHECK'
\item{Ninput} (integer)\\
number of input histograms (for 'HCREATOR' algorithms)
\item{Npars} (integer)\\
number of parameters
\item{AlgPars} (array(any length) of string of max length 15)\\
parameter names
\item{AlgDoc} (string of max length 1000)
documentation
\end{description}

\subsubsection{Properties of ANALYSIS}
\begin{description}
\item{AID} (integer)\\
unique analysis identifier (allowing to assign the same algorithm more
than once to the same histogram)
\item{HSET}  (valid HSID)
\item{Algorithm}  (valid AlgorithmName)
\end{description}

\subsubsection{Properties of ANASETTINGS}
\begin{description}
\item{AnaID} (valid AID)
\item{Histogram}  (valid HID)
\item{Mask} (boolean)\\
allow to mask the analysis for a single histogram
\item{Warnings, Alarms} (arrays(Npars) of floats)\\
2 sets of threshold levels
\end{description}

\subsubsection{Properties of HCREATOR}
when a HCREATOR entry is defined, the corresponding histogram is
created with Task='ANALYSIS' and Algorithm= the name of the analysis algorithm
\begin{description}
\item{HCID} (valid HID)
\item{Algorithm} (valid AlgorithmName)
\item{Sourceh} (arrays(8) of string of max length 12)\\
list of input histograms
\item{SourceHSet} (valid HSID)\\
input histogram set (if required by the algorithm)
\end{description}

\section{DB implementation}
A first prototype of the DB has been implemented under Oracle on the
CERN Oracle server and is available for tests.

The DB can be accessed through a C++ API or interactively through a
Web interface written in PHP. In order to minimize client load and
network traffic, and ease the maintenance of interface code, both
interfaces  are based on a set of common PL/SQL procedures that are
precompiled on the Oracle server.

\section{Web interface}

It is available for test at the address\\
{\it https://webafs3.cern.ch/ggrazian/lhcb/OnlineHistDB/index.php}

It is intended to be the most suitable tool to browse available
histograms, edit the display options and the automatic analysis, including
the definition of histograms to be produced at analysis level.

Presently, it is also possible to edit the viewer page configurations, though
a graphical editor in the presenter application will likely be the
most suitable tool for that task.

\section{C++ Interface}

The interface is available as a link library that can be compiled from
the package \\
{\it Online/OnlineHistDB} \\
in the LHCb code repository.

The C++ API allows to perform any operation on the DB but the deletion
of objects, that is reserved to the DB administrator.

You can add entries to the DB through the methods beginning with {\it
declare}, that create the specified entry if not existing, or 
update its fields otherwise. Thus, running the same code twice is
equivalent to run it once.


\subsection{OnlineHistDB class}
Each instantiation of this class opens a transaction with the DB
server. Its methods allow the creation of histograms and pages, and
the definitions of tasks, subsystems and algorithms. Histograms and
pages can be edited through the pointers to OnlineHistogram and
OnlineHistPage objects returned by the {\it getHistogram} and {\it
getPage} methods. Such objects make sense only within the transaction
and should never be deleted by the user (they are destroyed by the
OnlineHistDB destructor). 

Changes are committed to the DB only by an explicit call to the {\it
commit} method.

\begin{list}{$\bullet$}{}
\item {\bf HistDB}(std::string passwd, 
		std::string user="lhcb\_mon\_giacomo", 
		std::string db="devdb10");

constructor

\item void {\bf commit}();

commits all changes to the DB


\item void {\bf declareTask}(std::string Name, 
		   std::string SubDet1="NULL", 
		   std::string SubDet2="NULL", 
		   std::string SubDet3="NULL",
		   bool RunsOnPhysics=false, 
		   bool RunsOnCalib=false,\\ 
		   bool RunsOnEmpty=false);

creates or updates a Task definition. Tasks can be associated to up to
three subdetectors/subsystems. 


\item   void {\bf declareSubSystem}(std::string SubSys);

declares a  subsystem

\item   void {\bf declareHistByServiceName}(const std::string \&ServiceName);

declares an Histogram by its DIM service name. In the LHCb
DAQ, this is intended to be used only by the Experiment Control
System to dynamically update the DB with the published histograms.\\
Tasks not known to the DB are automatically created.\\
if histogram already exists, just updates the current DIM service name 

\item void {\bf setHistogramBufferDepth}(int N);

when creating histograms with the {\it declareHistByServiceName}
method, the histogram list is actually send to the DB server every N
histograms (or at commit) in order to optimize performance. The
default buffer depth (recommended value) is 1000.  

\item OnlineHistogram* {\bf declareAnalysisHistogram}( std::string 
Algorithm, std::string Title, std::vector $<$OnlineHistogram*$>$ \&Sources);

declares an histogram to be produced at analysis
level using algorithm Algorithm. Title is the histogram title. Sources
must contain the pointers to the input histograms. If the algorithm
requires an histogram set as input, use any histogram of the
set. Returns the pointer to the new histogram object.
 
\item   void {\bf declareCheckAlgorithm}(std::string Name, 
			     int Npars, 
			     std::string* pars=NULL, 
			     std::string doc="NONE");

declares to the DB an Analysis algorithm implemented in the Analysis
library. Npars is the number of algorithm's parameters, pars should
point to an array containing the parameter names, doc is a short
description of the algorithm.

\item  void {\bf declareCreatorAlgorithm}(std::string Name, 
			       int Ninput=0, 
			       std::string doc="NONE");

declares to the DB an available algorithm to produce histograms at
analysis time. Ninput is the number of input histograms, doc is a short
description of the algorithm.


\item  int {\bf getAlgorithmNpar}(std::string AlgName);

gets number of parameters needed by algorithm AlgName


\item std::string {\bf getAlgParName}(std::string AlgName,
		       int Ipar);

gets the name of parameter Ipar (starting from 1) of algorithm AlgName

\item  OnlineHistPage* {\bf getPage}(std::string Name, std::string Folder="");

gets a pointer to an OnlineHistPage object, to create a new page (in this case
Folder must be specified) or view/edit an existing one


\item  OnlineHistogram* {\bf getHistogram}(std::string Name,
				std::string Page="\_NONE\_");

gets a pointer to an OnlineHistogram object
that can be used to view/edit an histogram record. If Page
is specified, the default display options for the histogram are those
associated to the page (if available). 


\item int {\bf getHistogramsWithAnalysis}(std::vector$<$OnlineHistogram*$>$\&
list);

gets the list of histograms on which some check analysis has to be
performed. Returns the number of histograms found. list must be
created by the user.

\item int {\bf getAnalysisHistograms}(std::vector$<$OnlineHistogram*$>$\& list);

gets the list of histograms that have to be produced by analysis
task. Returns the number of histograms found. list must be
created by the user.

\item int {\bf getHistogramsBySubsystem}(std::string
SubSys,std::vector$<$OnlineHistogram*$>$\& list);

gets the list of histograms related to subsystem SubSys. Returns the
number of histograms found. list must be created by the user.

\item  int {\bf getHistogramsByTask}(std::string
Task,std::vector$<$OnlineHistogram*$>$\& list);

  gets the list of histograms related to task Task. Returns the
number of histograms found. list must be created by the user.

\item  int {\bf getHistogramsByPage}(std::string Page,std::vector$<$OnlineHistogram*$>$\& list);

  gets the list of histograms displayed on page Page. Returns the
number of histograms found. list must be created by the user.

\item  int {\bf getHistogramsBySet}(std::string SetName,std::vector$<$OnlineHistogram*$>$\& list);\\
int {\bf getHistogramsBySet}(const OnlineHistogram\& Set,std::vector$<$OnlineHistogram*$>$\& list);

  gets the list of histograms in a Set. Returns the
number of histograms found. list must be created by the user.

\end{list}


\subsection{OnlineHistogram class}

OnlineHistogram objects are instantiated within an
OnlineHistDB object, i.e. a DB transaction, through the {\it
getHistogram} method.

\begin{list}{$\bullet$}{}
\item   std::string {\bf name}();

full histogram name
\item   std::string {\bf page}();

page on which histogram is displayed (if specified)
\item void {\bf setPage}(std::string Page);

set page on which histogram is displayed (reload display options if needed)


\item   std::string {\bf hid}();

internal histogram ID (equivalent to hsid()/ihs())
\item   int {\bf hsid}();

internal histogram set ID
\item   std::string {\bf dimServiceName}();

name of the DIM service that is currently publishing the histogram 
\item   int {\bf nhs}();

number of histograms in set
\item   int {\bf ihs}();

position of this histogram in set (starting from 1)
\item   std::string {\bf  hstype}();

histogram type (``1D'', ``2D'' or ``PROF'')
\item   std::string {\bf hstitle}();

histogram set title
\item   std::string {\bf subtitle}();

subtitle 
\item   std::string {\bf task}();

task name
\item   std::string {\bf algo}();

algorithm name

\item   std::string {\bf descr}();

short description of the histogram
\item   std::string {\bf doc}();

link to a more extensive documentation

\item   int {\bf creation}();

creation date, as a unix timestamp
\item   int {\bf obsoleteness}();

if the histogram is no more in use, returns the end--of--validity date
as a unix timestamp, otherwise returns 0. 
\end{list}

Methods for display options:
\begin{list}{$\bullet$}{}
\item typedef enum { NONE, SET, HIST, HISTPAGE } DisplayOptionMode;\\
     DisplayOptionMode {\bf domode}();
  
specifies if the display options in this object are: not defined,
associated to the histogram set, associated to the histogram,
associated to the histogram on  page {\it page()}
  
\item  bool {\bf initDisplayOptionsFromSet}(); 

initializes display options associated to this histogram with the
options defined for the histogram set (if available). Returns true on
success.

\item  bool {\bf initHistoPageDisplayOptionsFromSet}(std::string PageName = "\_DEFAULT\_");

initializes display options associated to this histogram on page
PageName (default is {\it page()}) with the
options defined for the histogram set (if available). Returns true on
success.

\item  bool {\bf initHistoPageDisplayOptionsFromHist}(std::string PageName = "\_NONE\_"); 

initializes display options associated to this histogram on page
PageName (default is {\it page()}) with the
options defined for the histogram (if available). Returns true on
success.

\item   bool {\bf setHistoSetDisplayOption}(std::string ParameterName, 
				void* value);

sets a display option for the whole histogram set. The available
parameter names and the corresponding types are listed in section
\ref{dispopts}. Returns true on success.


\item bool {\bf setDisplayOption}(std::string ParameterName, 
			void* value);

sets a display option for the present histogram.
\item  bool {\bf setHistoPageDisplayOption}(std::string ParameterName, 
				 void* value,
				 std::string PageName = ``\_DEFAULT\_'');

sets a display option for the present histogram on page
PageName (default is {\it page()})

\item   bool {\bf getDisplayOption}(std::string ParameterName,
					 void* option);

if display option  ParameterName has been defined, puts its value into
option and returns true. In case different sets of display options
exists for histogram set, histogram and histogram in page, the most
specific one is used.

\end{list}

Methods for analysis options:
\begin{list}{$\bullet$}{}
\item   int {\bf  nanalysis}();

number of analysis to be performed on the histogram set
\item   const std::vector$<$int$>$\& {\bf anaId}();

vector of length {\it  nanalysis}() containing the analysis internal IDs
\item   const std::vector$<$std::string$>$\& {\bf anaName}();

vector of length {\it  nanalysis}() containing the analysis algorithm names
\item   boolean {\bf isAnaHist}();

true if the histogram is produced at analysis level
\item  int {\bf declareAnalysis}(std::string Algorithm, 
		      std::vector $<$float$>$* warningThr=NULL,  
		      std::vector $<$float$>$* alarmThr=NULL, 
		      int instance=1); 

declare an analysis to be performed on the histogram set. If the algorithm
requires some parameters, the warning and alarm values must be
specified as vectors of floats and will be set for all histograms in
set (then, you can specify values for single histograms with the {\it
setAnalysis} method. 
You can create more than one analysis
with the same algorithm by using instance$>$1. If the analysis
identified by Algorithm and instance already exists, parameters are
updated. Returns the internal analysis ID.

\item   void {\bf setAnalysis}(int AnaID, \\
		   std::vector$<$float$>$* warningThr = NULL, 
		   std::vector$<$float$>$* alarmThr = NULL);

updates parameters for analysis with ID AnaID (for this histogram only)
\item   void {\bf getAnaSettings}(int AnaID,
		      std::vector$<$float$>$* warn, 
		      std::vector$<$float$>$* alarm); 

gets parameters for analysis with ID AnaID
\item   void {\bf maskAnalysis}(int AnalysisAnaID,bool Mask=true); 

masks analysis with ID AnaID. Use Mask=false to unmask
\end{list}



\subsection{OnlineHistPage class}

\begin{list}{$\bullet$}{}
\item   const std::string\& {\bf name}();

page name
\item   const std::string\& {\bf folder}();

page folder name
\item   int {\bf nh}();

number of histograms on page
\item   const std::string\& {\bf doc}();

short page description
\item void {\bf setDoc}(std::string Doc);

set short page description
\item   void {\bf setFolder}(std::string Folder);

set page folder name. If not existing, page folder is created.
\item   void {\bf declareHistogram}(OnlineHistogram* h,
		    float Cx,
		    float Cy,
		    float Sx,
		    float Sy,
		    int instance=1);

adds or updates an histogram on the page. Use instance$>$1 to use the
same histogram more than once. 

\item   void {\bf removeHistogram}(OnlineHistogram* h, unsigned int instance=1);

remove histogram from page
\item   const std::vector$<$OnlineHistogram*$>$\& { \bf hlist}();

returns the list of histograms on this page
\item   bool {\bf getHistLayout}(OnlineHistogram* h,
		     float \&Cx,
		     float \&Cy,
		     float \&Sx,
		     float \&Sy,
		     unsigned int instance=1);


get the layout of given histogram 
\item void {\bf save}();

actually sends to DB the present page layout. This is automatically
called by  {\it OnlineHistDB::commit}, but you may need to call it
explicitly if you want to update page--specific histogram options
before commit 
\end{list}


\section{Code Example}

\begin{verbatim}
#include <OnlineHistDB/OnlineHistDB.h>
int main ()
{
\end{verbatim}
Open DB transaction:
\begin{verbatim}
 std::string password="ask_to_DB_manager";
 OnlineHistDB *HistDB = new OnlineHistDB(password);
\end{verbatim}
Declare the features of your task and your histograms
\begin{verbatim}
 HistDB->declareTask("EXAMPLE","MUON","GAS","",true,true,false);

 string ServiceName="H1DnodeMF001_EXAMPLE_01/SafetyCheck/Trips";
 HistDB->declareHistByServiceName(ServiceName);
 ServiceName="H1DnodeMF001_EXAMPLE_01/SafetyCheck/Trips after use of CRack";
 HistDB->declareHistByServiceName(ServiceName);
 ServiceName="H2DnodeMF001_EXAMPLE_01/OccupancyMap/Hit Map_$$_Region M1R1";
 HistDB->declareHistByServiceName(ServiceName); 
 ServiceName="H2DnodeMF001_EXAMPLE_01/OccupancyMap/Hit Map_$$_Region M1R2";
 HistDB->declareHistByServiceName(ServiceName);
 ServiceName="H2DnodeMF001_EXAMPLE_01/OccupancyMap/Hit Map_$$_Region M3R1";
 HistDB->declareHistByServiceName(ServiceName);
 HistDB->commit(); // needed to send histogram buffer to DB
\end{verbatim}
Now declare an histogram to be produced at analysis level by some
algorithm, and an automatic check to be performed on it (the
declaration of algorithms should be done by the developers of analysis
library)  
\begin{verbatim}
 HistDB->declareCreatorAlgorithm("Subtraction",2,
                                 "bin-by-bin subtraction");
 std::vector<OnlineHistogram*> sources;
 sources.push_back(HistDB->getHistogram
                      ("EXAMPLE/SafetyCheck/Trips"));
 sources.push_back(HistDB->getHistogram
                      ("EXAMPLE/SafetyCheck/Trips after use of CRack"));
 OnlineHistogram* htrips=HistDB->declareAnalysisHistogram("Subtraction",
                    "Trips due to CRack", sources);


 std::string mypar[1]={"Max"};
 HistDB->declareCheckAlgorithm("CheckMax",1,mypar,
                           "Checks all bins to be smaller than Max");
 
 if (htrips) {
   std::vector<float> warn(1,100.);
   std::vector<float> alarm(1,500.);
   htrips->declareAnalysis("CheckMax", &warn, &alarm);
 } 
\end{verbatim}
Now create a page and edit the display options of its
histograms. Start by setting options for the whole histogram set
\begin{verbatim}
 OnlineHistogram* h1=HistDB->getHistogram
           ("EXAMPLE/OccupancyMap/Hit Map_$$_Region M1R1");
 OnlineHistogram* h2=HistDB->getHistogram
           ("EXAMPLE/OccupancyMap/Hit Map_$$_Region M1R2");
 OnlineHistogram* h3=HistDB->getHistogram
           ("EXAMPLE/OccupancyMap/Hit Map_$$_Region M3R1");

 OnlineHistPage* pg=HistDB->getPage("Example Page","Examples");
 pg->declareHistogram(h1,0. ,0. ,0.5,0.5);
 pg->declareHistogram(h2,0. ,0.5,0.5,0.5);
 pg->declareHistogram(h3,0. ,0. ,0.5,1. );
 pg->save(); // // needed to actually send page definition to DB

 int lc=2, fs=7, fc=3;
 float ymax=20000.;
 h1->setHistoSetDisplayOption("LINECOLOR",(void*) &lc);
 h1->setHistoSetDisplayOption("FILLSTYLE",(void*) &fs);
 h1->setHistoSetDisplayOption("FILLCOLOR",(void*) &fc); 
 h1->setHistoSetDisplayOption("YMAX",(void*) &ymax); 
 \end{verbatim}
to set more specific display options for single histograms or
histograms in a given page, remember to initialize the option set if
you want to keep the less specific options
\begin{verbatim}
 lc=3;
 h2->initDisplayOptionsFromSet();
 h2->setDisplayOption("LINECOLOR",(void*)  &lc);

 fc=4;
 h3->initHistoPageDisplayOptionsFromSet("Example Page");
 h3->setHistoPageDisplayOption("FILLCOLOR",(void*) &fc);
\end{verbatim}
finally commit changes to the DB
\begin{verbatim}
 HistDB->commit();
 delete HistDB;
}
\end{verbatim}


\begin{thebibliography}{99}
\bibitem{genhisto}
LHCb Commissioning Group, ``Histogramming Framework'', EDMS 748834
\bibitem{dbdesign}
``Histogram DB and Analysys Tools for Online Monitoring'', EDMS 774740

\end{thebibliography}


\end{document}
