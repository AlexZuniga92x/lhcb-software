\item    inline std::string {\bf identifier}() const ;

 full histogram unique identifier Taskname/AlgorithmName/HistogramName


\item    inline std::string {\bf page}() const ;

 full path name of the page to which this histogram object is attached


\item    inline int {\bf instance}() const ;

 counter (starting from 1) to distinguish several instances of the same histogram on the same page


\item    inline std::string {\bf dimServiceName}() const  ;

 name of the DIM service that is currently publishing the histogram


\item    inline int {\bf hsid}() const ;

 internal histogram set ID


\item    inline int {\bf ihs}() const ;

 position of this histogram in set (starting from 1). Can be larger than nhs() if some histogram in the set has been deleted


\item    inline std::string {\bf hid}() const ;

 internal histogram ID (equivalent to hsid()/ihs())


\item    inline int {\bf nhs}() const ;

 number of histograms in set


\item    inline std::string {\bf hstype}() const ;

 histogram type (``H1D'', ``H2D'', ``P1D'',``P2D'' or ``CNT'' )


\item    inline OnlineHistDBEnv::HistType {\bf type}() const ;

 histogram type (``H1D'', ``H2D'', ``P1D'',``P2D'' or ``CNT'' )


\item    inline int {\bf dimension}() ;

 histogram dimension (1 or 2 for histograms, 0 for counters)


\item    inline std::string {\bf hname}() const ;

 histogram name


\item    inline std::string {\bf htitle}() const ;

 standard DB histogram title (not necessarly equal to the published ROOT title)


\item    inline std::string {\bf hsname}() const ;

 histogram set name


\item    inline std::string {\bf subname}() const ;

 subname


\item    inline std::string {\bf task}() const ;

 task name


\item    inline std::string {\bf algorithm}() const ;

 algorithm name


\item    std::string {\bf descr}() const ;

 short description of the histogram


\item    std::string {\bf doc}() const ;

 link to a more extensive documentation


\item    int {\bf creation}() const ;

 creation date, as a unix timestamp


\item    int {\bf obsoleteness}() const ;

 if the histogram is no more in use, returns the end--of--validity date
 as a unix timestamp, otherwise returns 0.


\item    bool {\bf setPage}(std::string FullPathPageName,\\\mbox{}~~~~~~~~~
	       int Instance=1);

 sets page on which histogram is displayed (reload display options if needed). Histogram has to be already
 been attached to the page through OnlineHistPage::declareHistogram()


\item    void {\bf unsetPage}();


 unsets page associated to histogram object


\item    bool {\bf setDimServiceName}(std::string DimServiceName);


 sets the DIM service name that is currently publishing the histogram. Returns true on success


\item    void {\bf dump}();


 dumps histogram data


\item    void {\bf dumpDisplayOptions}();


 dumps histogram display options


\item    int {\bf nPageInstances}();


 number of instances of this histogram on any page


\item    int {\bf nThisPageInstances}();


 number of instances of this histogram on its current page


\item    OnlineHistTask* {\bf getTask}();


 gets associated task object


\item   typedef enum { NONE, SET, HIST, HISTPAGE } DisplayOptionMode;
\item    DisplayOptionMode {\bf domode}() const ;

 specifies if the display options in this object are: not defined,
 associated to the histogram set, associated to the histogram,
 associated to the histogram on  page {\it page()}


\item    bool {\bf getDisplayOption}(std::string ParameterName,\\\mbox{}~~~~~~~~~
			void* option);

 gets the current value of display option ParameterName, returning false if
 the option is unset.
 option must be a pointer to a variable allocated by the user, of type 
 std::string, int (signed) or float depending on the option type.
 The available option names and the corresponding types are listed 
 in section \ref{dispopts}.


\item    bool {\bf isSetDisplayOption}(std::string ParameterName);


 checks is display option is set


\item    virtual bool {\bf setDisplayOption}(std::string ParameterName,\\\mbox{}~~~~~~~~~
			void* option);

 sets a display option (in the current display mode). Change is sent to the DB only 
 after a call to saveDisplayOptions()


\item    virtual bool {\bf unsetDisplayOption}(std::string ParameterName);


 unsets a display option (in the current display mode)


\item    virtual bool {\bf saveDisplayOptions}();


 saves current display options.  This is the recommended saving method, that
 chooses the most appropriate display option mode 
 (Histogram set, Histogram, or Histogram on Page) calling one of the next methods.


\item    virtual bool {\bf saveHistoSetDisplayOptions}();


 saves current display options for the whole histogram set


\item    virtual bool {\bf saveHistDisplayOptions}();


 saves current display options for the present histogram.


\item    virtual bool {\bf saveHistoPageDisplayOptions}(std::string Page = "\_DEFAULT\_",\\\mbox{}~~~~~~~~~
					   int Instance=-1);

 attaches current display options to instance Instance of the present histogram on page 
 Page (default values for Page, Instance are {\it page()},{\it instance()}  )


\item    bool {\bf initHistDisplayOptionsFromSet}(); 


 initializes display options associated to this histogram with the
 options defined for the histogram set (if available). Returns true on
 success.


\item    bool {\bf initHistoPageDisplayOptionsFromSet}(std::string FullPathPageName = "\_DEFAULT\_",\\\mbox{}~~~~~~~~~
					  int Instance=-1);

 initializes display options associated to this histogram on page
 FullPathPageName (default is {\it page()}) with the
 options defined for the histogram set (if available). Returns true on
 success.


\item    bool {\bf initHistoPageDisplayOptionsFromHist}(std::string FullPathPageName = "\_DEFAULT\_",\\\mbox{}~~~~~~~~~
					   int Instance=-1);

 initializes display options associated to this histogram on page
 FullPathPageName (default is {\it page()}) with the
 options defined for the histogram (if available). Returns true on
 success.


\item    bool {\bf getCreationDirections}(std::string \&Algorithm,\\\mbox{}~~~~~~~~~
			     std::vector$<$std::string$>$ \&source\_list,\\\mbox{}~~~~~~~~~
			     std::vector$<$float$>$ \&parameters);

 for analysis histogram, get the directions for creating histogram


\item    int {\bf nanalysis}() const ;

 number of analysis to be performed on the histogram set


\item    void {\bf getAnalyses}(std::vector$<$int$>$\& anaIDs,\\\mbox{}~~~~~~~~~
		   std::vector$<$std::string$>$\& anaAlgs) ;

 get analysy description as vectors of length {\it  nanalysis}() containing 
 the analysis internal IDs and the analysis algorithm names


\item    bool {\bf isAnaHist}() const ;

 true if the histogram is produced at analysis level


\item    int {\bf declareAnalysis}(std::string Algorithm,\\\mbox{}~~~~~~~~~ 
		      std::vector$<$float$>$* warningThr=NULL,\\\mbox{}~~~~~~~~~ 
		      std::vector$<$float$>$* alarmThr=NULL,\\\mbox{}~~~~~~~~~ 
		      int instance=1);

 declare an analysis to be performed on the histogram set. If the algorithm
 requires some parameters, the warning and alarm values must be
 specified as vectors of floats and will be set for all histograms in
 set (then, you can specify values for single histograms with the {\it
 setAnalysis} method. 
 You can create more than one analysis
 with the same algorithm by using instance $>$ 1. If the analysis
 identified by Algorithm and instance already exists, parameters are
 updated. Returns the internal analysis ID.


\item    bool {\bf setAnalysis}(int AnaID,\\\mbox{}~~~~~~~~~ 
		   std::vector$<$float$>$* warningThr=NULL,\\\mbox{}~~~~~~~~~ 
		   std::vector$<$float$>$* alarmThr=NULL);

 updates parameters for analysis with ID AnaID (for this histogram only). Returns true on success


\item    bool {\bf getAnaSettings}(int AnaID,\\\mbox{}~~~~~~~~~
		      std::vector$<$float$>$* warn,\\\mbox{}~~~~~~~~~ 
		      std::vector$<$float$>$* alarm,\\\mbox{}~~~~~~~~~
		      bool \&mask);

 gets parameters for analysis with ID AnaID. Returns true on success


\item    bool {\bf maskAnalysis}(int AnaID,\\\mbox{}~~~~~~~~~
		    bool Mask=true);

 masks analysis with ID AnaID. Use Mask=false to unmask. Returns true on success


